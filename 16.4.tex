本节,我们证明两个十分重要的定理. 它们分别刻画了紧空间上的连续函数空间的紧子集与稠密子集.

\mysubsection{Arzelà-Ascoli 定理}

\begin{definition}
    设 $X$ 为集合,$(Y,\rho)$ 为度量空间. 设 $\mathscr{F}=\set{f:X\to Y}$ 为一族函数.

    称 $\mathscr{F}$ 一致有界,若所有 $f\in\mathscr{F}$ 的值域之并在 $Y$ 中有界. 即
$$
V\triangleq\set{f(x)|x\in X,f\in\mathscr{F}}
$$

    在 $Y$ 中有界.

    称 $\mathscr{F}$ 完全有界,若 $V$ 是 $Y$ 中的完全有界集.
\end{definition}

\begin{hint}
    若 $Y=\RR^n$ 或 $\mathbb{C}^n$,则 $A\subset Y$ 有界 $\iff A$ 完全有界.

    从而在此时函数族 $\mathscr{F}$ 一致有界与完全有界等价.

    但若 $Y$ 为无穷维赋范线性空间,则完全有界的概念严格强于有界. 例如 $Y=C[a,b]$.
\end{hint}

\begin{definition}
    设 $(X,d),(Y,\rho)$ 均为度量空间,$\mathscr{F}=\set{f:X\to Y}$ 为一族函数.

    称 $\mathscr{F}$ 等度连续,若
$$
\forall\eps>0,\exists\delta>0,\forall x_1,x_2\in X,d(x_1,x_2)<\delta\implies\forall f\in\mathscr{F},\rho(f(x_1),f(x_2))<\eps
$$
\end{definition}

\begin{example}
    $\mathscr{F}=\set{x^n|n\ge 1},X=[0,1]$.

    则 $\mathscr{F}$ 一致有界,但不等度连续.
\end{example}

\begin{example}
    $\mathscr{F}=\set{\sin nx|n\in\mathbb{N}}$ 在任何区间 $[a,b]$ 上均不等度连续.
\end{example}

\begin{example}
    若 $\mathscr{F}$ 是一族从 $\RR$ 到 $\RR$ 的映射,且
$$
\exists L>0,\forall f\in\mathscr{F},\forall x,y\in\RR,\abs{f(x)-f(y)}\le L\abs{x-y}
$$

    则 $\mathscr{F}$ 等度连续.
\end{example}

如上的概念与一致收敛性有密切的连续:

\begin{lemma}
    设 $K$ 为紧度量空间,$Y$ 为完备度量空间.

    设 $f_n\in C(K;Y)$. 若 $\set{f_n}$ 在 $K$ 上一致收敛,则 $\set{f_n}$ 在 $K$ 上完全有界且等度连续.
\end{lemma}
\begin{proof}
    设 $f_n\rightrightarrows f:K\to Y$. 则 $f$ 也连续.

    \begin{itemize}
        \item 完全有界:
        
        由 $f_n,f$ 连续且 $K$ 紧知 $f_n(K),f(K)$ 均为 $Y$ 中的紧集. 从而完全有界. 由 $f(K)$ 完全有界知
$$
\forall\eps>0,\exists x_1,\cdots,x_k\in K,f(K)\subset\bigcup_{j=1}^kB\left(f(x_j),\frac{\eps}{2}\right)
$$

        由 $f_n\rightrightarrows f$ 知
$$
\begin{aligned}
    \exists N\in\mathbb{N},\forall n\ge N,&\forall x\in K,\abs{f_n(x)-f(x)}<\frac{\eps}{2}\\
    \implies&f_n(K)\subset\bigcup_{j=1}^kB(f(x_j),\eps)
\end{aligned}
$$

        故 $\bigcup\limits_{n\ge N}f_n(K)$ 完全有界.

        由 $f_1(K)\cup\cdots\cup f_N(K)$ 是有限个完全有界集之并,知其完全有界. 故
$$
V\triangleq\bigcup_{n=1}^\infty f_n(K)=\bigcup_{n=1}^Nf_n(K)\cup\bigcup_{n\ge N}f_n(K)
$$

        完全有界. 即 $\set{f_n}$ 完全有界.
        
        \item 等度连续:
        
        任取 $\eps>0$. 由 $f_n\rightrightarrows f$ 知
$$
\exists N\in\mathbb{N},\forall n\ge N,\forall x\in K,\abs{f_n(x)-f(x)}<\frac{\eps}{3}
$$

        由 $f_1,\cdots,f_N,f$ 在紧集 $K$ 上连续知它们一致连续. 从而
$$
\begin{aligned}
    \exists\delta>0,\forall x,x'\in K,d(x,x')<\delta\implies&\rho(f_j(x),f_j(x'))<\frac{\eps}{3},1\le j\le N\\
    &\rho(f(x),f(x'))<\frac{\eps}{3}
\end{aligned}
$$

        则对 $n\ge N$ 有
$$
\begin{aligned}
    \forall x,x'\in K,d(x,x')<\delta\implies&\rho(f_n(x),f_n(x'))\\
    \le&\rho(f_n(x),f(x))+\rho(f(x),f(x'))+\rho(f(x'),f_n(x'))\\
    <&\frac{\eps}{3}+\frac{\eps}{3}+\frac{\eps}{3}=\eps
\end{aligned}
$$

        即 $\set{f_n}$ 等度连续.
    \end{itemize}
\end{proof}

现在我们可以陈述 Arzelà-Ascoli 定理.

\begin{theorem}[Arzelà-Ascoli]
    设 $\mathscr{F}$ 是一族从紧度量空间 $K$ 到完备度量空间 $Y$ 的函数. 则

    $\forall\set{f_n}\subset\mathscr{F}$,存在子列 $n_k$ 使得 $\set{f_{n_k}}$ 一致收敛 $\iff\mathscr{F}$ 完全有界且等度连续.
\end{theorem}
\begin{proof}
    $\implies$:反证,设 $\mathscr{F}$ 不完全有界,即 $V=\set{f(x)|x\in K,f\in\mathscr{F}}$ 不完全有界.
    
    即存在 $\eps_0>0$ 使得 $V$ 不存在有限 $\eps_0$-网.

    则可以构造 $f_n\in\mathscr{F},x_n\in K$ 满足
$$
\rho(f_n(x_n),f_m(x_m))\ge\eps_0,\forall n\ne m
$$

    这说明 $\set{f_n}\subset\mathscr{F}$ 不完全有界,且 $\set{f_n}$ 的任何子列也不完全有界. 从而由引理知其不可能一致收敛.

    反证,设 $\mathscr{F}$ 不等度连续.

    则存在 $\eps_0>0$ 以及 $f_n\in\mathscr{F},x_n,x_n'\in K$ 满足
$$
\begin{cases}
    d(x_n,x_n')<\dfrac{1}{n}\\
    \rho(f_n(x_n),f_n(x_n'))\ge\eps_0
\end{cases},\forall n\in\mathbb{N}
$$

    则 $\set{f_n}$ 的任何子列都不等度连续. 从而不可能一致收敛.

    $\impliedby$:下设 $\mathscr{F}$ 完全有界且等度连续.

    设 $\set{f_n}\subset\mathscr{F}$. 由 $K$ 紧知:存在可数集 $D\subset K$ 使得 $\overline{D}=K$.

    可以这样构造:$\forall n\in\mathbb{N}$ 存在 $K$ 的有限 $\dfrac{1}{n}$-网 $D_n$. 令 $D=\bigcup\limits_{n\in\mathbb{N}}D_n$ 即可.

    现在对 $\set{f_n}$ 利用 Cantor 对角线法,可以选出子列 $\set{f_{n_k}}$ 使得 $\forall x\in D$ 有 $f_{n_k}(x)\to f_\infty(x),k\to\infty$.

    这里能选出子列的原因是 $\forall x\in D,\set{f_n(x)}$ 完全有界且 $Y$ 完备.

    下证:$\set{f_{n_k}}$ 一致收敛.

    任取 $\eps>0$. 由 $\mathscr{F}$ 等度连续知
$$
\exists\delta>0,\forall f\in\mathscr{F},\forall x,x'\in K,d(x,x')<\delta\implies\rho(f(x),f(x'))<\frac{\eps}{4}
$$

    由 $D\subset K$ 知 $D$ 完全有界. 从而其存在有限 $\dfrac{\delta}{2}$-网,记为 $\set{x_1,\cdots,x_m}\subset D$.

    由 $f_{n_k}(x_i)\to f_\infty(x_i),i=1,\cdots,m$ 知
$$
\exists K\in\mathbb{N},\forall k\ge K,\forall 1\le i\le m,\rho(f_{n_k}(x_i),f_\infty(x_i))<\frac{\eps}{4}
$$

    现任取 $k,l\ge K$. 则对 $\forall x\in K$ 可以选出 $x_i$ 使得 $d(x,x_i)<\delta$. 从而
$$
\begin{aligned}
    \rho(f_{n_k}(x),f_{n_l}(x))\le&\rho(f_{n_k}(x),f_{n_k}(x_i))+\rho(f_{n_k}(x_i),f_\infty(x_i))\\
    &+\rho(f_\infty(x_i),f_{n_l}(x_i))+\rho(f_{n_l}(x_i),f_{n_l}(x))\\
    <&\frac{\eps}{4}+\frac{\eps}{4}+\frac{\eps}{4}+\frac{\eps}{4}=\eps
\end{aligned}
$$

    即 $\set{f_{n_k}}$ 满足 Cauchy 准则,从而一致收敛.
\end{proof}