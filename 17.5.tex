一元含参积分的理论可以很方便地推广到重积分情形.

以下罗列一些基本事实.

\mysubsection{含参(正常)重积分}

含参重积分的基本框架:

设 $X\subset\RR^m$ 可测,$Y\subset\RR^n$. 设 $f:X\times Y\to\RR$ 满足
$$
\forall y\in Y,F(y)\triangleq\int_Xf(x,y)\dd x
$$

可积.

\begin{property}
    设 $X\subset\RR^m$ 为可测紧集,$Y\subset\RR^n$ 紧. 设 $f:X\times Y\to\RR$ 连续. 则
$$
F(y)\triangleq\int_Xf(x,y)\dd x
$$

    在 $Y$ 上连续.
\end{property}

\begin{property}
    设 $X\subset\RR^m$ 可测,$Y\subset\RR^n$ 为开集. 设 $f:X\times Y\to\RR$ 连续,$\dfrac{\partial f}{\partial y}(x,y)$ 存在且连续. 则
$$
F(y)\triangleq\int_Xf(x,y)\dd x\in C^{(1)}(Y)
$$

    且
$$
\forall y\in Y,F'(y)=\int_X\pard{f}{y}(x,y)\dd x
$$
\end{property}

\begin{property}
    设 $X\subset\RR^m,Y\subset\RR^n$ 均为可测紧集. 设 $f:X\times Y\to\RR$ 连续. 则
$$
\int_Y\int_X f(x,y)\dd x\dd y=\int_X\int_Yf(x,y)\dd y\dd x
$$
\end{property}

\mysubsection{含参广义积分}

本部分与一元情形的理论平行. 我们不再重复它们,仅以两个例子来示例.

\begin{example}
    $F(\lambda)\triangleq\displaystyle\iint\limits_{\RR^2}e^{-\lambda(x^2+y^2)}\dd x\dd y$

    其为含参广义二重积分,$\lambda$ 为参数. 易见 $F(\lambda)$ 的定义域为 $\set{\lambda>0}$.

    对 $E\subset\set{\lambda>0}$,称 $F$ 在 $E$ 上一致收敛,若
$$
\forall\eps>0,\exists M>0,\forall\lambda\in E,\iint\limits_{x^2+y^2\ge M^2}e^{-\lambda(x^2+y^2)}\dd x\dd y<\eps
$$

    由极坐标变换易证:若 $\inf E>0$,则 $F$ 在 $E$ 上一致收敛.
    
    反之若 $0\in\overline{E}$,则 $F$ 在 $E$ 上非一致收敛.
\end{example}

\begin{example}
    $F(y)\triangleq\displaystyle\int_{B(0,1)}\frac{\abs{x-y}}{(1-\abs{x})^\alpha}\dd x,x,y\in\RR^n$.

    当 $\abs{x}\to 1$ 时 $f(x,y)\triangleq\dfrac{\abs{x-y}}{(1-\abs{x})^\alpha}\to\infty$. 所以上面的积分为广义积分.

    由极坐标变换易证 $\forall y\in\RR^n,F(y)$ 收敛 $\iff\alpha<1$.

    现固定 $\alpha<1$,则对 $\forall E\subset\RR^n$ 有界,$F$ 在 $E$ 上一致收敛.
\end{example}

\mysubsection{奇点随参数变化的广义积分}

到目前位置,我们讨论的广义积分均有如下的特点:被积函数的奇点是固定的,不随参数而变化. 本节我们讨论奇点随参数变化的广义积分. 我们先来看一个典型的例子.

\begin{example}
    设 $X\subset\RR^3$ 可测,在 $X$ 上有电荷分布,其密度函数为 $\mu:X\to\RR$. 设 $\mu$ 在 $X$ 上有界且可积.

    则由这些电荷在点 $y\in\RR^3$ 处产生的电势为(忽略常数)
$$
U(y)=\int_X\frac{\mu(x)\dd x}{\abs{x-y}}
$$

    容易观察到 $U$ 在 $\RR^3\setminus\overline{X}$ 上是 $C^{\infty}$ 光滑的.

    但当 $y\in\overline{X}$ 时,$\dfrac{\mu(x)}{\abs{x-y}}\to\infty(x\to y)$,从而是一个广义积分,且奇点为 $y$.

    从而当 $y$ 在 $X$ 中移动时,奇点也随之移动. 这是在我们之前的所有讨论中都没有出现过的. 以下我们来详细地讨论 $U$ 的定义域及其正则性.
\end{example}

\begin{property}
    \begin{enumerate}
        \item $U$ 的定义域为 $\RR^3$,且 $U\in C^{(1)}(\RR^3)$.
        
        \item $U$ 在 $\RR^3\setminus\overline{X}$ 上,$U$ 为调和函数.
    \end{enumerate}
\end{property}
\begin{proof}
    已知 $U$ 在 $\RR^3\setminus\overline{X}$ 上可定义且 $C^\infty$ 光滑. 下证 $U$ 在 $\overline{X}$ 上可定义.

    任取 $y\in\overline{X}$. 设 $M=\sup\set{\mu(x)|x\in X}$. 任取 $\eps>0$,则一方面有
$$
\int_{X\setminus B(y,\eps)}\frac{\abs{\mu(x)}\dd x}{\abs{x-y}}\le\frac{M}{\eps}\mu(X)
$$

    另一方面有
$$
\begin{aligned}
    \int_{X\cap B(y,\eps)}\frac{\abs{\mu(x)}\dd x}{\abs{x-y}}&\le\int_{X\cap B(y,\eps)}\frac{M\dd x}{\abs{x-y}}\le\int_{B(y,\eps)}\frac{M\dd x}{\abs{x-y}}\\
    &=\int_0^\eps\int_{S^2}\frac{M}{r}r^2\dd\sigma\dd r=4\pi\int_0^\eps Mr\dd r\\
    &=2\pi M\eps^2
\end{aligned}
$$

    综上,$\displaystyle\int_X\frac{\mu(x)\dd x}{\abs{x-y}}$ 在 $\overline{X}$ 上绝对收敛,从而 $U$ 的定义域为 $\RR^3$.

    下证 $U$ 在 $y_0\in\overline{X}$ 处连续. 任取 $\eps>0$,设 $\abs{y-y_0}<\eps$. 则
$$
\begin{aligned}
    U(y)-U(y_0)=&\int_X\mu(x)\left(\frac{1}{\abs{x-y}}-\frac{1}{\abs{x-y_0}}\right)\dd x\\
    =&\int_{X\setminus B(y,2\eps)}\mu(x)\left(\frac{1}{\abs{x-y}}-\frac{1}{\abs{x-y_0}}\right)\dd x\\
    &+\int_{X\cap B(y,2\eps)}\mu(x)\left(\frac{1}{\abs{x-y}}-\frac{1}{\abs{x-y_0}}\right)\dd x
\end{aligned}
$$

    其中
$$
\begin{aligned}
    &\abs{\int_{X\setminus B(y,2\eps)}\mu(x)\left(\frac{1}{\abs{x-y}}-\frac{1}{\abs{x-y_0}}\right)\dd x}\\
    \le&\int_{X\setminus B(y,2\eps)}\frac{M\abs{\abs{x-y}-\abs{x-y_0}}}{\abs{x-y}\abs{x-y_0}}\dd x\\
    \le&\int_{X\setminus B(y,2\eps)}\frac{M\abs{y-y_0}}{\eps^2}\dd x\\
    \le&\frac{M\abs{y-y_0}}{\eps^2}\mu(X)
\end{aligned}
$$

    而
$$
\begin{aligned}
    &\int_{X\cap B(y,2\eps)}\mu(x)\left(\frac{1}{\abs{x-y}}-\frac{1}{\abs{x-y_0}}\right)\dd x\\
    \le&\int_{X\cap B(y,2\eps)}M\left(\frac{1}{\abs{x-y}}+\frac{1}{\abs{x-y_0}}\right)\dd x\\
    \le&2\pi M(2\eps)^2+2\pi M(3\eps)^2=26\pi M\eps^2
\end{aligned}
$$

    综上有
$$
\abs{U(y)-U(y_0)}\le\frac{M\abs{y-y_0}}{\eps^2}\mu(X)+26\pi M\eps^2
$$

    进而
$$
\overline{\lim_{y\to y_0}}\abs{U(y)-U(y_0)}\le 26\pi M\eps^2
$$

    由 $\eps$ 的任意性即得 $\abs{U(y)-U(y_0)}\to 0$,即 $U$ 在 $y_0$ 处连续.

    \begin{hint}
        另证:设
$$
U_\eps(y)\triangleq\int_{X\setminus B(y,\eps)}\frac{\mu(x)\dd x}{\abs{x-y}}
$$

        已知 $U_\eps(x)\xrightrightarrows[\eps\to 0]{}U(y)$.

        再证 $U_\eps(y)$ 连续即可.
    \end{hint}

    下证 $U\in C^{(1)}(\RR^3)$. 首先,若 $U$ 可微,形式上应有
$$
U(y)=\int_X\frac{\mu(x)\dd x}{\abs{x-y}}\implies\frac{\partial U}{\partial y_i}(y)\xlongequal{?}\int_X\frac{\mu(x)(x_i-y_i)\dd x}{\abs{x-y}^3},i=1,2,3
$$

    记 $g(x,y)\triangleq\dfrac{\mu(x)(x_i-y_i)}{\abs{x-y}^3}$. 则
$$
\abs{g(x,y)}\le\frac{\mu(x)\abs{x-y}}{\abs{x-y}^3}=\frac{M}{\abs{x-y}^2}
$$

    从而
$$
\int_{X\cap B(y,\eps)}\abs{g(x)}\dd x\le\int_{B(y,\eps)}\frac{M\dd x}{\abs{x-y}^2}=4\pi M\eps
$$

    进而同前可得 $V_i(x)\triangleq\displaystyle\int_X\frac{\mu(x)(x_i-y_i)}{\abs{x-y}^3}\dd x$ 在 $\RR^3$ 上可定义且连续.

    下证 $\dfrac{\partial U}{\partial y_i}$ 存在且 $\dfrac{\partial U}{\partial y_i}=V_i$.

    为此仅需证明:$\forall y\in\RR^3,\forall a\in\RR$ 有
$$
U(y+a e_i)-U(y)=\int_0^aV_i(y+te_i)\dd t
$$

    令 $\phi(y)\triangleq\dfrac{1}{\abs{x-y}}$. 则我们有
$$
\begin{aligned}
    \int_0^aV_i(y+te_i)\dd t&=\int_0^a\int_X\mu(x)\pard{\phi}{y_i}(y+te_i)\dd x\dd t\\
    &=\int_X\mu(x)\int_0^a\pard{\phi}{y_i}(y+te_i)\dd t\dd x\\
    &=\int_X\mu(x)(\phi(y+te_i)-\phi(y))\dd x\\
    &=U(y+ae_i)-U(y)
\end{aligned}
$$

    这里之所以可以交换积分的次序是因为 $g(x,y+te_i)$ 在 $X\times[0,a]$ 上绝对可积,从而可以使用 Fubini 定理.

    综上 $\dfrac{\partial U}{\partial y_i}=V_i\in C(\RR^3)$. 从而 $U\in C^{(1)}(\RR^3)$.

    已知对 $\phi(y)=\dfrac{1}{\abs{y}},y\ne 0$ 有 $\phi$ 在 $\RR^3\setminus\set{0}$ 上光滑,且 $\Delta\phi=0$.

    即 $\phi$ 在 $\RR^3\setminus\set{0}$ 上调和.

    从而在区域 $\RR^3\setminus\overline{X}$ 上有
$$
\Delta U(y)=\int_X\mu(x)\Delta\phi(y-x)\dd x=0
$$

    即 $U$ 在 $\RR^3\setminus\overline{X}$ 上调和.
\end{proof}

最后,我们来看一个进阶的例子.

\begin{example}
    设 $S\subset\RR^3$ 为光滑二维曲面. 设 $I=[0,1]^2,\varphi:I\to S$ 为 $S$ 的参数化.

    设在 $S$ 上有电荷分布,其密度函数为 $\nu:S\to\RR$ 有界连续.

    则由这些电荷在点 $y\in\RR^3$ 处产生的电势为(忽略常数)
$$
U(y)=\int_S\frac{\nu(x)\dd\sigma}{\abs{x-y}}
$$
\end{example}

\begin{property}
    \begin{enumerate}
        \item $U$ 的定义域为 $\RR^3$ 且 $U\in C(\RR^3)$.
        
        \item 在 $\RR^3\setminus S$ 上有 $\Delta U=0$.
    \end{enumerate}
\end{property}
\begin{proof}
    由第一型曲面积分的定义,有
$$
U(y)\int_S\frac{\nu(x)\dd\sigma}{\abs{x-y}}=\int_I\frac{\nu(\varphi(t))G(t)\dd t}{\abs{\varphi(t)-y}}
$$

    其中 $G(t)=\displaystyle\sqrt{\det\varphi'(t)^T\varphi'(t)}$ 是 $I$ 上的连续函数.

    从而 $g(t)\triangleq\nu(\varphi(t))G(t)$ 在 $I$ 上非负且有界连续.
$$
U(y)=\int_I\frac{g(t)\dd t}{\abs{\varphi(t)-y}}
$$

    若 $y\notin S$,则易见 $U$ 在 $\RR^3\setminus S$ 上 $C^\infty$ 光滑.

    下设 $y\in S$. 则 $\exists t_0\in I,y=\varphi(t_0)$. 由有限增量定理有
$$
\exists\eps>0,\forall t\in B(t_0,\eps),\abs{\varphi(t)-y}=\abs{\varphi(t)-\varphi(t_0)}\ge\frac{\norm{\varphi'(t_0)}}{2}\abs{t-t_0}
$$

    从而
$$
\begin{aligned}
    U(y)&=\int_{I\setminus B(t_0,\eps)}\frac{g(t)\dd t}{\abs{\varphi(t)-y}}+\int_{B(t_0,\eps)}\frac{g(t)\dd t}{\abs{\varphi(t)-y}}\\
    &\le\int_{I\setminus B(t_0,\eps)}\frac{g(t)\dd t}{\abs{\varphi(t)-y}}+C'\int_{B(t_0,\eps)}\frac{\dd t}{\abs{t-t_0}}\\
    &<+\infty
\end{aligned}
$$

    即积分存在. 再重复上一例子的证明可知 $U$ 在 $\RR^3$ 上连续.

    若 $y\in\RR^3\setminus S$,则
$$
\Delta U(y)=\int_S\nu(x)\Delta\frac{1}{\abs{x-y}}\dd x=0
$$
\end{proof}