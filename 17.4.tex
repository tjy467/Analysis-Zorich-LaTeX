作为含参积分的另一个重要应用,我们定义卷积并讨论其基本性质. 作为卷积的应用,我们证明可用 $C^\infty$ 光滑函数逼近连续函数.

\mysubsection{概念的引入与定义}

设 $f:\RR\to\RR$ 连续. 我们希望找到一列 $C^(1)$ 光滑的函数来逼近 $f$. 一个可能的做法如下:

对任意 $\delta>0$,定义 $f_\delta:\RR\to\RR$ 为
$$
f_\delta(x)\triangleq\frac{1}{2\delta}\int_{x-\delta}^{x+\delta}f(y)\dd y
$$

首先由不定积分的性质知 $f_\delta\in C^{(1)}(\RR)$ 且
$$
f_\delta'(x)=\frac{1}{2\delta}(f(x+\delta)-f(x-\delta))
$$

另一方面
$$
\abs{f_\delta(x)-f(x)}=\abs{\frac{1}{2\delta}\int_{x-\delta}^{x+\delta}(f(y)-f(x))\dd x}=f(\xi)
$$

其中 $\xi\in(x-\delta,x+\delta)$. 从而确有
$$
f_\delta(x)\to f(x),\forall x\in\RR
$$

若将 $x$ 限定在紧区间 $[a,b]$ 上,由一致连续性有
$$
f_\delta\rightrightarrows f\quad(x\in[a,b])
$$

上面定义的 $f_\delta$ 的几何直观非常简单,即对 $f$ 作平均. 具体来讲,对每个固定的 $x$,我们将 $f$ 在 $[x-\delta,x+\delta]$ 上作平均.

一个非常有用的观察如下:我们可以将如上的求平均表达成一个含参积分的形式,具体如下:

定义 $\Delta_\delta:\RR\to\RR$ 为
$$
\Delta_\delta(x)\triangleq\begin{cases}
    \dfrac{1}{2\delta} & x\in[-\delta,\delta]\\
    0 & \abs{x}>\delta
\end{cases}
$$

\begin{center}
    \begin{tikzpicture}[xscale=3,yscale=1.5]
        \draw[->] (-1.5,0)--(1.5,0) node [right] {$x$};
        \draw[->] (0,-0.5)--(0,2.5) node [right] {$y$};
        \node at (0,0) [below left] {$O$};
        \draw[-,red] (-1,0.5)--(1,0.5) node [right] {$\Delta_1$};
        \draw[-,dashed,red] (-1,0.5)--(-1,0) node [below] {$-1$};
        \draw[-,dashed,red] (1,0.5)--(1,0) node [below] {$1$};
        \draw[-,blue] (-0.5,1)--(0.5,1) node [right] {$\Delta_\frac{1}{2}$};
        \draw[-,dashed,blue] (-0.5,1)--(-0.5,0) node [below] {$-\frac{1}{2}$};
        \draw[-,dashed,blue] (0.5,1)--(0.5,0) node [below] {$\frac{1}{2}$};
        \node at (0,1) [above right] {$1$};
        \draw[-,purple] (-0.25,2)--(0.25,2) node [right] {$\Delta_\frac{1}{4}$};
        \draw[-,dashed,purple] (-0.25,2)--(-0.25,0) node [below] {$-\frac{1}{4}$};
        \draw[-,dashed,purple] (0.25,2)--(0.25,0) node [below] {$\frac{1}{4}$};
    \end{tikzpicture}
\end{center}

则
$$
f_\delta(x)=\frac{1}{2\delta}\int_{x-\delta}^{x+\delta}f(y)\dd y=\int_\RR f(y)\Delta_\delta(x-y)\dd y
$$

称最后的积分为 $f$ 与 $\Delta_\delta$ 的卷积,记为 $f*\Delta_\delta(x)$.

\begin{hint}
    一旦有了这个想法,我们发现不一定要将 $\Delta_\delta$ 取成如上的阶梯函数的样子. 例如我们可以取 $\Delta_\delta$ 如图

    \begin{center}
        \begin{tikzpicture}
            \draw[->] (-1.5,0)--(1.5,0) node [right] {$x$};
            \draw[->] (0,-0.5)--(0,2.5) node [right] {$y$};
            \node at (0,0) [below left] {$O$};
            \draw[-,blue] (-1,0)--(0,2);
            \draw[-,blue] (0,2)--(1,0);
            \node[blue] at (1.5,2) {连续};
        \end{tikzpicture}
        \qquad
        \begin{tikzpicture}
            \draw[->] (-1.5,0)--(1.5,0) node [right] {$x$};
            \draw[->] (0,-0.5)--(0,2.5) node [right] {$y$};
            \node at (0,0) [below left] {$O$};
            \draw[domain=-1:1,blue] plot (\x,{cos(pi*\x r)+1});
            \node[blue,align=center] at (1.5,2) {光滑\\紧支撑};
        \end{tikzpicture}
        \qquad
        \begin{tikzpicture}
            \draw[->] (-1.5,0)--(1.5,0) node [right] {$x$};
            \draw[->] (0,-0.5)--(0,2.5) node [right] {$y$};
            \node at (0,0) [below left] {$O$};
            \draw[domain=-1.5:1.5,blue] plot (\x,{2/(1+\x*\x)});
            \node[blue,align=center] at (2.5,2) {光滑\\在 $\infty$ 处骤降};
        \end{tikzpicture}
    \end{center}

    只要保持 $\Delta_\delta\ge 0,\displaystyle\int_\RR\Delta_\delta(x)\dd x=1$,且主要的积分值来源于 $0$ 的一个邻域. 或更数学化的说:对任意 $0$ 的邻域 $U$ 有
$$
\int_U\Delta_\delta(x)\dd x\to 1\quad(\delta\to 0)
$$
\end{hint}

现在我们可以正式地给出两个函数卷积的定义了.

\begin{definition}
    设 $u,v:\RR\to\RR$. 定义 $u$ 与 $v$ 的卷积为
$$
u*v(x)\triangleq\int_\RR u(y)v(x-y)\dd y
$$

    若 $\forall x\in\RR$ 如上的函数 $u(y)v(x-y)$ 在 $\RR$ 上广义可积.
\end{definition}

\begin{example}
    对上面定义的 $\Delta_\delta$ 以及 $f$ 连续,显然有 $f(y)\Delta(x-y)$ 对 $\forall x\in\RR$ 可积,从而
$$
f*\Delta_\delta(x)=\int_\RR f(x)\Delta_\delta(x-y)\dd y
$$

    可以定义.
\end{example}

以下,我们给出几个卷积可定义的充分条件. 这对我们的初步研究来说已经够用了. 我们首先给出几个定义.

\begin{definition}
    设 $G\subset\RR^n$ 为开集,$f:G\to\RR$. 称 $f$ 在 $G$ 上局部可积,若 $\forall x\in G$,存在 $x$ 的邻域 $U(x)\subset G$ 使得 $f$ 在 $U(x)$ 上可积.
\end{definition}

若 $G=\RR$,则 $f$ 在 $\RR$ 上局部可积 $\iff\forall [a,b]\subset\RR,f\in\mathcal{R}[a,b]$.

\begin{definition}
    定义 $f$ 的支撑为集合 $\set{x\in G|f(x)\ne 0}$ 在 $G$ 中的闭包,记为 $\mathrm{supp}(f)$.
    
    进一步,若 $\mathrm{supp}(f)$ 为紧集,则称 $f$ 在 $G$ 内有紧支撑.
\end{definition}

\begin{example}
    $G=(-1,1),f:G\to\RR,f(x)=1-x^2$.

    则 $\mathrm{supp}(f)=(-1,1)$. 从而 $f$ 在 $G$ 内没有紧支撑.
\end{example}

我们用 $C^{(m)}(G)$ 表示在 $G$ 上 $m$ 阶可微函数的全体.

用 $C^{(m)}_0(G)$ 表示在 $G$ 上 $m$ 阶可微且有紧支撑的函数全体.

在 $G=\RR$ 的情形,我们进一步简化为 $C^{(m)}$ 和 $C^{(m)}_0$.

\begin{property}
    设 $u,v:\RR\to\RR$ 均局部可积,若以下任一条件成立:

    \begin{enumerate}
        \item $\abs{u}^2$ 和 $\abs{v}^2$ 均在 $\RR$ 上可积.
        
        \item $\abs{u}$ 与 $\abs{v}$ 其中之一可积,而另一个有界.
        
        \item $u$ 或 $v$ 有紧支撑.
    \end{enumerate}

    则 $u*v$ 在 $\RR$ 上存在.
\end{property}
\begin{proof}
    \begin{enumerate}
        \item 由 Cauchy-Schwartz 不等式有
$$
\forall x\in\RR,\abs{\int_\RR u(y)v(x-y)\dd y}\le\left(\int_\RR\abs{u(y)}^2\dd y\right)^\frac{1}{2}\left(\int_\RR\abs{v(x-y)}^2\dd y\right)^\frac{1}{2}
$$

        作变量替换 $y'=x-y$ 得 $\displaystyle\int_\RR\abs{v(x-y)}^2\dd y=\int_\RR\abs{v(y)}^2\dd y$

        从而 $u*v(x)$ 存在且
$$
\forall x\in\RR,\abs{u*v(x)}\le\left(\int_\RR\abs{u(y)}^2\dd y\right)^\frac{1}{2}\left(\int_\RR\abs{v(x-y)}^2\dd y\right)^\frac{1}{2}
$$

        \item 不妨设 $\abs{u}$ 可积,$\abs{v}\le M$. 则
$$
\forall x\in\RR,\abs{\int_a^b u(y)v(x-y)\dd y}\le\int_a^b\abs{u(y)}\abs{v(x-y)}\dd y\le M\int_a^b\abs{u(y)}\dd y\le M\int_\RR\abs{u(y)}\dd y
$$

        从而 $u*v(x)$ 存在且
$$
\abs{u*v(x)}\le M\int_\RR\abs{u(x)}\dd y
$$

        \item 不妨设 $\mathrm{supp}(v)\subset[-M,M]$. 则
$$
\forall x\in\RR,\int_\RR u(y)v(x-y)\dd y=\int_{x-M}^{y+M}u(y)v(x-y)\dd y
$$

        由 $u$ 与 $v$ 均局部可积知 $u(y)v(x-y)$ 在 $[x-M,x+M]$ 上可积.

        从而 $u*v(x)$ 存在.
    \end{enumerate}
\end{proof}

\mysubsection{卷积的基本性质}

\mysubsubsection{对称性}

\begin{property}
    设 $u,v:\RR\to\RR$ 使得 $u*v$ 存在.

    则 $v*u$ 也存在,且 $u*v(x)=v*u(x)$.
\end{property}
\begin{proof}
    由假设,积分 $\displaystyle\int_\RR u(y)v(x-y)\dd y$ 存在.

    做变量替换 $y'=x-y$,则由广义积分的定义不难验证,$u(x-y')v(y')$ 作为 $y'$ 的函数也在 $\RR$ 上可积. 且由变量替换公式有
$$
\int_\RR u(x-y')v(y')\dd y'\xlongequal{y=x-y'}\int\RR u(y)v(x-y)\dd y
$$

    即 $v*u(x)=u*v(x)$.
\end{proof}

\mysubsubsection{平移不变性}

设 $u:\RR\to\RR,x_0\in\RR$. 定义
$$
(T_{x_0}u)(x)=u(x-x_0)
$$

我们称 $T_{x_0}$ 是 $\RR$ 上函数空间上的平移算子.

\begin{property}
    设 $u,v:\RR\to\RR$ 使得 $u*v$ 存在. 则
$$
T_{x_0}(u*v)=(T_{x_0}u)*v=u*(T_{x_0}v)
$$
\end{property}
\begin{proof}
    由对称性,我们只需证明第二个等式.

    一方面有
$$
T_{x_0}(u*v)(x)=u*v(x-x_0)=\int_\RR u(y)v(x-x_0-y)\dd y
$$

    另一方面有
$$
u*(T_{x_0}v)(x)=\int_\RR u(y)T_{x_0}v(x-y)\dd y=\int_\RR u(y)v(x-y-x_0)\dd y
$$

    故结论成立.
\end{proof}