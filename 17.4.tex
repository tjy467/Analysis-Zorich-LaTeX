作为含参积分的另一个重要应用,我们定义卷积并讨论其基本性质. 作为卷积的应用,我们证明可用 $C^\infty$ 光滑函数逼近连续函数.

\mysubsection{概念的引入与定义}

设 $f:\RR\to\RR$ 连续. 我们希望找到一列 $C^(1)$ 光滑的函数来逼近 $f$. 一个可能的做法如下:

对任意 $\delta>0$,定义 $f_\delta:\RR\to\RR$ 为
$$
f_\delta(x)\triangleq\frac{1}{2\delta}\int_{x-\delta}^{x+\delta}f(y)\dd y
$$

首先由不定积分的性质知 $f_\delta\in C^{(1)}(\RR)$ 且
$$
f_\delta'(x)=\frac{1}{2\delta}(f(x+\delta)-f(x-\delta))
$$

另一方面
$$
\abs{f_\delta(x)-f(x)}=\abs{\frac{1}{2\delta}\int_{x-\delta}^{x+\delta}(f(y)-f(x))\dd x}=f(\xi)
$$

其中 $\xi\in(x-\delta,x+\delta)$. 从而确有
$$
f_\delta(x)\to f(x),\forall x\in\RR
$$

若将 $x$ 限定在紧区间 $[a,b]$ 上,由一致连续性有
$$
f_\delta\rightrightarrows f\quad(x\in[a,b])
$$

上面定义的 $f_\delta$ 的几何直观非常简单,即对 $f$ 作平均. 具体来讲,对每个固定的 $x$,我们将 $f$ 在 $[x-\delta,x+\delta]$ 上作平均.

一个非常有用的观察如下:我们可以将如上的求平均表达成一个含参积分的形式,具体如下:

定义 $\Delta_\delta:\RR\to\RR$ 为
$$
\Delta_\delta(x)\triangleq\begin{cases}
    \dfrac{1}{2\delta} & x\in[-\delta,\delta]\\
    0 & \abs{x}>\delta
\end{cases}
$$

\begin{center}
    \begin{tikzpicture}[xscale=3,yscale=1.5]
        \draw[->] (-1.5,0)--(1.5,0) node [right] {$x$};
        \draw[->] (0,-0.5)--(0,2.5) node [right] {$y$};
        \node at (0,0) [below left] {$O$};
        \draw[-,red] (-1,0.5)--(1,0.5) node [right] {$\Delta_1$};
        \draw[-,dashed,red] (-1,0.5)--(-1,0) node [below] {$-1$};
        \draw[-,dashed,red] (1,0.5)--(1,0) node [below] {$1$};
        \draw[-,blue] (-0.5,1)--(0.5,1) node [right] {$\Delta_\frac{1}{2}$};
        \draw[-,dashed,blue] (-0.5,1)--(-0.5,0) node [below] {$-\frac{1}{2}$};
        \draw[-,dashed,blue] (0.5,1)--(0.5,0) node [below] {$\frac{1}{2}$};
        \node at (0,1) [above right] {$1$};
        \draw[-,purple] (-0.25,2)--(0.25,2) node [right] {$\Delta_\frac{1}{4}$};
        \draw[-,dashed,purple] (-0.25,2)--(-0.25,0) node [below] {$-\frac{1}{4}$};
        \draw[-,dashed,purple] (0.25,2)--(0.25,0) node [below] {$\frac{1}{4}$};
    \end{tikzpicture}
\end{center}

则
$$
f_\delta(x)=\frac{1}{2\delta}\int_{x-\delta}^{x+\delta}f(y)\dd y=\int_\RR f(y)\Delta_\delta(x-y)\dd y
$$

称最后的积分为 $f$ 与 $\Delta_\delta$ 的卷积,记为 $f*\Delta_\delta(x)$.

\begin{hint}
    一旦有了这个想法,我们发现不一定要将 $\Delta_\delta$ 取成如上的阶梯函数的样子. 例如我们可以取 $\Delta_\delta$ 如图

    \begin{center}
        \begin{tikzpicture}
            \draw[->] (-1.5,0)--(1.5,0) node [right] {$x$};
            \draw[->] (0,-0.5)--(0,2.5) node [right] {$y$};
            \node at (0,0) [below left] {$O$};
            \draw[-,blue] (-1,0)--(0,2);
            \draw[-,blue] (0,2)--(1,0);
            \node[blue] at (1.5,2) {连续};
        \end{tikzpicture}
        \qquad
        \begin{tikzpicture}
            \draw[->] (-1.5,0)--(1.5,0) node [right] {$x$};
            \draw[->] (0,-0.5)--(0,2.5) node [right] {$y$};
            \node at (0,0) [below left] {$O$};
            \draw[domain=-1:1,blue] plot (\x,{cos(pi*\x r)+1});
            \node[blue,align=center] at (1.5,2) {光滑\\紧支撑};
        \end{tikzpicture}
        \qquad
        \begin{tikzpicture}
            \draw[->] (-1.5,0)--(1.5,0) node [right] {$x$};
            \draw[->] (0,-0.5)--(0,2.5) node [right] {$y$};
            \node at (0,0) [below left] {$O$};
            \draw[domain=-1.5:1.5,blue] plot (\x,{2/(1+\x*\x)});
            \node[blue,align=center] at (2.5,2) {光滑\\在 $\infty$ 处骤降};
        \end{tikzpicture}
    \end{center}

    只要保持 $\Delta_\delta\ge 0,\displaystyle\int_\RR\Delta_\delta(x)\dd x=1$,且主要的积分值来源于 $0$ 的一个邻域. 或更数学化的说:对任意 $0$ 的邻域 $U$ 有
$$
\int_U\Delta_\delta(x)\dd x\to 1\quad(\delta\to 0)
$$
\end{hint}

现在我们可以正式地给出两个函数卷积的定义了.

\begin{definition}
    设 $u,v:\RR\to\RR$. 定义 $u$ 与 $v$ 的卷积为
$$
u*v(x)\triangleq\int_\RR u(y)v(x-y)\dd y
$$

    若 $\forall x\in\RR$ 如上的函数 $u(y)v(x-y)$ 在 $\RR$ 上广义可积.
\end{definition}

\begin{example}
    对上面定义的 $\Delta_\delta$ 以及 $f$ 连续,显然有 $f(y)\Delta(x-y)$ 对 $\forall x\in\RR$ 可积,从而
$$
f*\Delta_\delta(x)=\int_\RR f(x)\Delta_\delta(x-y)\dd y
$$

    可以定义.
\end{example}

以下,我们给出几个卷积可定义的充分条件. 这对我们的初步研究来说已经够用了. 我们首先给出几个定义.

\begin{definition}
    设 $G\subset\RR^n$ 为开集,$f:G\to\RR$. 称 $f$ 在 $G$ 上局部可积,若 $\forall x\in G$,存在 $x$ 的邻域 $U(x)\subset G$ 使得 $f$ 在 $U(x)$ 上可积.
\end{definition}

若 $G=\RR$,则 $f$ 在 $\RR$ 上局部可积 $\iff\forall [a,b]\subset\RR,f\in\mathcal{R}[a,b]$.

\begin{definition}
    定义 $f$ 的支撑为集合 $\set{x\in G|f(x)\ne 0}$ 在 $G$ 中的闭包,记为 $\mathrm{supp}(f)$.
    
    进一步,若 $\mathrm{supp}(f)$ 为紧集,则称 $f$ 在 $G$ 内有紧支撑.
\end{definition}

\begin{example}
    $G=(-1,1),f:G\to\RR,f(x)=1-x^2$.

    则 $\mathrm{supp}(f)=(-1,1)$. 从而 $f$ 在 $G$ 内没有紧支撑.
\end{example}

我们用 $C^{(m)}(G)$ 表示在 $G$ 上 $m$ 阶可微函数的全体.

用 $C^{(m)}_0(G)$ 表示在 $G$ 上 $m$ 阶可微且有紧支撑的函数全体.

在 $G=\RR$ 的情形,我们进一步简化为 $C^{(m)}$ 和 $C^{(m)}_0$.

\begin{property}
    设 $u,v:\RR\to\RR$ 均局部可积,若以下任一条件成立:

    \begin{enumerate}
        \item $\abs{u}^2$ 和 $\abs{v}^2$ 均在 $\RR$ 上可积.
        
        \item $\abs{u}$ 与 $\abs{v}$ 其中之一可积,而另一个有界.
        
        \item $u$ 或 $v$ 有紧支撑.
    \end{enumerate}

    则 $u*v$ 在 $\RR$ 上存在.
\end{property}
\begin{proof}
    \begin{enumerate}
        \item 由 Cauchy-Schwartz 不等式有
$$
\forall x\in\RR,\abs{\int_\RR u(y)v(x-y)\dd y}\le\left(\int_\RR\abs{u(y)}^2\dd y\right)^\frac{1}{2}\left(\int_\RR\abs{v(x-y)}^2\dd y\right)^\frac{1}{2}
$$

        作变量替换 $y'=x-y$ 得 $\displaystyle\int_\RR\abs{v(x-y)}^2\dd y=\int_\RR\abs{v(y)}^2\dd y$

        从而 $u*v(x)$ 存在且
$$
\forall x\in\RR,\abs{u*v(x)}\le\left(\int_\RR\abs{u(y)}^2\dd y\right)^\frac{1}{2}\left(\int_\RR\abs{v(x-y)}^2\dd y\right)^\frac{1}{2}
$$

        \item 不妨设 $\abs{u}$ 可积,$\abs{v}\le M$. 则
$$
\forall x\in\RR,\abs{\int_a^b u(y)v(x-y)\dd y}\le\int_a^b\abs{u(y)}\abs{v(x-y)}\dd y\le M\int_a^b\abs{u(y)}\dd y\le M\int_\RR\abs{u(y)}\dd y
$$

        从而 $u*v(x)$ 存在且
$$
\abs{u*v(x)}\le M\int_\RR\abs{u(x)}\dd y
$$

        \item 不妨设 $\mathrm{supp}(v)\subset[-M,M]$. 则
$$
\forall x\in\RR,\int_\RR u(y)v(x-y)\dd y=\int_{x-M}^{y+M}u(y)v(x-y)\dd y
$$

        由 $u$ 与 $v$ 均局部可积知 $u(y)v(x-y)$ 在 $[x-M,x+M]$ 上可积.

        从而 $u*v(x)$ 存在.
    \end{enumerate}
\end{proof}

\mysubsection{卷积的基本性质}

\mysubsubsection{对称性}

\begin{property}
    设 $u,v:\RR\to\RR$ 使得 $u*v$ 存在.

    则 $v*u$ 也存在,且 $u*v(x)=v*u(x)$.
\end{property}
\begin{proof}
    由假设,积分 $\displaystyle\int_\RR u(y)v(x-y)\dd y$ 存在.

    做变量替换 $y'=x-y$,则由广义积分的定义不难验证,$u(x-y')v(y')$ 作为 $y'$ 的函数也在 $\RR$ 上可积. 且由变量替换公式有
$$
\int_\RR u(x-y')v(y')\dd y'\xlongequal{y=x-y'}\int\RR u(y)v(x-y)\dd y
$$

    即 $v*u(x)=u*v(x)$.
\end{proof}

\mysubsubsection{平移不变性}

设 $u:\RR\to\RR,x_0\in\RR$. 定义
$$
(T_{x_0}u)(x)=u(x-x_0)
$$

我们称 $T_{x_0}$ 是 $\RR$ 上函数空间上的平移算子.

\begin{property}
    设 $u,v:\RR\to\RR$ 使得 $u*v$ 存在. 则
$$
T_{x_0}(u*v)=(T_{x_0}u)*v=u*(T_{x_0}v)
$$
\end{property}
\begin{proof}
    由对称性,我们只需证明第二个等式.

    一方面有
$$
T_{x_0}(u*v)(x)=u*v(x-x_0)=\int_\RR u(y)v(x-x_0-y)\dd y
$$

    另一方面有
$$
u*(T_{x_0}v)(x)=\int_\RR u(y)T_{x_0}v(x-y)\dd y=\int_\RR u(y)v(x-y-x_0)\dd y
$$

    故结论成立.
\end{proof}

\mysubsubsection{卷积的微分}

从我们引入的例子可见两个函数做卷积之后函数的性质倾向于变好. 例如 $f$ 连续,$\Delta_\delta$ 为阶梯函数,但 $f*\Delta_\delta$ 为 $C^{(1)}$ 光滑.

由于卷积是一种特殊的含参积分,我们可以期待:若其中一个函数有光滑性,则卷积也具有相同的光滑性.

\begin{property}
    设 $u,v:\RR\to\RR$ 满足 $u$ 连续,$v$ 为 $C^{(m)}$ 光滑,且二者之一有紧支撑.

    则 $u*v$ 也为 $C^{(m)}$ 光滑. 且
$$
(u*v)^{(k)}=u*v^{(k)},\forall 0\le k\le m
$$
\end{property}
\begin{proof}
    任取 $x_0\in\RR$. 由 $u$ 或 $v$ 有紧支撑,可得在 $x_0$ 的邻域上存在 $M>0$ 使得
$$
u*v(x)=\int_{-M}^Mu(y)v(x-y)\dd y
$$

    已知 $f(x,y)\triangleq u(y)v(x-y)$ 关于 $x$ 可导且 $\displaystyle\pard{f}{x}=u(y)v'(x-y)$.

    从而 $u*v(x)$ 也可导,且
$$
(u*v)'(x)=\int_{-M}^Mu(y)v'(x-y)\dd y=(u*v')(x)
$$

    由 $v\in C^{(m)}$ 可得
$$
(u*v)^{(k)}(x)=\int_{-M}^M u(y)v^{(k)}(x-y)\dd y=(u*v^{(k)})(x),\forall 0\le k\le m
$$
\end{proof}

事实上,我们并不需要 $u$ 连续. 我们有如下的性质:

\begin{property}
    设 $u:\RR\to\RR$ 局部可积,$v\in C^{(m)}_0(\RR),0\le m\le\infty$.

    则 $u*v\in C^{(m)}$ 且
$$
(u*v)^{(k)}=u*v^{(k)},\forall 0\le k\le m
$$
\end{property}
\begin{proof}
    任取 $x_0\in\RR$ 以及 $\delta>0$. 由 $v$ 有紧支撑知
$$
\exists M>0,\forall x\in[x_0-\delta,x_0+\delta],u*v(x)=\int_{-M}^M u(y)v(x-y)\dd y
$$

    令 $f(x,y)=u(y)v(x-y)$. 则 $\displaystyle\pard{f}{x}(x,y)=u(y)v'(x-y)$.

    由 $v'$ 与 $v$ 有相同的支撑知
$$
u*v'(x)=\int_{-M}^M u(y)v'(x-y)\dd y
$$

    从而
$$
\begin{aligned}
    &\abs{u*v(x)-u*v(x_0)-u*v'(x_0)(x-x_0)}\\
    =&\abs{\int_{-M}^M u(y)(v(x-y)-v(x_0-y)-v'(x_0-y)(x-x_0))\dd y}\\
    =&\abs{\int_M^M u(y)(v'(\xi-y)-v'(x_0-y))(x-x_0)\dd y}\qquad\xi\in[x,x_0]\\
    \le&C\eta(\abs{x-x_0})\abs{x-x_0}2M
\end{aligned}
$$

    其中 $C$ 为 $\abs{u}$ 在 $[-M,M]$ 上的上界,$\eta(\delta)$ 为 $v'$ 的连续模.

    由 $v'$ 在 $[-M,M]$ 上连续知当 $\delta\to 0$ 时 $\eta(\delta)\to 0$.

    由此可得 $u*v$ 在 $x_0$ 处可导且 $(u*v)'(x_0)=u*v'(x_0)$.

    由 $x_0$ 的任意性知 $u*v\in C^{(1)}$ 且 $(u*v)'=u*v'$.

    由归纳法得 $u*v\in C^{(m)}$ 且 $(u*v)^{(k)}=u*v^{(k)},\forall 0\le k\le m$.
\end{proof}

\begin{hint}
    \begin{enumerate}
        \item 若我们用 $D$ 表示 $\dfrac{\dd}{\dd x}$,则以上性质表明
$$
D^k(u*v)=u*D^kv
$$

        我们将其表述为求微分运算与求卷积运算可交换.

        \item 注意到在上式中 $D^k$ 仅可作用于 $v$,因为我们仅对 $v$ 有光滑性的假设.
        
        \item 性质表明一个局部可积的函数与一个 $C_0^{(m)}$ 的函数做完卷积之后光滑性自动提升到 $C^{(m)}$. 这就是卷积的磨光性质.
    \end{enumerate}
\end{hint}

\mysubsection{逼近单位与 Weierstrass 定理的另证}

以下我们来解释如何使用卷积来逼近函数. 我们首先给出如下的重要概念:

\begin{definition}
    设 $\set{\Delta_\alpha|\alpha\in A}$ 是一族从 $\RR$ 到 $\RR$ 的函数. 设 $\mathcal{B}$ 是 $A$ 的一个基.

    称 $\set{\Delta_\alpha|\alpha\in A}$ 关于 $\mathcal{B}$ 是一个逼近单位,若以下三条成立:

    \begin{enumerate}
        \item $\Delta_\alpha(x)\ge 0,\forall\alpha\in A,\forall x\in\RR$.
        
        \item $\displaystyle\int_\RR\Delta_\alpha(x)\dd x=1,\forall\alpha\in A$.
        
        \item 对任意 $0$ 的邻域 $U$ 有
$$
\lim_{\mathcal{B}}\int_U\Delta_\alpha(x)\dd x=1
$$
    \end{enumerate}
\end{definition}

\begin{hint}
    基于前两个条件,我们可将第三个条件等价地替换成
$$
\lim_{\mathcal{B}}\int_{\RR\setminus U}\Delta_\alpha(x)\dd x=0
$$
\end{hint}

\begin{example}
    设 $\varphi:\RR\to\RR_+$ 有紧支撑且 $\displaystyle\int_\RR\varphi(x)\dd x=1$.

    令 $\varphi_\alpha(x)\triangleq\dfrac{1}{\alpha}\varphi\left(\dfrac{x}{\alpha}\right)$.

    则 $\set{\varphi_\alpha|\alpha>0}$ 关于 $\alpha\to 0$ 是逼近单位.
\end{example}

\begin{example}
    $\Delta_n(x)=\begin{cases}
        \dfrac{(1-x^2)^n}{\int_{-1}^1(1-x^2)^n\dd x} & \abs{x}\le 1\\
        0 & \abs{x}>1
    \end{cases}$

    则 $\set{\Delta_n(x)}$ 关于 $n\to\infty$ 是逼近单位.
\end{example}

\begin{example}
    $\Delta_n(x)=\begin{cases}
        \dfrac{\cos^{2n}x}{\int_{-\frac{\pi}{2}}^\frac{\pi}{2}\cos^{2n}(x)\dd x} & \abs{x}\le\dfrac{\pi}{2}\\
        0 & \abs{x}>\dfrac{\pi}{2}
    \end{cases}$

    则 $\set{\Delta_n(x)}$ 关于 $n\to\infty$ 是逼近单位.
\end{example}

\begin{example}
    $\Delta_\alpha(x)=\begin{cases}
        \frac{1}{2\alpha} & \abs{x}\le\alpha\\
        0 & \abs{x}>\alpha
    \end{cases}$

    则 $\set{\Delta_\alpha|\alpha>0}$ 关于 $\alpha\to 0$ 是逼近单位.
\end{example}

\begin{hint}
    以上两个例子中的 $\Delta_n(x)$ 均不光滑. 以下我们构造一个 $C_0^\infty$ 的例子.
\end{hint}

\begin{example}\label{appfac}
    $G(x)=\begin{cases}
        ae^{-\frac{1}{1-x^2}} & \abs{x}\le 1\\
        0 & \abs{x}>1
    \end{cases}$

    其中 $a=1\Big/\displaystyle\int_{-1}^1e^{-\frac{1}{1-x^2}}\dd x$. 则 $G\in C_0^\infty(\RR)$.

    \begin{center}
        \begin{tikzpicture}[scale=2]
            \draw[->] (-1.5,0)--(1.5,0) node [right] {$x$};
            \draw[->] (0,-0.5)--(0,1.5) node [right] {$y$};
            \node at (0,0) [below left] {$O$};
            \draw[-] (-1,0.05)--(-1,-0.05) node [below] {$-1$};
            \draw[-] (1,0.05)--(1,-0.05) node [below] {$1$};
            \node[blue] at (0.8,0.8) {$G(x)$};

            \draw[domain=-0.999:0.999,blue] plot (\x,{2.25228*exp(-1/(1-\x*\x))});
            \draw[-,blue] (-1.5,0)--(-0.999,0);
            \draw[-,blue] (0.999,0)--(1.5,0);
        \end{tikzpicture}
    \end{center}

    令 $G_\alpha(x)=\dfrac{1}{\alpha}G\left(\dfrac{x}{\alpha}\right)$.

    则 $\set{G_\alpha:\alpha>0}$ 关于 $\alpha\to 0$ 是逼近单位,且 $G_\alpha\in C_0^\infty,\forall\alpha>0$.
\end{example}

以下我们证明:对“好的” $f$ 有 $f*\Delta_\alpha\rightrightarrows f$.

为此我们再给出一个定义:

\begin{definition}
    设 $G\subset\RR^n$ 为开集,$f:G\to\RR$ 连续.

    对 $E\subset G$,称 $f$ 在 $E$ 上(相对于 $G$)一致连续,若
$$
\forall\eps>0,\exists\delta>0,\forall x\in E,\forall y\in G,\abs{x-y}<\delta\implies\abs{f(x)-f(y)}<\eps
$$
\end{definition}

\begin{hint}
    若 $E=G$,则该定义即为通常的 $f$ 在 $G$ 上一致连续.

    但在 $E\subsetneqq G$ 时,该定义强于 $f$ 在 $E$ 上一致连续.
\end{hint}

\begin{property}
    设 $E\subset\RR,f:\RR\to\RR$ 有界. 设 $\set{\Delta_\alpha|\alpha\in A}$ 关于基 $\mathcal{B}$ 是逼近单位.

    若 $\forall\alpha\in A,f*\Delta_\alpha$ 存在,且 $f$ 在 $E$ 上(关于 $\RR$)一致连续,则在 $E$ 上
$$
f*\Delta_\alpha\xrightrightarrows[\mathcal{B}]{}f
$$
\end{property}
\begin{proof}
    任取 $\eps>0$. 由 $f$ 在 $E$ 上关于 $\RR$ 一致连续知
$$
\exists\delta>0,\forall x\in E,\forall y\in G,\abs{x-y}<\delta\implies\abs{f(x)-f(y)}<\eps
$$

    设 $\abs{f}$ 在 $\RR$ 上的上界为 $M$.

    任取 $x\in E$ 有 $f(x)=\displaystyle\int_\RR f(x)\Delta_\alpha(x-y)\dd y$. 从而
$$
\begin{aligned}
    &\abs{f(x)-f*\Delta_\alpha(x)}\\
    =&\abs{\int_\RR(f(x)-f(y))\Delta_\alpha(x-y)\dd y}\\
    \le&\int_{B(x,\delta)}\abs{f(x)-f(y)}\Delta_\alpha(x-y)\dd y+\int_{\RR\setminus B(x,\delta)}\abs{f(x)-f(y)}\Delta_\alpha(x-y)\dd y\\
    \le&\eps\int_{B(x,\delta)}\Delta_\alpha(x-y)\dd y+2M\int_{\RR\setminus B(x,\delta)}\Delta_\alpha(x-y)\dd y\\
    \le&\eps+2M\int_{\RR\setminus B(0,\delta)}\Delta_\alpha(y)\dd y
\end{aligned}
$$

    由逼近条件的性质知
$$
\exists B\in\mathcal{B},\forall\alpha\in B,\int_{\RR\setminus B(0,\delta)}\Delta_\alpha(y)\dd y<\frac{\eps}{2M}
$$

    从而对 $\forall\alpha\in B$ 有
$$
\forall x\in E,\abs{f(x)-f*\Delta_\alpha(x)}<2\eps
$$

    即在 $E$ 上有 $f*\Delta_\alpha\xrightrightarrows[\mathcal{B}]{}f$.
\end{proof}

\begin{hint}
    \begin{enumerate}
        \item 该性质表明在 $f$ 很好,或者 $E$ 很好的时候,$f*\Delta_\alpha$ 可以在 $E$ 上一致逼近 $f$.
        
        \item 在 $E=\set{x_0}$ 为单点集时,该性质表明:若 $f$ 在 $x_0$ 处连续,则 $f*\Delta_\alpha(x_0)\to f(x_0)$.
        
        即在每个 $f$ 的连续点处,$f*\Delta(x_0)$ 会收敛到 $f(x_0)$.
    \end{enumerate}
\end{hint}

\begin{property}
    设 $f:\RR\to\RR$ 一致连续且有界.

    则存在 $\varphi_n\in C^{\infty}(\RR)$ 使得 $\varphi_n\rightrightarrows f$.

    进一步,若 $f\in C_0(\RR)$,则存在 $\varphi_n\in C_0^\infty$ 使得 $\varphi_n\rightrightarrows f$.
\end{property}
\begin{proof}
    若 $f:\RR\to\RR$ 一致连续,利用之前例子 \ref{appfac} 中构造的逼近因子 $\set{\Delta_n\triangleq nG(nx)}$ 即可.

    若 $f\in C_0(\RR)$,则 $f$ 一致连续且有界.

    则只需证明 $\varphi_n\triangleq f*\Delta_n$ 有紧支撑. 直接由定义可得
$$
\mathrm{supp}(\varphi_n)=\mathrm{supp}(f*\Delta_n)\subset\mathrm{supp}(f)+\mathrm{supp}(\Delta_n)
$$

    从而 $\mathrm{supp}(\varphi_n)$ 紧.
\end{proof}

进一步,我们有

\begin{inference}
    若 $f\in C_0^{(m)}(\RR)$,则存在 $\varphi_n\in C_0^\infty(\RR)$ 使得
$$
\varphi_n^{(k)}\rightrightarrows f^{(k)},\forall 0\le k\le m
$$
\end{inference}
\begin{proof}
    若 $f\in C_0^{(m)}$,则 $f,f',\cdots,f^{(m)}\in C_0(\RR)$.

    从而 $\varphi_n^{(k)}=(f*\Delta_n)^{(k)}=f^{(k)}*\Delta_n\rightrightarrows f^{(k)}$.
\end{proof}

作为一个应用,我们可以再次证明 Weierstrass 逼近定理.

\begin{theorem}
    多项式全体在 $C[a,b]$ 中稠密.
\end{theorem}
\begin{proof}
    不妨设 $0<a<b<1$.

    (这是因为我们总可以通过仿射变换 $L(x)=kx+l$ 将 $[a,b]$ 映成 $(0,1)$ 的子区间,而多项式在仿射变换下仍为多项式.)

    任取 $f\in C[a,b]$. 将其延拓成 $\RR$ 上的连续函数如图

    \begin{center}
        \begin{tikzpicture}[xscale=4,yscale=2]
            \draw[->] (-0.5,0)--(1.5,0) node [right] {$x$};
            \draw[->] (0,-0.5)--(0,1.5) node [right] {$y$};
            \node at (0,0) [below left] {$O$};
            \draw[-] (1,0.1)--(1,-0.1) node [below] {$1$};

            \draw[domain=0.25:0.75,blue] plot (\x,{(4*\x-1)*(4*\x-2)*(4*\x-3)+1});
            \draw[-,dashed] (0.25,1)--(0.25,0) node [below] {$a$};
            \draw[-,dashed] (0.75,1)--(0.75,0) node [below] {$b$};
            \draw[-,blue] (-0.5,0)--(0,0);
            \draw[-,blue] (0,0)--(0.25,1);
            \draw[-,blue] (0.75,1)--(1,0);
            \draw[-,blue] (1,0)--(1.5,0);
            \draw[-,dashed] (0.5,0)--(0.5,1) node [above right] {$f(x)$};
            \node at (0.5,0) [below] {$x$};
        \end{tikzpicture}
    \end{center}

    取 $\Delta_n$ 为
$$
\Delta_n(x)=\begin{cases}
    \dfrac{(1-x^2)^n}{\int_{-1}^1(1-x^2)^n\dd x} & \abs{x}\le 1\\
    0 & \abs{x}>1
\end{cases}
$$

    则在 $\RR$ 上有 $f*\Delta_n\rightrightarrows f$.

    另一方面,若 $x\in[a,b]$,则
$$
\begin{aligned}
    f*\Delta_n(x)&=\int_\RR f(y)\Delta_n(x-y)\dd y\\
    &=\int_0^1f(y)\Delta_n(x-y)\dd y\\
    &=\int_0^1f(y)c_n(1-(x-y)^2)^n\dd y\\
    &=\int_0^1f(y)\sum_{j=0}^{2n}a_j(y)x^j\dd y\\
    &=\sum_{j=0}^{2n}\left(\int_0^1 f(y)a_j(y)\dd y\right)x^j
\end{aligned}
$$

    为多项式. 其中 $c_n$ 为系数,$a_n(y)$ 为关于 $y$ 的多项式.

    记 $P_n\triangleq f*\Delta_n$ 为多项式,则在 $[a,b]$ 上有 $P_n\rightrightarrows f$.

    即证多项式在 $C[a,b]$ 上稠密.
\end{proof}

\begin{hint}
    \begin{enumerate}
        \item 我们可以将 $[a,b]$ 换成任何一个紧集 $K\subset\RR$.
        
        \item 设 $G\subset\RR$ 为开集,$f\in C^{(m)}(G)$,则可以证明:存在多项式序列 $\set{P_n}$ 使得对任意紧集 $K\subset G$ 有
$$
P_n^{(k)}(x)\rightrightarrows f^{(k)}(x),0\le k\le m,x\in K
$$

        若进一步 $G$ 有界,且 $f\in C^{(m)}(\overline{G})$,则可以要求在 $\overline{G}$ 上 $P_n^{(k)}\rightrightarrows f^{(k)}$.

        \item 利用类似的方法可证:三角多项式在 $C_p[0,2\pi]$ 中稠密.
        
        其中 $C_p[0,2\pi]=\set{f\in C[0,2\pi]|f(0)=f(2\pi)}$. 而三角多项式形如
$$
T_n(x)=\sum_{k=0}^n(a_k\cos kx+b_k\sin kx)
$$
    \end{enumerate}
\end{hint}

以上讨论的均为具有紧支撑的逼近单位. 接下来我们讨论两个非紧支撑的逼近单位,它们都有重要的应用.

\begin{example}
    $\Delta_y(x)=\dfrac{y}{\pi(x^2+y^2)},y>0$.
\end{example}

\begin{property}
    \begin{enumerate}
        \item $\set{\Delta_y|y>0}$ 关于 $y\to 0$ 为逼近单位.
        
        \item $\Delta_y$ 为调和函数. 更为确切的,$\nabla^2\Delta(x,y)=0$,其中 $\Delta(x,y)\triangleq\Delta_y(x)$.
        
        \item 设 $f:\RR\to\RR$ 有界连续,则 $F(x,y)\triangleq f*\Delta_y(x)$ 在上半平面调和,且 $\displaystyle\lim_{y\to 0}F(x,y)=f(x)$. 对 $x\in[a,b]$,该收敛为一致收敛.
    \end{enumerate}
\end{property}
\begin{proof}
    \begin{enumerate}
        \item 易见 $\Delta_y\ge 0$ 且 $\displaystyle\int_\RR\Delta_y(x)\dd x=1$. 且有
$$
\forall\delta>0,\int_{-\delta}^\delta\Delta_y(x)\dd x=\frac{1}{\pi}\int_{-\frac{\delta}{y}}^{\frac{\delta}{y}}\frac{\dd z}{1+z^2}=\frac{\arctan z}{\pi}\Bigg|_{-\frac{\delta}{y}}^{\frac{\delta}{y}}\to 1\quad(y\to 0)
$$
        
        从而 $\set{\Delta_y|y>0}$ 为逼近单位.

        \item 直接计算可得.
        
        \item 由逼近单位性质知 $\forall[a,b]\subset\RR$ 在 $[a,b]$ 上有 $F(x,y)=f*\Delta_y\rightrightarrows f(x)$.
        
        由定义知
$$
f*\Delta_y(x)=\frac{1}{\pi}\int_\RR f(\xi)\frac{y}{(x-\xi)^2+y^2}\dd\xi
$$

        令 $\Phi(\xi,x,y)\triangleq f(\xi)\dfrac{y}{(x-\xi)^2+y^2}$.

        容易验证
$$
f*\dfrac{\partial\Phi}{\partial x},f*\dfrac{\partial\Phi}{\partial y},f*\dfrac{\partial^2\Phi}{\partial x^2},f*\dfrac{\partial^2\Phi}{\partial y^2}
$$
        均可定义. 且对任意的紧集 $K\subset\mathbb{H}=\set{(x,y)|y>0}$ 均有卷积在 $K$ 上一致收敛.

        从而由含参积分的可微性知 $f*\Delta_y(x)$ 在 $\mathbb{H}$ 上二阶连续可微.
        
        且有 $\nabla^2F(x,y)=f*\nabla^2\Delta_y(x)=0$.
    \end{enumerate}
\end{proof}

故该卷积可以用于求解 $\mathbb{H}$ 上的 Dirichlet 问题.

\begin{example}
    $\Delta_t(x)=\dfrac{1}{2\sqrt{\pi t}}e^{-\frac{x^2}{4t}}$.
\end{example}

\begin{property}
    \begin{enumerate}
        \item $\set{\Delta_t|t>0}$ 关于 $t\to 0$ 为逼近单位.
        
        \item $\Delta_t(x)$ 满足 $\left(\dfrac{\partial}{\partial t}-\dfrac{\partial^2}{\partial x^2}\right)\Delta_t(x)=0$.
        
        \item 设 $f:\RR\to\RR$ 有界连续,则 $u(t,x)\triangleq f*\Delta_t(x)$ 满足 $\left(\dfrac{\partial}{\partial t}-\dfrac{\partial^2}{\partial x^2}\right)u(t,x)=0$,且 $\displaystyle\lim_{t\to 0}u(t,x)=f(x)$.
    \end{enumerate}
\end{property}

故该卷积可以用于求解热传导方程,其中的 $\Delta_t(x)$ 称为热核.