定向的概念我们并不陌生. 在高中物理学习中经常出现顺时针、逆时针、左手系和右手系等概念. 这些都是有关定向的概念. 本节我们详细地讨论定向这个概念以及如何对曲面定向的概念.

\mysubsection{$\RR^n$ 及其子空间的定向}

\begin{definition}
称 $\xi=(\xi_1,\cdots,\xi_n)\in(\RR^n)^n$ 是 $\RR^n$ 的一个标架,若 $\xi$ 是 $\RR^n$ 的一个基.
\end{definition}

等价地,$\xi$ 是一个标架当且仅当 $\xi$ 是一个有序基底.

我们用 $\mathscr{F}(\RR^n)$ 表示 $\RR^n$ 所有标架的全体.

\begin{hint}
若将 $\xi\in\mathscr{F}(\RR^n)$ 等同于一个矩阵,则 $\mathscr{F}(\RR^n)=\mathrm{GL}(n;\RR)$ 为所有可逆矩阵的全体.
\end{hint}

在 $\mathscr{F}(\RR^n)$ 上可以定义一个等价关系
$$
\xi\sim\xi'\iff\det\xi~\text{与}~\det\xi'~\text{同号}
$$

易见 $(\mathscr{F}(\RR^n),\sim)$ 仅有两个等价类. 我们称每个等价类是 $\RR^n$ 的一个定向. 称取定了 $\RR^n$ 的一个定向,若我们固定了其中的某个等价类.

\begin{hint}
在实际应用中,我们通常取定等价类中的某个特殊标架来确定相应的定向.

例如,在 $\RR^2$ 中,我们有 $(e_1,e_2)$ 和 $(e_2,e_1)$ 两个定向.
\end{hint}

设 $V$ 是 $\RR^n$ 的 $k(k<n)$ 维线性子空间. 仿照 $n$ 维情形,我们也用标架来定义.

\begin{definition}
$V$ 的标架定义为 $V$ 的一个有序基底,将 $V$ 所有标架的全体记为 $\mathscr{F}(V)$.
\end{definition}

注意到此时我们不再能用行列式来定义等价类. 为此我们先回到 $n$ 维情形,寻找可能的等价刻画.

\begin{property}
设 $\xi,\xi'\in\mathscr{F}(\RR^n)$ ,且 $A$ 满足 $\xi'=\xi\cdot A$.

则 $\xi\sim\xi'\iff\det A>0$.
\end{property}

\begin{definition}
设 $\xi,\xi'\in\mathscr{F}(V)$ ,且 $A\in M_k(\RR)$ 满足 $\xi'=\xi\cdot A$.

定义 $\xi\sim\xi'\iff\det A>0$.
\end{definition}

\begin{property}
\begin{enumerate}
    \item $\sim$ 是 $\mathscr{F}(V)$ 上的等价关系.
    
    \item $(\mathscr{F}(V),\sim)$ 恰有两个等价类,每个等价类称为 $V$ 的一个定向.
\end{enumerate}
\end{property}

\mysubsection{$\RR^n$ 中开集的定向}

设 $S$ 是 $\RR^n$ 中的一个光滑 $k(k<n)$ 维曲面. 则在每一点 $x_0\in S$ 处有切空间 $T_{x_0}S$ ,我们可以在其上指定一个定向.

当 $x_0$ 在 $S$ 上移动时,一般来说 $T_{x_0}S$ 也不再是一个固定的子空间,从而相应的定向也随之变化. 我们的目标是以一种“好的”、“一致的”对所有的切空间 $T_{x_0}S$ 统一地指定一个定向.

为了明确什么是“好的”、“一致的”定向,我们再次回到 $n$ 维情形.

\img{0.5}{12.2.1.png}

\begin{definition}
设区域 $D\subset\RR^n$ ,则在每点 $x\in D$ 出的切空间均为 $T_xD=\RR^n$.

称在 $D$ 上制定了一个定向,若在每个点 $x$ 处均指定了一个标架 $\xi(x)$ ,满足 $\forall x,y\in D,\xi(x)\sim\xi(y)$.
\end{definition}

\img{0.4}{12.2.2.png}

\begin{hint}
\begin{enumerate}
    \item 在此情形下,我们可以令 $\xi(x)\equiv\xi$ (任何一个固定的标架),从而在 $D$ 上确实可以指定一个定向,且易见此时 $D$ 上只有两种本质不同的定向.
    
    \item 这种定向的方法显然无法推广到曲面的情形,为此我们需要寻找一些等价的指定定向的方法.
\end{enumerate}
\end{hint}

\begin{definition}
设 $F:D\to\mathscr{F}(D)$ 是一个(连续映射),则称 $F$ 是一个(连续)标架场.
\end{definition}

\begin{property}
设区域 $D\subset\RR^n,F:D\to\mathscr{F}(D)$ 是一个连续标架场. 则 $F$ 在 $D$ 上指定了一个定向.
\end{property}
\begin{proof}
定义 $d:D\to\RR$ 为 $d(x)\triangleq\det F(x)$.

则 $d$ 为联通集 $D$ 上的连续函数,且 $d(x)\ne 0,\forall x\in D$.

从而 $\forall x\in D,d(x)>0$ 或 $\forall x\in D,d(x)<0$.

这就说明 $F(x)\sim F(y),\forall x,y\in D$. 从而 $F$ 在 $D$ 上指定了一个定向.
\end{proof}

根据这个性质,为了在 $D$ 上指定一个定向,只需在其上定义一个连续的标架场. 受此启发,我们可以以一种典型的方式来构造这样的标架场.

\begin{definition}
设开集 $D\subset\RR^n,\varphi:\widetilde{D}\to D$ 为微分同胚,则称 $(\varphi,\widetilde{D})$ 是 $D$ 的一个光滑曲面坐标系.
\end{definition}

\begin{property}
设区域 $D\subset\RR^n$.

\begin{enumerate}
    \item 若 $\varphi:\widetilde{D}\to D$ 是一个光滑曲面坐标系,则 $(\varphi'(x)e_1,\cdots,\varphi'(x)e_n)$ 指定了 $D$ 上的一个定向.

    \item 若 $\varphi_1:\widetilde{D}_1\to D$ 与 $\varphi_2:\widetilde{D}_2\to D$ 是两个光滑曲面坐标系,则
$$
\begin{aligned}
&\varphi_1'(x)\eps~与~\varphi_2'(x)\eps~指定相同的定向\\
\iff&\det(\varphi_2^{-1}\circ\varphi_1)'(t)>0,\forall t\in\widetilde{D}_1\\
\iff&\det(\varphi_2^{-1}\circ\varphi_1)'(t)>0,\exists t\in\widetilde{D}_1\\
\end{aligned}
$$

    其中 $\eps=(e_1,\cdots,e_n)$.
\end{enumerate}
\end{property}

\img{0.8}{12.2.3.png}

以上我们讨论了如何在区域(联通开集)$D$ 上指定一个定向. 接下来设 $D\subset\RR^n$ 是一个开集. 若 $D=D_1\cup\cdots\cup D_n$ ,其中每个 $D_i$ 均为联通开集且互不相交,则为指定 $D$ 上的定向我们只需对每个 $D_i$ 指定定向. 从而 $D$ 上本质不同的定向个数为 $2^n$.

\mysubsection{曲面的定向}

\mysubsubsection{初等曲面的定向}

有了以上的预备讨论,我们现在可以来讨论曲面的定向了. 我们从初等曲面开始.

设 $S$ 为 $k$ 维初等光滑曲面. 设 $\varphi:\RR^k\to S$ 满足 $\varphi\in C^{(m)}$.

\begin{definition}
设 $S$ 是 $\RR^n$ 中的一个 $k$ 维初等光滑曲面.

若 $F$ 是 $S$ 上的一个连续标架场,则称 $F$ 在 $S$ 上指定了一个定向.

其中 $F:S\to(\RR^n)^k$ 满足 $\forall x\in S,F(x)\in\mathscr{F}(T_xS)$.
\end{definition}

我们来证明初等光滑曲面可定向.

\begin{property}
设 $F(x)\triangleq\varphi'(t)\eps=\varphi'(\varphi^{-1}(x))\eps$ ,其中 $\eps=(e_1,\cdots,e_n)$. 则 $F$ 是 $S$ 上的一个定向.
\end{property}
\begin{proof}
由 $\varphi$ 为光滑同胚知 $F$ 连续.

由 $\varphi$ 满秩知 $\varphi'(t)\eps\in T_xS,\forall x\in S$.

从而 $F$ 是 $S$ 上的连续标架场.
\end{proof}

\begin{definition}
设 $S$ 为初等光滑曲面,$F$ 与 $G$ 均为 $S$ 上的连续标架场.

称 $F\sim G$ ,若 $F(x)\sim G(x),\forall x\in S$. 此时称 $F$ 和 $G$ 在 $S$ 上指定了相同的定向.
\end{definition}

\begin{property}
设 $S$ 为初等光滑曲面,则

\begin{enumerate}
    \item 如上定义的 $\sim$ 是 $S$ 上连续标架场集合上的等价关系,且其有 $2$ 个等价类.
    
    \item 若 $\varphi_1:D_1\to S,\varphi_2:D_2\to S$ 是 $S$ 的两个不同的图,则 $\varphi_2^{-1}\circ\varphi_1:D_1\to D_2$ 是微分同胚,且其光滑程度与 $\varphi_1,\varphi_2$ 相同.
    
    且有 $F_{\varphi_1}\sim F_{\varphi_2}\iff\det(\varphi_2^{-1}\circ\varphi_1)'(t)>0$.

    其中 $F_{\varphi_1}$ 与 $F_{\varphi_2}$ 为由 $\varphi_1$ 和 $\varphi_2$ 给出的定向.

    特别的,不同的图所指定的定向本质上只有两个.
\end{enumerate}
\end{property}
\begin{proof}
\begin{enumerate}
    \item 留作习题.
    
    \item 由 $\varphi_1,\varphi_2$ 为同胚知 $\varphi_2^{-1}\circ\varphi:D_1\to D_2$ 为同胚.
    
    下证其光滑. 设 $\varphi_1,\cdots,\varphi_2\in C^{(m)}$. 我们已经证明了如下的结论:

    对任意 $x\in S$ 存在 $\eps>0$ 以及
$$
\begin{aligned}
\phi_1:I_\eps^n(t_1,0)\to U(x)\\
\phi_2:I_\eps^n(t_2,0)\to U(x)
\end{aligned}
$$

    为 $C^{(m)}$ 光滑微分同胚,满足 $\phi_1|_{I_\eps^k(t_1)}=\varphi_1,\phi_2|_{I_\eps^k(t_2)}=\varphi_2$.

    其中 $t_1=\varphi_1^{-1}(x),t_2=\varphi_2^{-1}(x),U(x)\subset\RR^n$ 为 $x$ 的邻域.

    从而 $\varphi_2^{-1}\circ\varphi_1=(\phi_2^{-1}\circ\phi_1)|_{I_\eps^k(t_1)}\in C^{(m)}$.

    即证 $\varphi_2^{-1}\circ\varphi_1$ 为 $C^{(m)}$ 微分同胚.
    
    \img{0.8}{12.2.4.png}

    剩下的结论留作习题.
\end{enumerate}
\end{proof}

\mysubsubsection{一般曲面的定向}

受上一小节的启发,我们希望:称一个一般的曲面可定向,若可以在其上定义一个连续的标架场. 但不幸的是,这会排除掉很多我们希望可以定向的曲面,例如 $2$ 维球面 $S^2$.

\begin{theorem}[毛球定理,发球定理,Hair Ball Theorem]
在 $S^2$ 上不存在处处非零的连续切向量场.

等价的,不存在 $F:S^2\to\RR^3$ 连续且满足 $\forall x\in S^2,F(x)\ne 0$ 且 $F(x)\in T_xS^2$.
\end{theorem}

根据以上定理,球面上不存在连续的标架场.

因此,我们必须减弱这个条件. 以下我们来定义一般曲面的定向.

\begin{definition}
设 $S$ 为 $k$ 维光滑曲面,
$$
\begin{aligned}
\varphi_1:D_1\to U_1\subset S\\
\varphi_2:D_2\to U_2\subset S
\end{aligned}
$$

是 $S$ 的两个图. 称它们相容,要么 $U_1\cap U_2=\varnothing$ ,要么
$$
U_1\cap U_2\ne\varnothing~\text{且}~\det(\varphi_2^{-1}\circ\varphi_1)'(t)>0,\forall t\in\varphi_1^{-1}(U_1\cap U_2).
$$
\end{definition}

\img{0.8}{12.2.5.png}

\begin{hint}
这个定义的直观理解是:$\varphi_1$ 和 $\varphi_2$ 在它们有效域的交集上指定了相同的定向.
\end{hint}

\begin{definition}
设 $S$ 为 $k$ 维曲面,$\mathscr{A}=\set{\varphi_i|i\in I}$ 是 $S$ 的一个可数图册.

若 $\forall i\ne j$ 都有 $\varphi_i$ 与 $\varphi_j$ 相容,则称 $\mathscr{A}$ 是 $S$ 的一个定向图册.
\end{definition}

\begin{definition}
称 $S$ 是一个可定向曲面,若 $S$ 存在一个定向图册.
\end{definition}

\begin{definition}
设 $\mathscr{A}_1,\mathscr{A}_2$ 均为 $S$ 的定向图册,定义 $\mathscr{A}_1\sim\mathscr{A}_2\iff \mathscr{A}_1\cup\mathscr{A}_2$ 是 $S$ 的定向图册.
\end{definition}

\begin{property}
\begin{enumerate}
    \item $\sim$ 是 $S$ 的所有定向图册 $\set{\mathscr{A}_i|i\in I}$ 上的等价关系.
    
    \item 若 $S$ 联通,则 $\sim$ 恰有两个等价类.
\end{enumerate}
\end{property}

\begin{definition}
设 $S$ 为可定向曲面,则 $S$ 所有定向图册集合的每个等价类称为 $S$ 的一个定向.

称取定了 $S$ 的一个定向,若我们固定了其中的某个等价类.
\end{definition}

\mysubsection{曲面可定向的判定}

综合以前的讨论,我们已知:初等光滑曲面必可定向,且其图诱导了其上的一个定向. 另一方面,如下的充分条件成立:

\begin{property}
若 $S$ 为光滑曲面,且其上可以定义一个连续标架场,则 $S$ 可定向.
\end{property}

除此之外,目前为止,我们并没有其它的方法来判断曲面能否定向. 以下我们描述一种方法,其对于 $n$ 维空间中 $n-1$ 维曲面可定向的判定十分有效.

\begin{definition}
设 $S$ 为 $n-1$ 维光滑曲面,称 $N:S\to\RR^n$ 是 $S$ 上的(连续)单位法向量场,若 $N$ 为(连续)映射且
$$
N(x)\perp T_xS,\abs{N(x)}=1,\forall x\in S
$$
\end{definition}

\begin{property}
设 $S$ 为 $n-1$ 维光滑曲面,则 $S$ 可定向 $\iff S$ 上存在一个连续单位法向量场.
\end{property}
\begin{proof}
$\implies$ :设 $\set{\varphi_i:D_i\to U_i}$ 是 $S$ 的一个定向图册. 我们先在每个 $U_i$ 上定义一个法向量场.

已知此时 $(\varphi_i'(t)e_1,\cdots,\varphi_i'(t)e_{n-1})$ 是 $U_i$ 上的一个连续标架场,其中 $t=\varphi_i^{-1}(x)$.

定义 $n_i:U_i\to\RR^n$ 满足
$$
\begin{cases}
n_i(x)\perp T_xS,\abs{n_i(x)}=1\\
(n_i(x),\varphi_i'(\varphi_i^{-1}(x))e_1,\cdots,\varphi_i'(\varphi_i^{-1}(x))e_{n-1})\sim(e_1,\cdots,e_n)
\end{cases}
$$

不难验证,$n_i$ 是 $U_i$ 上的连续单位法向量场.

由 $\varphi_i$ 与 $\varphi_j$ 相容,易证 $n_i|_{U_i\cap U_j}=n_j|_{U_i\cap U_j}$.

于是定义 $N:S\to\RR^n$ 为:$\forall x\in S,\exists U_i,x\in U_i$ ,定义 $N(x)\triangleq n_i(x)$.

则 $N$ 为 $S$ 上的连续单位法向量场.

$\impliedby$ :下设 $N:S\to\RR^n$ 是连续单位法向量场. 设 $\set{\varphi_i:D_i\to U_i}$ 是 $S$ 的图册.

对任意 $i$ ,定义 $\widetilde\varphi_i$ 如下:

任取 $x\in U_i$ ,设 $t\in D_i$ 满足 $\varphi_i(t)=x$.

若 $(N(x),\varphi_i'(t)e_1,\cdots,\varphi_i'(t)e_{n-1})\sim(e_1,\cdots,e_n)$ ,则令 $\widetilde\varphi_i=\varphi_i$.

若 $(N(x),\varphi_i'(t)e_1,\cdots,\varphi_i'(t)e_{n-1})\not\sim(e_1,\cdots,e_n)$ ,则令
$$
\widetilde\varphi_i(t_1,\cdots,t_{n-1})=\varphi_i(t_2,t_1,t_3,\cdots,t_{n-1})
$$

相应地定义 $\widetilde{D}_i$ ,则可以验证 $\set{\widetilde\varphi_i:\widetilde{D}_i\to U_i}$ 是 $S$ 的定向图册,从而 $S$ 可定向.
\end{proof}

若 $S$ 是一个 $n-1$ 维可定向的联通曲面,则其上可以定义一个连续单位法向量场 $N:S\to\RR^n$. 其给定了曲面的一个定向.
注意到 $-N$ 也是一个连续单位法向量场,且显然 $N$ 与 $-N$ 给出了不同的定向.
由于 $S$ 联通,$S$ 上仅有两个定向. 由此可知,$N$ 与 $-N$ 决定了所有的定向.

正是在这个意义下,我们称 $S$ 是一个双面曲面.
\mysubsection{一些例子}

利用法向量的准则,我们可以得到二维球面 $S^2$ 与二维环面 $\Pi^2$ 可定向.

一般的,$n$ 维球面 $S^n$ 与 $n$ 维环面 $\Pi^n$ 可定向.

\begin{example}
Möbius 带不可定向,因为其上不存在连续单位法向量场.
\end{example}

\begin{example}
Klein 瓶不可定向,因为 Klein 瓶中包含了 Möbius 带.
\end{example}