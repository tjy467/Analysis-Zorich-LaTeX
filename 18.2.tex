上节中,我们研究了内积空间中向量关于一个正交组的抽象 Fourier 级数的性质. 本节我们回到具体的空间 $\mathcal{R}_2[-\pi,\pi]$,并详尽地研究如上的问题.

\mysubsection{三角级数的收敛性(平均收敛)}

\mysubsubsection{形式 Fourier 级数}

我们称如下的两种级数为形式三角 Fourier 级数:
$$
\begin{matrix}
    \displaystyle\frac{a_0}{2}+\sum_{n\ge 1}a_n\cos nx+b_n\sin nx & \qquad\text{(三角形式)}\\
    \displaystyle\sum_{n\in\mathbb{Z}}c_ne^{inx} & \qquad\text{(指数形式)}
\end{matrix}
$$

我们关心如下的一些问题:

\begin{enumerate}
    \item 这些级数是否可以定义收敛性?
    
    \item 可以有哪些不同的收敛性?
    
    \item\label{isfourier} 它们是否是某个 $f\in \mathcal{R}_2[-\pi,\pi]$ 的 Fourier 级数?
    
    \item 若是,这个级数是否与 $f$ 相等?
\end{enumerate}

当然为了定义收敛性,最自然的是先考虑部分和
$$
\begin{aligned}
    T_n&\triangleq\frac{a_0}{2}+\sum_{k=1}^na_k\cos kx+b_k\sin kx\\
    S_n&\triangleq\sum_{\abs{k}\le n}c_ke^{ikx}
\end{aligned}
$$

可见 $T_n,S_n$ 均在 $\RR$ 上为 $C^\infty$ 光滑函数. 我们称它们为三角多项式. 从而我们可以考虑

\begin{itemize}
    \item $S_n$ 是否会在 $\RR$ 上逐点收敛?一致收敛?或在积分的意义下收敛?等等可能的问题.
\end{itemize}

当然在不加任何假设的情况下,以上所有的问题均可能有否定的答案.

例如,我们来考虑问题 \ref{isfourier}.

设 $\displaystyle\sum_{n\in\mathbb{Z}}c_ne^{inx}$ 是某 $f\in \mathcal{R}_2[-\pi,\pi]$ 的 Fourier 级数. 则
$$
c_n=\frac{1}{2\pi}\int_{-\pi}^{\pi}f(x)e^{-inx}\dd x
$$

且 Bessel 不等式成立
$$
\sum_{n\in\mathbb{Z}}\abs{c_n}^2\le\frac{1}{2\pi}\int_{-\pi}^{\pi}\abs{f(x)}^2\dd x<+\infty
$$

特别的,这说明 $c_n$ 在 $n\to\infty$ 时不能下降太慢.

例如
$$
\displaystyle\sum_{k=1}^\infty\frac{\sin kx}{\sqrt{k}}=\sum_{k=1}^\infty\frac{e^{ikx}-e^{-ikx}}{\sqrt{k}i}
$$

就不是一个 $f\in \mathcal{R}_2[-\pi,\pi]$ 的 Fourier 级数,因为 $\displaystyle\sum_{k\ne 0}\abs{c_k}^2=2\sum_{k\ge 1}\frac{1}{k}=+\infty$.

Bessel 不等式告诉我们:为了使 $\displaystyle\sum_{n\in\mathbb{Z}}c_ne^{inx}$ 是某个 $f\in \mathcal{R}_2[-\pi,\pi]$ 的 Fourier 级数,必要条件是 $\displaystyle\sum_{n\in\mathbb{Z}}\abs{c_n}^2<+\infty$.

那么下一个自然的问题是:

\begin{itemize}
    \item 若 $\displaystyle\sum_{n\in\mathbb{Z}}c_ne^{inx}$ 的确满足 $\displaystyle\sum_{n\in\mathbb{Z}}\abs{c_n}^2<+\infty$,是否必然存在 $f\in \mathcal{R}_2[-\pi,\pi]$ 使得 $S_n\to f$?
    
    其中 $S_n\to f$ 的含义为 $\norm{S_n-f}_2\to 0$.
\end{itemize}

答案依然是否定的. 因为此时 $S_n$ 可能没有极限(可证 $\set{S_n}$ 必为 Cauchy 列).

实际上这是因为 $\mathcal{R}_2[-\pi,\pi]$ 不完备. 若我们将其完备化,则会得到 $L_2[-\pi,\pi]$,即 $[-\pi,\pi]$ 上所有 Lebesgue 平方可积的函数全体. 则上面的命题对 $L_2[-\pi,\pi]$ 是成立的.

此时我们有一个同构 $U$:
$$
\begin{aligned}
    U:l^2(\mathbb{Z})&\to L_2[-\pi,\pi]\\
    c=(c_n)_{n\in\mathbb{Z}}&\mapsto\sum_{n\in\mathbb{Z}}c_ne^{inx}
\end{aligned}
$$

\mysubsubsection{三角函数的平均收敛}

我们在第二节已经提到 $\set{e^{inx}|n\in\mathbb{Z}}$ 与 $\set{1;\cos nx,\sin nx|n\ge 1}$ 均在 $\mathcal{R}_2[-\pi,\pi]$ 中完备.

但为了证明这一点,我们需要首先研究级数的点态收敛. 但我们先陈述这个结论.

\begin{theorem}
    \begin{enumerate}
        \item $\set{e^{inx}|n\in\mathbb{Z}}$ 与 $\set{1;\cos nx,\sin nx|n\ge 1}$ 均在 $\mathcal{R}_2[-\pi,\pi]$ 中完备.
        
        \item 若 $f\in \mathcal{R}_2[-\pi,\pi]$,则
$$
\norm{S_n(f)-f}_2\to 0,\qquad\norm{T_n(f)-f}_2\to 0
$$

        其中
$$
\begin{aligned}
    &S_n(f)=\sum_{\abs{k}\le n}c_k(f)e^{ikx}\\
    &T_n(f)=\frac{a_0(f)}{2}+\sum_{k=1}^na_k(f)\cos kx+b_k(f)\sin kx
\end{aligned}
$$

        \item Parsevel 等式成立.
$$
\begin{aligned}
    \norm{f}^2&=2\pi\sum_{n\in\mathbb{Z}}\abs{c_n(f)}^2\\
    \norm{f}^2&=\frac{\pi}{2}\abs{a_0(f)}^2+\pi\sum_{n=1}^\infty(\abs{a_n(f)}^2+\abs{b_n(f)}^2)
\end{aligned}
$$
    \end{enumerate}
\end{theorem}

\mysubsection{三角 Fourier 级数的点态收敛}

\mysubsubsection{一些历史注记}

定义 $\mathcal{R}_1[-\pi,\pi]$ 为在 $[-\pi,\pi]$ 上广义可积且绝对可积的函数全体.

即 $f$ 在 $[-\pi,\pi]$ 上的广义积分存在且 $\displaystyle\int_{-\pi}^{\pi}\abs{f}\dd x<+\infty$.

我们在第一节已经证明

\begin{property}
    $\mathcal{R}_2[-\pi,\pi]\subset \mathcal{R}_1[-\pi,\pi]$.
\end{property}

对每个 $f\in \mathcal{R}_1[-\pi,\pi]$,由 $\abs{f}$ 可积知我们也可以定义 $f$ 的 Fourier 级数
$$
f\sim\sum_{n\in\mathbb{Z}}c_n(f)e^{inx}
$$

其中 $c_n(f)=\displaystyle\int_{\pi}^{\pi}f(x)e^{-inx}\dd x$.

一个著名的问题是:对 $\forall f\in C[-\pi,\pi]$ 满足 $f(-\pi)=f(\pi)$,是否有 $S_n(f)(x)\to f(x)$ 几乎处处成立?

当时,人们已知存在 $f\in C[-\pi,\pi],f(-\pi)=f(\pi)$ 使得 $S_n(f)(x)$ 在某些点处发散.

\begin{itemize}
    \item Kolmogorov  构造了一个 $f\in L_1[-\pi,\pi]$ 使得对 $\forall x\in[-\pi,\pi],S_n(f)(x)$ 发散.
    
    \item Menshov 构造了一个三角级数 $\displaystyle\sum_{n\in\mathbb{Z}}c_ne^{inx}$ 使得 $S_n(f)(x)$ 几乎处处收敛到 $0$.
\end{itemize}

Luzin 提出以下猜测;对 $\forall f\in L_2[-\pi,\pi]$ 有 $S_n(f)(x)\to f(x)$ 几乎处处成立.

1966 年 Carleson 证明了如上的猜测.

由 $C[-\pi,\pi]\subset L_2[-\pi,\pi]$,最开始的问题也得到了回答.

综上,Fourier 级数的收敛性一般来讲是十分困难的问题. 但以下我们讨论在一些更强的假设下级数的收敛性.

\mysubsubsection{Dirichlet 核与部分和的表示}

下设 $f\in \mathcal{R}_1[-\pi,\pi]$,记其 Fourier 级数的 $n$ 个部分和为
$$
S_n(f)(x)\triangleq\sum_{k=-n}^nc_k(f)e^{ikx}
$$

其中 $c_k(f)=\displaystyle\frac{1}{2\pi}\int_{-\pi}^{\pi}f(x)e^{-ikx}\dd x$.

将 $c_k(f)$ 的表达式代入 $S_n(f)$ 可得
$$
\begin{aligned}
    S_n(f)(x)&=\sum_{k=-n}^n\frac{1}{2\pi}\int_{-\pi}^{\pi}f(y)e^{-iky}\dd y\cdot e^{ikx}\\
    &=\frac{1}{2\pi}\int_{-\pi}^{\pi}f(y)\sum_{k=-n}^ne^{ik(x-y)}\dd y
\end{aligned}
$$

若令 $D_n(u)\triangleq\displaystyle\sum_{k=-n}^ne^{iku}$,则
$$
S_n(f)(x)=\frac{1}{2\pi}\int_{-\pi}^{\pi}f(y)D_n(x-y)\dd y
$$

我们称 $D_n(u)$ 为第 $n$ 个 Dirichlet 核.

\begin{property}
    \begin{enumerate}
        \item $D_n(x)=\dfrac{\sin\left(n+\frac{1}{2}\right)x}{\sin\frac{x}{2}},\forall x\in[-\pi,\pi]$
        
        \item $\displaystyle\frac{1}{2\pi}\int_{-\pi}^{\pi}D_n(x)\dd x=1$
    \end{enumerate}
\end{property}
\begin{proof}
    \begin{enumerate}
        \item 由等比级数的性质直接计算可得.
        
        \item $\displaystyle\frac{1}{2\pi}\int_{-\pi}^{\pi}D_n(x)\dd x=\frac{1}{2\pi}\int_{-\pi}^{\pi}1\dd x=1$
    \end{enumerate}
\end{proof}

\begin{center}
    \begin{tikzpicture}[yscale=0.8,xscale=1.5]
        \draw[->] (-4,0)--(4,0) node [right] {$x$};
        \draw[->] (0,-2)--(0,6) node [right] {$y$};
        \node at (0,0) [below left] {$O$};
        \draw[red,samples=100,domain={0.01}:{pi}] plot(\x,{sin(\x*5/2 r)/(sin(\x/2 r))});
        \draw[red,samples=100,domain={-pi}:{-0.01}] plot(\x,{sin(\x*5/2 r)/(sin(\x/2 r))});
        \draw[-,red] (-0.01,5)--(0.01,5); 
        \node at (0,5) [above right] {$5$};
        \draw[-,dashed] ({-pi},1)--({-pi},0) node [below] {$-\pi$};
        \draw[-,dashed] ({pi},1)--({pi},0) node [below] {$\pi$};
        \node at ({-pi*4/5},0) [above right] {$-\frac{4\pi}{5}$};
        \node at ({pi*4/5},0) [above left] {$\frac{4\pi}{5}$};
        \node at ({-pi*2/5},0) [below right] {$-\frac{2\pi}{5}$};
        \node at ({pi*2/5},0) [below left] {$\frac{2\pi}{5}$};
        \node at (1.2,3) {$D_2(x)$};
    \end{tikzpicture}
\end{center}

\begin{hint}
    $D_n(x)$ 在 $[-\pi,\pi]$ 上振荡且 $\displaystyle\int_{-\pi}^{\pi}\abs{D_n(x)}\dd x\to\infty$.
\end{hint}

以下,我们将 $f$ 周期地延拓到 $\RR$,其周期为 $2\pi$.

\begin{property}
$$
\begin{aligned}
    S_n(f)(x)&=\frac{1}{2\pi}\int_{-\pi}^\pi f(x-y)D_n(y)\dd y\\
    &=\frac{1}{2\pi}\int_0^\pi[f(x+y)+f(x-y)]D_n(y)\dd y
\end{aligned}
$$
\end{property}
\begin{proof}
    作变量替换 $y'=x-y$,并使用 $f$ 与 $D_n$ 的周期性,有:
$$
\begin{aligned}
    S_n(f)(x)&=\frac{1}{2\pi}\int_{-\pi}^{\pi}f(y)D_n(x-y)\dd y\\
    &=\frac{1}{2\pi}\int_{x-\pi}^{x+\pi}f(x-y')D_n(y')\dd y'\\
    &=\frac{1}{2\pi}\int_{-\pi}^\pi f(x-y)D_n(y)\dd y
\end{aligned}
$$

    再利用 $D_n(-y)=D_n(y)$ 可得
$$
\begin{aligned}
    S_n(f)(x)&=\frac{1}{2\pi}\left[\int_{0}^\pi f(x-y)D_n(y)\dd y+\int_{-\pi}^0f(x-y)D_n(y)\dd y\right]\\
    &=\frac{1}{2\pi}\int_0^\pi[f(x+y)+f(x-y)]D_n(y)\dd y
\end{aligned}
$$
\end{proof}

\begin{hint}
    \begin{enumerate}
        \item 记 $\mathbb{T}=\RR/_{\displaystyle 2\pi\mathbb{Z}}$.
        
        则每个定义在 $\RR$ 上的 $2\pi$-周期函数都可以看成是定义在 $\mathbb{T}$ 上的函数. 在此看法下
$$
\begin{aligned}
    S_n(f)(x)&=\frac{1}{2\pi}\int_{-\pi}^\pi f(y)D_n(x-y)\dd y\\
    &=\frac{1}{2\pi}\int_{\mathbb{T}}f(y)D_n(x-y)\dd y\\
    &=\frac{1}{2\pi}f*D_n(x)
\end{aligned}
$$

        即 $2\pi S_n(f)(x)$ 是 $f$ 与 Dirichlet 核 $D_n$ 的卷积.

        \item 上述性质告诉我们,$S_n(f)(x)$ 的收敛性仅与 $f$ 在 $x$ 的一个小领域内的表现有关(至少在 $f$ 非常正则时是这样).
        
        从而我们可以将上述公式认为是一个局部化的公式.
    \end{enumerate}
\end{hint}

\mysubsubsection{Riemann-Lebesgue 引理与局部化准则}

上述的局部化公式以及我们马上要证明的 Riemann-Lebesgue 引理是研究 Fourier 级数收敛性的基本工具. 以下我们来证明 Riemann-Lebesgue 引理.

\begin{lemma}[Riemann-Lebesgue]
    设 $f:(a,b)\to\RR$ 局部可积,且 $\abs{f}$ 在 $(a,b)$ 上广义可积. 则
$$
\int_a^b f(x)e^{i\lambda x}\dd x\to 0,\qquad\lambda\in\RR,\lambda\to\infty
$$
\end{lemma}
\begin{proof}
    由 $\abs{f}$ 在 $(a,b)$ 上广义可积知 $f(x)e^{i\lambda x}$ 也在 $(a,b)$ 上广义可积.

    任取 $\eps>0$. 则存在 $[a',b']\subset(a,b)$ 使得
$$
\abs{\int_a^b f(x)e^{i\lambda x}\dd x-\int_{a'}^{b'}f(x)e^{i\lambda x}\dd x}<\frac{\eps}{3}
$$

    由 $f$ 在 $[a',b']$ 上 Riemann 可积,存在 $[a',b']$ 的分划 $P=\set{x_0,x_1,\cdots,x_n}$ 使得其 Riemann 下和 $s(f,P)$ 满足
$$
0\le\int_{a'}^{b'}f(x)\dd x-s(f,P)<\frac{\eps}{3}
$$

    其中
$$
s(f,P)=\sum_{i=1}^n m_i\Delta x_i,\qquad m_i=\inf_{x\in[x_{i-1},x_i]}f(x)
$$

    令
$$
g(x)=\begin{cases}
    m_1 & x\in[x_0,x_1]\\
    m_2 & x\in[x_1,x_2]\\
    \vdots\\
    m_n & x\in[x_{n-1},x_n]
\end{cases}
$$

    则
$$
\begin{aligned}
    &\abs{\int_{a'}^{b'}f(x)e^{i\lambda x}\dd x-\int_{a'}^{b'}g(x)e^{i\lambda x}\dd x}\\
    =&\abs{\sum_{j=1}^n\int_{x_{j-1}}^{x_j}(f(x)-m_j)e^{i\lambda x}\dd x}\\
    \le&\sum_{j=1}^n\int_{x_{j-1}}^{x_j}\abs{f(x)-m_j}\dd x\\
    =&\int_{a'}^{b'}f(x)\dd x-\sum_{j=1}^mm_j\Delta x_j<\frac{\eps}{3}
\end{aligned}
$$

    最后我们有
$$
\int_{x_{j-1}}^{x_j}g(x)e^{i\lambda x}\dd x=m_j\int_{x_{j-1}}^{x_j}e^{i\lambda x}\dd x=m_j\frac{e^{i\lambda x_j}-e^{i\lambda x_{j-1}}}{i\lambda}\to 0,\qquad\lambda\to\infty
$$

    故
$$
\begin{aligned}
    \abs{\int_a^bf(x)e^{i\lambda x}\dd x}&\le\frac{\eps}{3}+\frac{\eps}{3}+\abs{\int_{a'}^{b'}g(x)e^{i\lambda x}\dd x}\\
    &\le\eps+\sum_{j=1}^n\abs{\int_{x_{j-1}}^{x_j}g(x)e^{i\lambda x}\dd x}\\
    \implies&\overline{\lim_{\lambda\to\infty}}\abs{\int_a^b f(x)e^{i\lambda x}\dd x}\le\eps
\end{aligned}
$$

    由 $\eps$ 的任意性知
$$
\lim_{\lambda\to\infty}\int_a^bf(x)e^{i\lambda x}\dd x=0
$$
\end{proof}

作为推论,我们有

\begin{inference}
    \begin{enumerate}
        \item 若 $f$ 定义与引理相同,则有
$$
\lim_{\lambda\to\infty}\int_a^bf(x)\cos\lambda x\dd x=\lim_{\lambda\to\infty}\int_a^bf(x)\sin\lambda x\dd x=0
$$

        \item 若 $f(a,b)\to\mathbb{C}$ 满足 $f$ 在 $(a,b)$ 上局部可积且 $\abs{f}$ 广义可积,则有
$$
\lim_{\lambda\to\infty}\int_a^b f(x)e^{i\lambda x}\dd x=0
$$
    \end{enumerate}
\end{inference}
\begin{proof}
    对实部和虚部分别应用以上引理即可.
\end{proof}

\begin{hint}
    \begin{enumerate}
        \item 若已知 $f\in \mathcal{R}_2[-\pi,\pi]$,则我们由 Bessel 不等式知
$$
\lim_{\abs{n}\to\infty}\int_{-\pi}^{\pi}f(x)e^{inx}\dd x=0
$$

        这里我们得到的结论更强:$f$ 仅需在 $\mathcal{R}_1[-\pi,\pi]$ 中. 且我们是对 $\lambda\to\infty$ 均有极限为 $0$,而不仅仅是对 $n\to\infty$.

        \item 现在我们可以将之前的“$S_n(f)(x)$ 的收敛性仅依赖于 $f$ 在 $x$ 局部的性质”这句话精确化.
    \end{enumerate}
\end{hint}

\textbf{断言:}若 $f\in\mathcal{R}_1[-\pi,\pi]$,则对 $\forall x\in[-\pi,\pi],\forall\delta>0$ 有
$$
S_n(f)(x)=\frac{1}{2\pi}\int_0^{\delta}[f(x+t)+f(x-t)]D_n(t)\dd t+o(1),\qquad n\to\infty
$$

这是因为:由 $f\in\mathcal{R}_1[-\pi,\pi]$,Riemann-Lebesgue 引理的条件成立,从而当 $t\in[\delta,\pi]$ 时有
$$
\begin{aligned}
    &\sin\frac{t}{2}\ge\sin\frac{\delta}{2}>0\\
    \implies&\abs{D_n(t)}=\abs{\frac{\sin\left(n+\frac{1}{2}\right)t}{\sin\frac{t}{2}}}\le\abs{\frac{\sin\left(n+\frac{1}{2}\right)t}{\sin\frac{\delta}{2}}}\\
    \implies&\abs{\frac{1}{2\pi}\int_\delta^\pi[f(x+t)+f(x-t)]\frac{\sin\left(n+\frac{1}{2}\right)t}{\sin\frac{t}{2}}\dd t}\\
    \le&\frac{1}{2\pi}\frac{1}{\abs{\sin\frac{\delta}{2}}}\int_\delta^\pi[f(x+t)+f(x-t)]\sin\left(n+\frac{1}{2}\right)t\dd t\to 0,\qquad n\to\infty
\end{aligned}
$$

由此可得

\begin{inference}[局部化准则]
    设 $f,g\in\mathcal{R}_1[-\pi,\pi]$. 设 $x_p\in[-\pi,\pi]$,若
$$
\exists\delta>0,f|_{(x_0-\delta,x_0+\delta)}=g|_{(x_0-\delta,x_0+\delta)}
$$

    则 $S_n(f)(x_0)$ 与 $S_n(g)(x_0)$ 同时收敛或同时发散.

    当它们均收敛时,收敛值相同.
\end{inference}

\begin{hint}
    在 $x_0$ 是端点 $-\pi$ 或 $\pi$ 时,由我们之前的约定,$f,g$ 均为以 $2\pi$ 为周期的函数. 从而此时的条件为
$$
f|_{[-\pi,-\pi+\delta)\cup(\pi-\delta,\pi]}=g|_{{[-\pi,-\pi+\delta)\cup(\pi-\delta,\pi]}}
$$
\end{hint}

\mysubsubsection{Fourier 级数在某点处收敛的充分条件}

本节讨论 Fourier 级数在某点出收敛的充分条件,即 Dini 条件.

\begin{definition}
    设 $x_0\in\RR,\mathring{U}$ 是 $x_0$ 的一个去心邻域.

    设 $f:\mathring{U}\to\mathbb{C}$. 称 $f$ 在 $x_0$ 处满足 Dini 条件,若

    \begin{enumerate}
        \item $\displaystyle\lim_{x\to x_0+}f(x)=f(x_0+)$ 与 $\displaystyle\lim_{x\to x_0-}f(x)=f(x_0-)$ 均存在.
        
        \item 存在 $\delta>0$ 使得
$$
\int_0^{\delta}\frac{\abs{f(x_0)+t-f(x_0+)+f(x_0-t)-f(x_0-)}}{t}\dd t<+\infty
$$
    \end{enumerate}
\end{definition}

\begin{example}
    设 $f:\mathring{U}(x_0)\to\mathbb{C}$ 满足 Hölder 条件:

    \begin{enumerate}
        \item $f(x_0+)$ 与 $f(x_0-)$ 均存在.
        
        \item 存在 $\alpha\in(0,1],M>0,\delta>0$ 使得
$$
\abs{f(x_0+t)-f(x_0+)}\le Mt^{\alpha},\abs{f(x_0-t)-f(x_0-)}\le Mt^{\alpha},\forall 0<t\le\delta
$$
    \end{enumerate}

    则 $f$ 在 $x_0$ 处满足 Dini 条件. 这是因为
$$
\int_0^{\delta}\frac{\abs{f(x_0)+t-f(x_0+)+f(x_0-t)-f(x_0-)}}{t}\dd t\le\int_0^\delta\frac{2Mt^{\alpha}}{t}\dd t=\frac{2M\delta^{\alpha}}{\alpha}<+\infty
$$

    特别的,若 $f:U(x_0)\to\mathbb{C}$ 连续且
$$
\exists\alpha\in(0,1],M>0,\delta<0,\forall\abs{t}\le\delta,\abs{f(x_0+t)-f(x_0)}\le M\abs{t}^\alpha
$$

    则 $f$ 在 $x_0$ 处满足 Dini 条件.
\end{example}

\begin{definition}
    称 $f:[a,b]\to\mathbb{C}$ 分段连续,若 $\exists a=x_0<x_1<\cdots<x_n=b$ 使得 $f$ 在 $(x_{i-1},x_i)$ 上连续且 $f(x_i\pm)$ 均存在.

    (对 $a$ 仅要求 $f(a+)$ 存在,对 $b$ 仅要求 $f(b-)$ 存在)
\end{definition}

\begin{definition}
    称 $f:[a,b]\to\mathbb{C}$ 分段连续可微,若 $f$ 在除了有限个点外可导且其导函数分段连续。(在不可导处任意指定函数值)
\end{definition}

\begin{property}
    若 $f:[a,b]\to\mathbb{C}$ 分段连续可导,则 $f$ 在 $[a,b]$ 上满足 Dini 条件.
\end{property}

\begin{example}
    $f(x)=\mathrm{sgn}\;x$ 在 $\forall x\in\RR$ 处满足 Dini 条件.
\end{example}

\begin{example}
    $f(x)=\abs{x}$ 分段连续可微且 $f'(x)=\mathrm{sgn}\;x$.
\end{example}

\begin{theorem}
    设 $f:\RR\to\mathbb{C}$ 以 $2\pi$ 为周期,且 $f|_{[-\pi,\pi]}$ 局部可积且绝对可积.

    若 $f$ 在 $x\in[-\pi,\pi]$ 处满足 Dini 条件,则由其 Fourier 级数在 $x$ 处收敛且
$$
\sum_{n\in\mathbb{Z}}c_n(f)e^{inx}=\frac{f(x+)+f(x-)}{2}
$$
\end{theorem}
\begin{proof}
    由 $\displaystyle\frac{1}{2\pi}\int_0^\pi D_n(t)\dd t=\frac{1}{2}$ 知
$$
\frac{f(x+)+f(x-)}{2}=\frac{1}{2\pi}\int_0^\pi[f(x+)+f(x-)]D_n(t)\dd t
$$

    所以
$$
\begin{aligned}
    &S_n(f)(t)-\frac{f(x+)+f(x-)}{2}\\
    =&\frac{1}{2\pi}\int_0^\pi[f(x+t)-f(x+)+f(x-t)-f(x-)]D_n(t)\dd t\\
    =&\frac{1}{2\pi}\int_0^\delta\frac{f(x+t)-f(x+)+f(x-t)-f(x-)}{t}tD_n(t)\dd t \qquad & (I)\\
    &+\frac{1}{2\pi}\int_\delta^\pi[f(x+t)-f(x+)+f(x-t)-f(x-)]D_n(t)\dd t & (II)
\end{aligned}
$$

    注意到当 $t\in[0,\pi]$ 时 $\abs{tD_n(t)}=\dfrac{t}{\sin\frac{t}{2}}\abs{\sin\left(n+\dfrac{1}{2}\right)t}$ 有界.

    从而由 $f$ 在 $x$ 处满足 Dini 条件知:$\forall\eps>0,\exists\delta>0$ 使得 $\abs{I}<\dfrac{\eps}{2}$.

    现固定该 $\delta$,由 Riemann-Lebesgue 引理知
$$
\overline{\lim_{n\to\infty}}\abs{S_n(f)(t)-\frac{f(x+)+f(x-)}{2}}\le\frac{\eps}{2}
$$

    由 $\eps$ 的任意性知 $\displaystyle\lim_{n\to\infty}S_n(f)(t)=\frac{f(x-)+f(x+)}{2}$.
\end{proof}

\begin{hint}
    注意到若我们仅仅改变 $f$ 在一个点处的取值,则其不会改变 $c_k(f)$,从而不会改变 $S_n(f)$ 以及 $S_n(f)(x)$ 在 $x$ 处的极限行为.

    另一方面,由局部化准则以及 Dini 条件的定义,我们看到 $S_n(f)(x)$ 的极限性质事实上由 $f$ 在 $x$ 的一个小邻域内的平均积分性质所决定.
\end{hint}

\begin{example}
    考虑 $f:(-\pi,\pi]\to\RR,f(x)=x$ 并将其周期地延拓到 $\RR$. 则我们已知
$$
f\sim\sum_{k=1}^\infty(-1)^{k+1}\frac{2}{k}\sin kx
$$

    又由 $f$ 在 $[-\pi,\pi]$ 上满足 Dini 条件且 $f(\pi+)=-\pi,f(\pi-)=\pi$ 可得
$$
\sum_{k=1}^\infty(-1)^{k+1}\frac{2}{k}\sin kx=\begin{cases}
    x & \abs{x}<\pi\\
    0 & \abs{x}=\pi
\end{cases}
$$

    取 $x=\dfrac{\pi}{2}$ 得
$$
\sum_{n=0}^\infty\frac{(-1)^n}{m+1}=\frac{\pi}{4}
$$
\end{example}

\begin{center}
    \begin{tikzpicture}[scale=0.5]
        \draw[->] (-10,0)--(10,0) node [right] {$x$};
        \draw[->] (0,-4)--(0,4) node [right] {$y$};
        \node at (0,0) [below left] {$O$};
        \draw[-] ({-pi},{-pi})--({pi},{pi});
        \draw[-] ({pi},{-pi})--({3*pi},{pi});
        \draw[-] ({-3*pi},{-pi})--({-pi},{pi});
        \draw[-,dashed] ({-pi},{pi})--({-pi},{-pi});
        \draw[-,dashed] ({pi},{pi})--({pi},{-pi});
        \node at ({pi,0}) [above right] {$\pi$};
        \node at ({-pi,0}) [above right] {$-\pi$};
        \draw[-,red,dashed] ({pi/2},{pi/2})--({pi/2},0) node [below] {$\frac{\pi}{2}$};
    \end{tikzpicture}
\end{center}

\begin{example}
    我们来证明之前用过的一个恒等式
$$
\frac{\sin\pi\alpha}{\pi\alpha}=\prod_{n=1}^\infty\left(1-\frac{\alpha^2}{n^2}\right),\abs{\alpha}\le 1
$$

    当 $\abs{\alpha}=0,1$ 时结论显然成立. 下证 $0<\abs{\alpha}<1$ 的情形.

    考虑以 $2\pi$ 为周期的函数 $f(x)$. 其在 $[-\pi,\pi]$ 上的限制为 $\cos\alpha x$.

    \begin{center}
        \begin{tikzpicture}[yscale=2]
            \draw[->] (-5,0)--(5,0) node [right] {$x$};
            \draw[->] (0,-1)--(0,1.5) node [right] {$y$};
            \node at (0,0) [below left] {$O$};
            \draw[domain={-pi}:{pi}] plot(\x,{cos(\x*0.7 r)});
            \draw[domain={pi}:5] plot(\x,{cos((\x-2*pi)*0.7 r)});
            \draw[domain=-5:{-pi}] plot(\x,{cos((\x+2*pi)*0.7 r)});
            \draw[-,dashed] ({-pi},{cos(pi*0.7 r)})--({-pi},0) node [above] {$-\pi$};
            \draw[-,dashed] ({pi},{cos(pi*0.7 r)})--({pi},0) node [above] {$\pi$};
            \node at (0,1) [above right] {$1$};
            \node at ({pi/(2*0.7)},0) [below left] {$\frac{\pi}{2\alpha}$};
            \node at ({-pi/(2*0.7)},0) [below right] {$-\frac{\pi}{2\alpha}$};
        \end{tikzpicture}
    \end{center}

    易见 $f(x)$ 连续且在任何点处满足 Dini 条件.

    从而其 Fourier 级数在任何点处收敛到 $f(x)$. 即
$$
f(x)=\frac{a_0(f)}{2}+\sum_{n=1}^\infty a_n(f)\cos nx+b_n(f)\sin nx
$$

    由 $f$ 为偶函数知 $b_n(f)=0$. 计算得
$$
a_n(f)=\frac{1}{\pi}(-1)^n\sin\pi\alpha\frac{2\alpha}{\alpha^2-n^2},n\ge 0
$$

    令 $x=\pi$ 得
$$
\begin{aligned}
    \cos\pi\alpha&=\frac{a_0(f)}{2}+\sum_{n=1}^\infty a_n(f)\cos n\pi\\
    &=\frac{\sin\pi\alpha}{\pi\alpha}+\frac{1}{\pi}\sum_{n=1}^\infty\sin\pi\alpha\frac{2\alpha}{\alpha^2-n^2}
\end{aligned}
$$

    则有
$$
\begin{aligned}
    &\pi\frac{\cos\pi\alpha}{\sin\pi\alpha}-\frac{1}{\alpha}=\sum_{n=1}^\infty\frac{2\alpha}{\pi^2-n^2}\\
    \implies&(\ln\sin\pi\alpha-\ln\alpha)'=\sum_{n=1}^\infty\left(\ln\left(1-\frac{\alpha^2}{n^2}\right)\right)'
\end{aligned}
$$

    下面固定 $\beta\in(0,1)$. 注意到右边的函数项级数在 $\alpha\in[0,\beta]$ 上一致收敛. 两边关于 $\alpha$ 在 $[0,\beta]$ 上积分得
$$
\ln\frac{\sin\pi\alpha}{\alpha}\Bigg|_0^\beta=\sum_{n=1}^\infty\ln\left(1-\frac{\alpha^2}{n^2}\right)\Bigg|_0^\beta
$$

    故
$$
\ln\frac{\sin\pi\beta}{\pi\beta}=\sum_{n=1}^\infty\ln\left(1-\frac{\beta^2}{n^2}\right)
$$

    此即
$$
\frac{\sin\pi\beta}{\pi\beta}=\prod_{n=1}^\infty\left(1-\frac{\beta^2}{n^2}\right)
$$
\end{example}