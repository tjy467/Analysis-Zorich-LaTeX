本节与第四节均为前面的抽象理论的应用. 本节我们研究 Euler 的 Beta 函数和 Gamma 函数. 定义如下:
$$
\begin{aligned}
    &B(\alpha,\beta)\triangleq\int_0^1x^{\alpha-1}(1-x)^{\beta-1}\dd x\\
    &\Gamma(\alpha)\triangleq\int_0^\infty x^{\alpha-1}e^{-x}\dd x
\end{aligned}
$$

在这里,我们仅考虑 $\alpha,\beta\in\RR$. 在复分析课程中,我们将会发现它们均可以延拓到 $\mathbb{C}$ 非常大的子集上.

\mysubsection{Beta 函数}

\mysubsubsection{定义域}

若 $\alpha\ge 1$ 且 $\beta\ge 1$,则 $B(\alpha,\beta)$ 为正常积分.

当 $\alpha<1$ 时,$0$ 是积分的奇点,且 $\displaystyle\int_0^\eps x^{\alpha-1}\dd x$ 可积 $\iff\alpha>0$.

当 $\beta<1$ 时,$1$ 是积分的奇点,且 $\displaystyle\int_{1-\eps}^1 x^{\beta-1}\dd x$ 可积 $\iff\beta>0$.

综上 $B(\alpha,\beta)$ 的定义域为 $\set{\alpha>0,\beta>0}$.

\mysubsubsection{对称性}

\begin{property}
    $B(\alpha,\beta)=B(\beta,\alpha)$
\end{property}
\begin{proof}
    令 $x=1-y$.
\end{proof}

\mysubsubsection{递推性质}

\begin{property}
    若 $\alpha>1$,则 $B(\alpha,\beta)=\dfrac{\alpha-1}{\alpha+\beta-1}B(\alpha-1,\beta)$.

    若 $\beta>1$,则 $B(\alpha,\beta)=\dfrac{\beta-1}{\alpha+\beta-1}B(\alpha,\beta-1)$.
\end{property}
\begin{proof}
    由对称性质,只需证明第一条. 设 $\alpha>1$. 则
$$
\begin{aligned}
    B(\alpha,\beta)&=\int_0^1x^{\alpha-1}(1-x)^{\beta-1}\dd x=-\frac{1}{\beta}\int_0^1x^{\alpha-1}\dd(1-x)^\beta\dd x\\
    &=-\frac{1}{\beta}\left[x^{\alpha-1}(1-x)^\beta\biggr |_0^1-\int_0^1(1-x)^\beta(\alpha-1)x^{\alpha-2}\dd x\right]\\
    &=\frac{\alpha-1}{\beta}\int_0^1x^{\alpha-2}(1-x)^\beta\dd x\\
    &=\frac{\alpha-1}{\beta}\left[\int_0^1x^{\alpha-2}(1-x)^{\beta-1}\dd x-\int_0^1x^{\alpha-1}(1-x)^{\beta-1}\dd x\right]\\
    &=\frac{\alpha-1}{\beta}[B(\alpha-1,\beta)-B(\alpha,\beta)]
\end{aligned}
$$

    移项整理即证.
\end{proof}

\begin{inference}
    设 $n,m\in\mathbb{N}$,则
$$
\begin{aligned}
    &B(\alpha,n)=\frac{(n-1)!}{\alpha(\alpha+1)\cdots(\alpha+n-1)}\\
    &B(m,n)=\frac{(m-1)!(n-1)!}{(m+n-1)!}
\end{aligned}
$$
\end{inference}
\begin{proof}
    由 $B(\alpha,1)=\displaystyle\int_0^1x^{\alpha-1}\dd x=\frac{1}{\alpha}$,结合递推式即得第一条结论.

    代入 $\alpha=m$ 即得第二条结论.
\end{proof}

\mysubsubsection{Beta 函数的等价定义}

\begin{property}
$$
B(\alpha,\beta)=\int_0^\infty\frac{x^{\alpha-1}}{(1+x)^{\alpha+\beta}}\dd x
$$
\end{property}
\begin{proof}
    在 $B(\alpha,\beta)=\displaystyle\int_0^1x^{\alpha-1}(1-x)^{\beta-1}\dd x$ 中令 $y=\dfrac{x}{1-x}$ 即可.
\end{proof}

\mysubsection{Gamma 函数}

\mysubsubsection{定义域}

注意到 $0$ 与 $+\infty$ 均为 $\Gamma(\alpha)=\displaystyle\int_0^\infty x^{\alpha-1}e^{-x}\dd x$ 的奇点.

因为 $\displaystyle\int_0^\eps x^{\alpha-1}\dd x$ 收敛 $\iff\alpha>0$,而 $\displaystyle\int_1^\infty x^{\alpha-1}e^{-x}\dd x$ 总收敛,故定义域为 $\set{\alpha>0}$.

\begin{hint}
    在复分析课程中会证明:$\Gamma$ 可以延拓到 $\mathbb{C}\setminus\set{0,-1,-2,\cdots}$ 且 $\Gamma$ 在每个负整数处为一阶极点.
\end{hint}

\mysubsubsection{光滑性与导数公式}

\begin{property}
    $\Gamma$ 在 $(0,+\infty)$ 上无穷阶可导,且
$$
\Gamma^{(n)}(\alpha)=\int_0^\infty x^{\alpha-1}(\ln x)^ne^{-x}\dd x
$$
\end{property}
\begin{proof}
    任意固定 $n\in\mathbb{N},0<\eps<1$. 我们来证上式的右端关于 $\alpha\in\left[\eps,\dfrac{1}{\eps}\right]$ 一致收敛.

    为此仅需证明:$\displaystyle\int_0^1x^{\alpha-1}(\ln x)^ne^{-x}\dd x$ 与 $\displaystyle\int_1^\infty x^{\alpha-1}(\ln x)^ne^{-x}\dd x$ 均在 $\left[\eps,\dfrac{1}{\eps}\right]$ 上一致收敛.

    注意到当 $x\in[0,1],\alpha\in\left[\eps,\dfrac{1}{\eps}\right]$ 时有
$$
\abs{x^{\alpha-1}(\ln x)^ne^{-x}}\le x^{\eps-1}\left(\ln\frac{1}{x}\right)^n
$$

    而 $\displaystyle\int_0^1x^{\eps-1}\left(\ln\frac{1}{x}\right)^n\dd x\xlongequal{y=\ln\frac{1}{x}}\int_0^\infty y^ne^{-\eps y}\dd y<+\infty$.

    从而由 Weierstrass 判别法知 $\displaystyle\int_0^1x^{\alpha-1}(\ln x)^ne^{-x}\dd x$ 一致收敛.

    当 $x\in[1,+\infty),\alpha\in\left[\eps,\dfrac{1}{\eps}\right]$ 时有
$$
\abs{x^{\alpha-1}(\ln x)^ne^{-x}}\le x^{\frac{1}{\eps}-1+n}e^{-x}
$$

    而 $\displaystyle\int_1^\infty x^{\frac{1}{\eps}-1+n}e^{-x}\dd x<+\infty$.

    从而由 Weierstrass 判别法知 $\displaystyle\int_1^\infty x^{\alpha-1}(\ln x)^ne^{-x}\dd x$ 一致收敛.

    综上,$\displaystyle\int_0^\infty x^{\alpha-1}(\ln x)^ne^{-x}\dd x$ 在 $\left[\eps,\dfrac{1}{\eps}\right]$ 上一致收敛.

    由 $\eps>0$ 的任意性以及 $n$ 的任意性,对 $n$ 归纳即证.
\end{proof}

\mysubsubsection{递推公式}

\begin{property}
    对 $\forall\alpha>0$ 有 $\Gamma(\alpha+1)=\alpha\Gamma(\alpha)$.
\end{property}
\begin{proof}
$$
\begin{aligned}
    \Gamma(\alpha+1)=\int_0^\infty x^\alpha e^{-x}\dd x=-\int_0^\infty x^\alpha\dd e^{-x}=\int_0^\infty e^{-x}\dd x^\alpha=\alpha\Gamma(\alpha)
\end{aligned}
$$
\end{proof}

注意到 $\Gamma(1)=\displaystyle\int_0^\infty e^{-x}\dd x=1$. 从而 $\Gamma(n+1)=n!,\forall n\in\mathbb{N}$.

\begin{hint}
    正是通过使用该递推关系式,我们可以将 $\Gamma$ 的定义域延拓到 $\mathbb{C}\setminus\set{0,-1,-2,\cdots}$.
\end{hint}

\mysubsubsection{Euler-Gauss 公式}

\begin{property}
$$
\begin{aligned}
    \Gamma(\alpha)&=\lim_{n\to\infty}n^\alpha\frac{(n-1)!}{\alpha(\alpha+1)\cdots(\alpha+n-1)}\\
    &=\lim_{n\to\infty}n^\alpha B(\alpha,n)
\end{aligned}
$$
\end{property}
\begin{proof}
    令 $x=\ln\dfrac{1}{y}$,则
$$
\Gamma(\alpha)=\int_0^\infty x^{\alpha-1}e^{-x}\dd x=\int_0^1\left(\ln\frac{1}{y}\right)^{\alpha-1}\dd y
$$

    先设 $\alpha\ge 1$. 已知对 $\forall y\in(0,1]$ 有
$$
n(1-y^\frac{1}{n})\uparrow\ln\frac{1}{y}\implies\left[n(1-y^\frac{1}{n})\right]^{\alpha-1}\uparrow\left(\ln\frac{1}{y}\right)^{\alpha-1}
$$

    且 $\displaystyle\int_0^1\left(\ln\frac{1}{y}\right)^{\alpha-1}\dd y$ 收敛. 从而由推论 \ref{1722} 有
$$
\lim_{n\to\infty}\int_0^1\left[n(1-y^\frac{1}{n})\right]^{\alpha-1}\dd y=\int_0^1\left(\ln\frac{1}{y}\right)^{\alpha-1}\dd y=\Gamma(\alpha)
$$

    而
$$
\begin{aligned}
    \int_0^1\left[n(1-y^\frac{1}{n})\right]^{\alpha-1}\dd y&=n^\alpha\int_0^1\frac{1}{n}(1-y^\frac{1}{n})^{\alpha-1}\dd y\\
    &\xlongequal{x=y^\frac{1}{n}}n^\alpha\int_0^1(1-x)^{\alpha-1}x^{n-1}\dd x\\
    &=n^\alpha B(\alpha,n)
\end{aligned}
$$

    从而结论成立.

    现设 $\alpha>0$. 则
$$
\begin{aligned}
    \Gamma(\alpha)&=\frac{\Gamma(\alpha+1)}{\alpha}=\frac{1}{\alpha}\lim_{n\to\infty}n^{\alpha+1}B(\alpha+1,n)\\
    &=\frac{1}{\alpha}\lim_{n\to\infty}n^{\alpha+1}\frac{\alpha}{\alpha+n}B(\alpha,n)=\lim_{n\to\infty}n^\alpha B(\alpha,n)
\end{aligned}
$$
\end{proof}

\mysubsubsection{互补公式}

\begin{property}
    若 $0<\alpha<1$,则 $\Gamma(\alpha)\Gamma(1-\alpha)=\dfrac{\pi}{\sin\pi\alpha}$.
\end{property}
\begin{proof}
    由 Euler-Gauss 公式,我们有
$$
\begin{aligned}
    \Gamma(\alpha)\Gamma(1-\alpha)&=\lim_{n\to\infty}\frac{n^\alpha(n-1)!}{\alpha(\alpha+1)\cdots(\alpha+n-1)}\frac{n^{1-\alpha}(n-1)!}{(1-\alpha)(2-\alpha)\cdots(n-\alpha)}\\
    &=\lim_{n\to\infty}\frac{n}{\alpha}\left(\frac{1}{\displaystyle\left(1+\frac{\alpha}{1}\right)\cdots\left(1+\frac{\alpha}{n-1}\right)}\frac{1}{\displaystyle\left(1-\frac{\alpha}{1}\right)\cdots\left(1-\frac{\alpha}{n-1}\right)}\right)\frac{1}{n-\alpha}\\
    &=\lim_{n\to\infty}\frac{1}{\alpha}\frac{1}{\displaystyle\prod_{j=1}^{n-1}\left(1-\frac{\alpha^2}{j^2}\right)}\\
    &=\frac{1}{\alpha}\frac{1}{\displaystyle\prod_{j=1}^\infty\left(1-\frac{\alpha^2}{j^2}\right)}\\
\end{aligned}
$$

    在第十八章,我们将会证明
$$
\frac{\sin\pi\alpha}{\pi\alpha}=\prod_{n=1}^\infty\left(1-\frac{\alpha^2}{n^2}\right)
$$

    从而结论成立.
\end{proof}

作为应用,$\Gamma\left(\dfrac{1}{2}\right)=\sqrt{\pi}$. 从而
$$
\sqrt{\pi}=\int_0^\infty x^{-\frac{1}{2}}e^{-x}\dd x\xlongequal{x=u^2}2\int_0^\infty e^{-u^2}\dd u=\int_{-\infty}^{+\infty}e^{-u^2}\dd u
$$

从而我们再次证明了 Euler-Poisson 公式.

\mysubsection{Beta 函数与 Gamma 函数的关系}

\begin{property}
    对 $\forall\alpha,\beta>0$ 有
$$
B(\alpha,\beta)=\frac{\Gamma(\alpha)\Gamma(\beta)}{\Gamma(\alpha+\beta)}
$$
\end{property}
\begin{proof}
    在 $\Gamma(\alpha)=\displaystyle\int_0^\infty x^{\alpha-1}e^{-x}\dd x$ 中做变量替换 $x=\widetilde{x}y,y>0$ 可得
$$
\begin{aligned}
    \Gamma(\alpha)&=\int_0^\infty(\widetilde{x}y)^{\alpha-1}e^{-\widetilde{x}y}\dd\widetilde{x}y\\
    &=y^\alpha\int_0^\infty x^{\alpha-1}e^{-xy}\dd x\\
    \implies\Gamma(\alpha+\beta)&=(1+y)^{\alpha+\beta}\int_0^\infty x^{\alpha+\beta-1}e^{-x(y+1)}\dd x\\
    \implies\frac{\Gamma(\alpha+\beta)y^{\beta-1}}{(1+y)^{\alpha+\beta}}&=y^{\beta-1}\int_0^\infty x^{\alpha+\beta-1}e^{-(y+1)x}\dd x
\end{aligned}
$$

    现设 $\alpha>1,\beta>1$,两边积分并利用右边的积分可交换次序(见之前的例子)可得
$$
\begin{aligned}
    \Gamma(\alpha+\beta)B(\alpha,\beta)&=\int_0^\infty\int_0^\infty y^{\beta-1}e^{-(y+1)x}x^{\alpha+\beta-1}\dd y\dd x\\
    &=\left(\int_0^\infty x^{\alpha-1}e^{-x}\dd x\right)\left(\int_0^\infty (xy)^{\beta-1}e^{-xy}\dd xy\right)\\
    &=\Gamma(\alpha)\Gamma(\beta)
\end{aligned}
$$

    对 $\alpha<1$ 或 $\beta<1$ 的情形,由递推公式有
$$
\begin{aligned}
    B(\alpha,\beta)&=\frac{\alpha+\beta}{\alpha}\frac{\alpha+\beta+1}{\beta}B(\alpha+1,\beta+1)\\
    &=\frac{\alpha+\beta}{\alpha}\frac{\alpha+\beta+1}{\beta}\frac{\Gamma(\alpha+1)\Gamma(\beta+1)}{\Gamma(\alpha+\beta+2)}\\
    &=\frac{\Gamma(\alpha)\Gamma(\beta)}{\Gamma(\alpha+\beta)}
\end{aligned}
$$
\end{proof}

\mysubsection{几个例子}

\begin{example}
$$
\int_0^\frac{\pi}{2}\sin^{\alpha-1}\varphi\cos^{\beta-1}\varphi\dd\varphi=\frac{1}{2}B\left(\frac{\alpha}{2},\frac{\beta}{2}\right)
$$
\end{example}
\begin{proof}
    令 $x=\sin^2\varphi$. 则 $\cos\varphi=(1-x)^\frac{1}{2}$. 从而
$$
\begin{aligned}
    \int_0^\frac{\pi}{2}\sin^{\alpha-1}\varphi\cos^{\beta-1}\varphi\dd\varphi&=\int_0^1 x^\frac{\alpha-1}{2}(1-x)^{\beta-1}{2}\frac{\dd x}{2x^\frac{1}{2}(1-x)^\frac{1}{2}}\\
    &=\frac{1}{2}\int_0^1x^{\frac{\alpha}{2}-1}(1-x)^{\frac{\beta}{2}-1}\dd x\\
    &=\frac{1}{2}B\left(\frac{\alpha}{2},\frac{\beta}{2}\right)
\end{aligned}
$$

    作为特例,有
$$
\int_0^\frac{\pi}{2}\sin^{\alpha-1}\varphi\dd\varphi=\int_0^\frac{\pi}{2}\cos^{\alpha-1}\varphi\dd\varphi=\frac{1}{2}B\left(\frac{\alpha}{2},\frac{1}{2}\right)=\frac{\sqrt{\pi}}{2}\frac{\Gamma\left(\dfrac{\alpha}{2}\right)}{\Gamma\left(\dfrac{\alpha+1}{2}\right)}
$$
\end{proof}

\begin{example}
    已知若 $n$ 维球体的半径为 $r$,则其体积为
$$
V_n(r)=c_nr^n
$$

    以下我们用 $\Gamma$ 函数来表达 $c_n$.

    由 Fubini 定理
$$
\begin{aligned}
    V_n(r)&=c_nr^n=\int_{-r}^rc_{n-1}\left(\sqrt{r^2-s^2}\right)^{n-1}\dd s\\
    &=2c_{n-1}\int_0^r\left(\sqrt{r^2-s^2}\right)^{n-1}\dd s\\
    &\xlongequal{s=rx^\frac{1}{2}}c_{n-1}r^n\int_0^1(1-x)^\frac{n-1}{2}x^{-\frac{1}{2}}\dd x\\
    &=c_{n-1}r^nB\left(\frac{n+1}{2},\frac{1}{2}\right)
\end{aligned}
$$

    从而
$$
\frac{c_n}{c_{n-1}}=B\left(\frac{n+1}{2},\frac{1}{2}\right)=\frac{\Gamma\left(\dfrac{1}{2}\right)\Gamma\left(\dfrac{n+1}{2}\right)}{\Gamma\left(\dfrac{n+2}{2}\right)}=\sqrt{\pi}\frac{\Gamma\left(\dfrac{n+1}{2}\right)}{\Gamma\left(\dfrac{n+2}{2}\right)}
$$

    故
$$
\begin{aligned}
    c_n&=\frac{c_n}{c_{n-1}}\cdots\frac{c_2}{c_1}c_1\\
    &=\pi^\frac{n-1}{2}\frac{\Gamma\left(\dfrac{n+1}{2}\right)}{\Gamma\left(\dfrac{n+2}{2}\right)}\cdots\frac{\Gamma\left(\dfrac{3}{2}\right)}{\Gamma\left(\dfrac{4}{2}\right)}\cdot 2\\
    &=\frac{\pi^\frac{n}{2}}{\Gamma\left(\dfrac{n}{2}+1\right)}
\end{aligned}
$$

    由此即得
$$
V_n(r)=\frac{\pi^\frac{n}{2}}{\Gamma\left(\dfrac{n}{2}+1\right)}r^n
$$

    根据几何意义可知
$$
\dd V_n(r)=S_{n-1}(r)\dd r
$$

    其中 $S_{n-1}(r)$ 是半径为 $r$ 的 $n-1$ 维球面的面积. 从而
$$
S_{n-1}(r)=\frac{\dd V_n(r)}{\dd r}=\frac{2\pi^\frac{n}{2}}{\Gamma\left(\dfrac{n}{2}\right)}r^{n-1}
$$
\end{example}