\mysubsection{逐点收敛的定义与例子}

\begin{definition}
    设 $X$ 为非空集合. $f_n:X\to\RR$ 是一列函数.

    设 $x\in X$. 称 $f_n$ 在 $x$ 处收敛,若极限
$$
\lim_{n\to\infty}f_n(x)
$$

    存在.
\end{definition}

\begin{definition}
    令 $E\triangleq\set{x\in X|\lim_{n\to\infty}f_n(x)\text{ 存在}}$.

    称 $E$ 为函数列 $\set{f_n}$ 的收敛集(收敛域).
\end{definition}

\begin{definition}
    设 $E$ 为 $\set{f_n}$ 的收敛集. 定义 $f:E\to\RR$ 为
$$
f(x)\triangleq\lim\limits_{n\to\infty}f_n(x)
$$

    称 $f$ 为 $\set{f_n}$ 在收敛集 $E$ 上的极限.
\end{definition}

\begin{definition}
    设 $f_n,f$ 均为 $X$ 上的实值函数.

    若 $\forall x\in X,f_n(x)\to f(x)$,则称 $f_n$ 逐点收敛到 $f$. 记为 $f_n\to f$.
\end{definition}

\begin{example}
    $f_n(x)=x^n$

    则 $\set{f_n}$ 的收敛集为 $E=(-1,1]$,且其极限函数为
$$
f(x)=\begin{cases}
    0 & x\in(-1,1)\\
    1 & x=1
\end{cases}
$$
\end{example}

\begin{example}
    $f_n(x)=\dfrac{\sin n^2x}{n}$
\end{example}

\begin{example}
    $f_n(x)=\dfrac{\sin nx}{n^2}$
\end{example}

\begin{example}
    $f_n(x)=2(n+1)x(1-x^2)^n,X=[0,1]$
\end{example}

\begin{example}
    $f_n(x)=\lim_{m\to\infty}\cos(n!\pi x)^{2m}$

    \vspace{0.2em}

    可见 $f_n(x)\to D(x)$,其中 $D(x)$ 为 Dirichlet 函数.
\end{example}

\mysubsection{序列收敛的基本问题}

在分析中,我们经常取极限. 而我们经常关心的问题是:序列的性质在取完极限后是否得到保持?

例如有函数列 $f_n:\RR\to\RR$ 以及 $f:\RR\to\RR$ 满足 $f_n\to f$.

则有以下几个基本问题:

\begin{enumerate}
    \item 若 $f_n$ 均连续,则是否有 $f$ 连续?
    
    \item 若 $f_n$ 均可到,则是否有 $f$ 可导?进一步是否有 $f_n'\to f'$?
    
    \item 若 $f_n\in\mathcal{R}[a,b]$,则是否有 $f\in\mathcal{R}[a,b]$?进一步是否有 $\displaystyle\int_a^bf_n(x)\dd x\to\int_a^bf(x)\dd x$?
\end{enumerate}

以下,我们用第 $1$ 节的例子来证明:在没有进一步假设的情况下,以上问题的答案均可能为否定.

\begin{example}
    $f_n(x)=x^n$ 均在 $[0,1]$ 上连续,但其极限 $f(x)=\begin{cases}
        0 & x\in[0,1)\\
        1 & x=1
    \end{cases}$ 不连续.
\end{example}

\begin{example}
    $f_n(x)=\dfrac{\sin n^2x}{n}\to f(x)=0$.

    \vspace{0.2em}

    但 $f'_n(x)=n\cos n^2x\not\to f'(x)=0$
\end{example}

\begin{example}
    $f_n(x)=\dfrac{\sin nx}{n^2}\to f(x)=0$

    \vspace{0.2em}

    且 $f'_n(x)=\dfrac{\cos nx}{n}\to f'(x)=0$
\end{example}

\begin{example}
    $f_n(x)=2(n+1)x(1-x^2)^n\to f(x)=0,\forall x\in X=[0,1]$.

    \vspace{0.2em}

    但 $\displaystyle\int_0^1f_n(x)\dd x=1\not\to\int_0^1f(x)\dd x=0$
\end{example}

\begin{example}
    $f_n(x)=\lim\limits_{m\to\infty}(\cos n!\pi x)^{2m},x\in[0,1]$ 仅在有限个点上非零,从而 $f_n\in\mathcal{R}[0,1]$.

    \vspace{0.2em}

    但 $f_n(x)\to D(x)$,而 $D(x)\notin\mathcal{R}[0,1]$.
\end{example}

\begin{example}
    在分析中一个十分重要的特例如下:

    设 $I=[a,b]$. 设 $f_n:I\to\RR$ 为 $C^{(1)}$ 光滑.

    则有以下问题:

    \begin{enumerate}
        \item 形式函数项级数 $\sum\limits_{n=1}^\infty f_n(x)$ 在哪些点收敛?
        
        \item 若其在 $[a,b]$ 上收敛,其极限 $S(x)$ 是否为 $C^{(1)}$ 光滑?
        
        \item 若 $S(x)\in C^{(1)}(I)$,是否有
$$
S'(x)=\left(\sum_{n=1}^\infty f_n(x)\right)'=\sum_{n=1}^\infty f_n'(x)
$$
    \end{enumerate}

    用刚才的术语来描述这个问题,可以这样来表达:令
$$
S_n(x)\triangleq\sum_{j=1}^nf_j(x)
$$

    则 $S_n(x)\in C^{(1)}(I)$. 上面三个问题变为:

    \begin{enumerate}
        \item $S_n(x)$ 是否在 $[a,b]$ 上逐点收敛?
        
        \item 若 $S_n(x)\to S(x)$,是否有 $S\in C^{(1)}(I)$?
        
        \item 若 $S\in C^{(1)}(I)$,是否有 $S_n'(x)\to S'(x)$?
    \end{enumerate}

    基于我们之前的讨论,我们知道:如上问题的答案均可能为否定!
\end{example}

\begin{example}
    以上的问题有一个正面的结论:

    设 $\sum\limits_{n=0}^\infty a_nx^n$ 为形式幂级数. 设其收敛半径
$$
R=\frac{1}{\displaystyle\overline{\lim\limits_{n\to\infty}}\sqrt[n]{\abs{a_n}}}>0
$$

    则 $\sum\limits_{n=0}^\infty a_nx^n$ 在 $(-R,R)$ 上绝对收敛,$C^{(\infty)}$ 光滑且
$$
\left(\sum_{n=0}^\infty a_nx^n\right)'=\sum_{n=1}^\infty na_nx^{n-1}
$$

    本章,我们还会就幂级数展开做更深入的讨论.
\end{example}

\mysubsection{函数族的收敛与一致收敛}

本节我们定义函数列的一致收敛. 且为了更广泛的应用,我们讨论一族函数的一致收敛性.

\begin{definition}
    设 $X,T$ 为非空集合,函数 $F:X\times T\to\RR$. 在固定 $t\in T$ 时,我们使用记号
$$
f_t:X\to\RR,f_t(x)\triangleq F(x,t)
$$

    由此我们得到一族 $X$ 上的函数 $\set{f_t|t\in T}$.

    我们称 $T$ 为参数集(参数域).
\end{definition}

在本课程中,$T$ 绝大多数时候取为 $\mathbb{N},\RR,\mathbb{C}$. 但从定义的角度,其可以是任何非空集合.

在这个更广泛的框架下,首先我们可以重复之前逐点收敛的定义.

\begin{definition}
    设 $\set{f_t:X\to\RR|t\in T}$ 是一族函数. 设 $\mathcal{B}$ 是 $T$ 的一个基.

    设 $x\in X$. 若 $\lim\limits_\mathcal{B}f_t(x)$ 存在,则称 $\set{f_t}$ 在 $x$ 处收敛. 定义
$$
E\triangleq\set{x\in X|\lim\limits_\mathcal{B}f_t(x)\text{ 存在}}
$$

    称 $E$ 为函数族 $\set{f_t|t\in T}$ 的收敛集(收敛域).

    定义 $f:E\to\RR$ 为
$$
f(x)\triangleq\lim\limits_\mathcal{B}f_t(x)
$$

    称 $f$ 为 $\set{f_t}$ 在 $E$ 上的极限函数.
\end{definition}

\begin{definition}
    设 $\set{f_t|t\in T}$ 是 $X$ 上的一族函数. 设 $\mathcal{B}$ 是 $T$ 的一个基.

    设 $E\subset X$,函数 $f:E\to\RR$. 称 $\set{f_t}$ 在 $E$ 上逐点收敛到 $f$,若
$$
\lim_\mathcal{B}f_t(x)=f(x),\forall x\in E
$$

    记为(在 $E$ 上)$f_t\xrightarrow[\mathcal{B}]{}f$.
\end{definition}

\begin{definition}
    设 $\set{f_t:X\to\RR,t\in T}$ 是一族函数. 设 $\mathcal{B}$ 是 $T$ 的一个基.

    设 $E\subset X$,函数 $f:E\to\RR$. 称 $f_t$ 在 $E$ 上一致收敛到 $f$,若
$$
\forall\eps>0,\exists B\in\mathcal{B},\forall t\in B,\forall x\in E,\abs{f_t(x)-f(x)}<\eps
$$

    记为(在 $E$ 上)$f_t\underset{\mathcal{B}}{\rightrightarrows}f$.
\end{definition}