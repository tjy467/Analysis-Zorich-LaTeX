本节,我们证明三个重要的积分公式. 从证明过程可以看出它们都是 Newton-Leibniz 公式的推论.

\mysubsection{Green 公式}

\begin{theorem}[Green 公式]
    设 $D\subset\RR^2$ 为区域,$\overline{D}$ 紧,$\partial D=\partial\overline{D}$ 为分片光滑曲线.

    设 $P,Q:\overline{D}\to\RR$ 为 $C^{(1)}$ 光滑. 则
$$
\int_D\left(\pard{Q}{x}-\pard{P}{y}\right)\dd x\dd y=\int_{\partial D}P\dd x+Q\dd y
$$

    其中 $D$ 取标准定向 $(e_1,e_2)$,而 $\partial D$ 上取由 $\overline{D}$ 的定向所诱导的定向.
\end{theorem}

我们仅对一类特殊的区域 $D$ 证明上式. 从实用的角度来说,这一类区域已经够用了. 以下我们给出定义

\begin{definition}
    称 $D\subset\RR^2$ 是一个 $x$-型区域,若存在 $\varphi_1,\varphi_2:[a,b]\to\RR$ 连续且分段光滑,满足 $\varphi_1(x)\le\varphi_2(x),\forall x\in [a,b]$,使得
$$
\overline{D}=\set{(x,y)\in\RR^2|x\in[a,b],\varphi_1(x)\le y\le\varphi_2(x)}
$$
\end{definition}

\img{0.8}{13.3.1.png}

类似地可以定义 $y$-型区域. 我们省略其定义.

\begin{lemma}
    设 $D$ 为 $x$-型区域,且由 $\varphi_1,\varphi_2:[a,b]\to\RR$ 给定. 设 $P\in C^{(1)}(\overline{D})$. 则
$$
\int_{\partial D}P\dd x=\int_D-\pard{P}{y}\dd x\dd y
$$
\end{lemma}
\begin{proof}
    一方面,由 Fubini 定理有
$$
\begin{aligned}
    \int_D-\pard{P}{y}\dd x\dd y&=-\int_a^b\int_{\varphi_1(x)}^{\varphi_2(x)}\pard{P}{y}(x,y)\dd y\dd x\\
    &\xlongequal{\text{N-L}}-\int_a^b\left(P(x,\varphi_2(x))-P(x,\varphi_1(x))\right)\dd x\\
    &=\int_a^b P(x,\varphi_1(x))\dd x-\int_a^b P(x,\varphi_2(x))\dd x
\end{aligned}
$$

\img{0.5}{13.3.2.png}

    另一方面,$\partial D=\gamma_1\cup\gamma_2\cup\gamma_3\cup\gamma_4$ 如图. 其由 $\overline{D}$ 诱导的定向为逆时针方向. 我们分别给出它们的参数化
$$
\begin{aligned}
    &\gamma_1:\quad\phi_1:[a,b]\to\RR^2,t\mapsto(t,\varphi_1(t))\\
    &\gamma_2:\quad\phi_2:[\varphi_1(b),\varphi_2(b)]\to\RR^2,t\mapsto(b,t)\\
    &\gamma_3:\quad\phi_3:[a,b]\to\RR^2,t\mapsto(t,\varphi_2(t))\\
    &\gamma_4:\quad\phi_4:[\varphi_1(a),\varphi_2(a)]\to\RR^2,t\mapsto(a,t)\\
\end{aligned}
$$

    其中 $\phi_1,\phi_2$ 诱导了 $\gamma_1,\gamma_2$ 上的定向,而 $\phi_3,\phi_4$ 诱导了 $\gamma_3,\gamma_4$ 上相反的定向. 从而
$$
\begin{aligned}
    &\int_{\gamma_1}P\dd x=\int_a^bP(t,\varphi_1(t))\dd t\\
    &\int_{\gamma_2}P\dd x=0\\
    &\int_{\gamma_3}P\dd x=-\int_a^bP(t,\varphi_2(t))\dd t\\
    &\int_{\gamma_4}P\dd x=0
\end{aligned}
$$

    综上
$$
\int_{\partial D}P\dd x=\int_D-\pard{P}{y}\dd x\dd y
$$
\end{proof}

\begin{lemma}\label{green:x}
    设 $D\subset\RR^2$ 为区域,存在有限个不交的 $x$-型区域 $D_1,\cdots,D_k$ 使得 
$$
\overline{D}=\overline{D}_1\cup\cdots\cup\overline{D}_k
$$

    设 $P\in C^{(1)}(\overline{D})$,则
$$
\int_{\partial D}P\dd x=\int_D-\pard{P}{y}\dd x\dd y
$$
\end{lemma}
\begin{proof}
    一方面
$$
\begin{aligned}
    \int_D-\pard{P}{y}\dd x\dd y&=\sum_{i=1}^k-\int_{D_i}\pard{P}{y}\dd x\dd y\\
    &=\sum_{i=1}^k\int_{\partial D_i}P\dd x
\end{aligned}
$$

    另一方面,已知若 $D_i$ 与 $D_j$ 有公共边界,则 $D_i$ 与 $D_j$ 在公共边界上诱导相反的定向. 故
$$
\sum_{i=1}^k\int_{\partial D}P\dd x=\int_{\partial D}P\dd x
$$
\end{proof}

\img{0.8}{13.3.3.png}

同理可以证明

\begin{lemma}\label{green:y}
    设 $D\subset\RR^2$ 为区域,存在有限个不交的 $y$-型区域 $G_1,\cdots,G_k$ 使得 
$$
\overline{D}=\overline{G}_1\cup\cdots\cup\overline{G}_k
$$

    设 $Q\in C^{(1)}(\overline{D})$,则
$$
\int_{\partial D}Q\dd y=\int_D\pard{Q}{x}\dd x\dd y
$$
\end{lemma}

因此我们做出如下的定义

\begin{definition}
    称 $D$ 是一个初等区域,若存在有限个 $x$-型区域 $D_1,\cdots,D_k$ 满足
$$
D_i\cap D_j=\varnothing,\forall i\ne j~\text{且}~\overline{D}=\bigcup_{i=1}^k\overline{D}_i
$$

    且存在有限个 $y$-型区域 $G_1,\cdots,G_k$ 满足
$$
    G_i\cap G_j=\varnothing,\forall i\ne j~\text{且}~\overline{D}=\bigcup_{i=1}^k\overline{G}_i
$$
\end{definition}

\img{0.8}{13.3.4.png}

将引理 \ref{green:x} 与引理 \ref{green:y} 相加,我们即对初等区域证明了 Green 公式.

\begin{hint}
    \begin{enumerate}
        \item 我们可以这样来看 Green 公式:
        
        令 $\omega=P\dd x+Q\dd y$ 为 $\overline{D}$ 上的 $1$-形式. 则
$$
\dd\omega=-\pard{P}{y}\dd x\wedge\dd y+\pard{Q}{x}\dd x\wedge\dd y
$$

        从而 Green 公式变为
$$
\int_{\partial D}\omega=\int_D\dd\omega
$$

        \item 这种观察还可以启发我们以另一种方式来证明 Green 公式:设 $I$ 为矩形区域,则其既是 $x$-型区域也是 $y$-型区域. 从而我们有
$$
\int_I\dd\omega=\int_{\partial I}\omega
$$

        其中 $\omega=P\dd x+Q\dd y$.

        若区域 $\overline{D}$ 有参数化 $\varphi:\overline{I}\to\overline{D},\omega=P\dd x+Q\dd y$ 为 $\overline{D}$ 上的 $1$-形式,则
$$
\int_{\overline{D}}\dd\omega=\int_{I}\varphi^*(\dd\omega)=\int_I\dd\varphi^*(\omega)\xlongequal{\text{Green 公式}}\int_{\partial I}\varphi^*\omega=\int_{\partial D}\omega
$$

        即通过以矩形上的 Green 公式为桥梁,我们证明了 $\overline{D}$ 上的 Green 公式.

        若进一步 $\overline{D}$ 可以分割成有限个如上类型区域的并 $\overline{D}=\overline{D}_1\cup\cdots\cup\overline{D}_k$,则
$$
\int_{\overline{D}}\dd\omega=\sum_{i=1}^k\int_{\overline{D}_i}\dd\omega=\sum_{i=1}^k\int_{\partial D_i}\omega=\int_{\partial D}\omega
$$

        最后一步再次使用了观察:若 $\overline{D}_i\cap\overline{D}_j\ne\varnothing$,则 $D_i$ 与 $D_j$ 在公共边界上诱导了相反的定向.
    \end{enumerate}
\end{hint}

以下来看几个应用实例.

\begin{example}
    设 $D$ 为初等区域. 取 $P=-y,Q=x$,则有
$$
\begin{aligned}
    &\int_{\partial D}-y\dd x=\int_D1\dd x\dd y=S(D)\\
    &\int_{\partial D}x\dd y=\int_D1\dd x\dd y=S(D)
\end{aligned}
$$
\end{example}

\begin{example}[ 椭圆的面积]
\end{example}

\begin{example}[ Brouwer 不动点定理]
    若 $D$ 为单位圆盘 $\set{(x,y)\in\RR^2|x^2+y^2<1}$,而 $f:\overline{D}\to\overline{D}$ 为 $C^{(1)}$ 光滑映射,则 $f$ 必有不动点.
\end{example}
\begin{proof}
    反证. 设 $f(x)\ne x,\forall x\in\overline{D}$.

    设 $\varphi:\overline{D}\to\partial{D},\varphi(x)$ 为从 $f(x)$ 指向 $x$ 的射线与 $\partial D$ 的交点.
    
    取 $\omega = \dfrac{-y\dd x+x\dd y}{x^2+y^2} \in\Omega^1(\partial D)$. 则 $\varphi^*\omega\in\Omega^1(\overline{D})$. 从而
$$
\int_{\partial D}\varphi^*\omega=\int_{\overline{D}}\dd\varphi^*\omega=\int_{\overline{D}}\varphi^*\dd\omega=0
$$

    另一方面由 $\varphi|_{\partial D}=\mathrm{id}_{\partial D}$ 知
$$
\int_{\partial D}\varphi^*\omega=\int_{\partial D}\omega\ne 0
$$

    矛盾,故 $f$ 必有不动点.
\end{proof}

\begin{hint}
    事实上,仅需 $f$ 连续即可证明不动点存在.
\end{hint}

\mysubsection{Gauss-Ostrogradskii 公式}

\begin{theorem}[Gauss-Ostrogradskii 公式]
    设 $D\subset\RR^3$ 为开区域,$\overline{D}$ 紧,$\partial D$ 为分片光滑曲面.

    在 $D$ 上取标准定向 $(e_1,e_2,e_3)$,在 $\partial D$ 上取由 $D$ 的定向诱导的定向. 设 $P,Q,R\in C^{(1)}(\overline{D})$,则
$$
\int_{\partial D}P\dd y\wedge\dd z+Q\dd z\wedge\dd x+R\dd x\wedge\dd y=\int_{\overline{D}}\left(\pard{P}{x}+\pard{Q}{y}+\pard{R}{z}\right)\dd x\dd y\dd z
$$
\end{theorem}

与 Green 公式的证明类似,我们仅对一类特殊的区域来证明:

\begin{definition}
    称 $D$ 是一个 $z$-型区域,若存在区域 $G\subset\RR^2$ 满足 $\overline{G}$ 紧,以及连续且分片光滑函数 $\varphi_1,\varphi_2:\overline{G}\to\RR$,满足 $\varphi_1(p)\le\varphi_2(p),\forall p\in G$,且
$$
\overline{D}=\set{(x,y,z)\in\RR^3|(x,y)\in\overline{G},\varphi_1(x,y)\le z\le\varphi_2(x,y)}
$$
\end{definition}

类似可定义 $x$-型与 $y$-型区域.

\begin{definition}
    设 $D$ 为初等区域,若 $\overline{D}$ 可以分解成有限个 $x$-型($y$-型,$z$-型)区域的并,且内部不交.
\end{definition}

\begin{lemma}
    设 $D$ 为如上定义的 $z$-型区域,且 $R\in C^{(1)}(\overline{D})$. 则有
$$
\int_{\partial D}R\dd x\wedge\dd y=\int_D\pard{R}{z}\dd x\dd y\dd z
$$
\end{lemma}
\begin{proof}
    一方面,由 Fubini 定理知
$$
\begin{aligned}
    &\int_D\pard{R}{z}\dd x\dd y\dd z\\
    =&\int_G\int_{\varphi_1(x,y)}^{\varphi_2(x,y)}\pard{R}{z}(x,y,z)\dd z\dd x\dd y\\
    =&\int_G(R(x,y,\varphi_2(x,y))-R(x,y,\varphi_1(x,y)))\dd x\dd y
\end{aligned}
$$

    另一方面,$\partial D$ 由 $S_1,S_2$ 与 $S$ 三部分组成. 其定向如图所示(由法向量给定).

    \img{0.6}{13.3.5.png}

    $S_1$ 与 $S_2$ 有参数化
$$
S_1:(x,y)\mapsto(x,y,\varphi_1(x,y))
S_2:(x,y)\mapsto(x,y,\varphi_2(x,y))
$$

    其中 $S_1$ 的参数化诱导了与 $S_1$ 相反的定向,$S_2$ 的参数化诱导了与 $S_2$ 相同的定向.

    设 $\gamma:[a,b]\to\RR^2$ 为 $\partial G$ 的参数化.

    则 $S$ 有参数化
$$
\psi:\widetilde{D}\to\RR^3,(t,z)\mapsto(\gamma_1(t),\gamma_2(t),z)
$$

    其中 $\widetilde{D}=\set{(t,z)\in\RR^2|t\in[a,b],\varphi_1(\gamma(t))\le z\le\varphi_2(\gamma(t))}$. 则有
$$
\begin{aligned}
    &\int_{S_1}R\dd x\wedge\dd y=-\int_GR(x,y,\varphi_1(x,y))\dd x\dd y\\
    &\int_{S_2}R\dd x\wedge\dd y=\int_GR(x,y,\varphi_2(x,y))\dd x\dd y\\
    &\int_{S_3}R\dd x\wedge\dd y=\int_{\widetilde{D}}R(\psi(t,z))\dd\gamma_1(t)\wedge\dd\gamma_2(t)=0
\end{aligned}
$$

    综上
$$
\int_D\pard{R}{z}\dd x\dd y\dd z=\int_{\partial D}R\dd x\wedge\dd y
$$
\end{proof}

设 $D$ 为初等区域. 与 Green 公式的证明类似,由 $D$ 可以分解成有限个 $z$-型区域的并,可得
$$
\int_D\pard{R}{z}\dd x\dd y\dd z=\int_{\partial D}R\dd x\wedge\dd y
$$

类似的
$$
\begin{aligned}
    \int_D\pard{P}{x}\dd x\dd y\dd z=\int_{\partial D}P\dd y\wedge\dd z\\
    \int_D\pard{Q}{y}\dd x\dd y\dd z=\int_{\partial D}Q\dd z\wedge\dd x
\end{aligned}
$$

三式相加,即对初等区域 $D$ 证明了 G-O 公式.

\begin{hint}
    令 $\omega=P\dd y\wedge\dd z+Q\dd z\wedge\dd x+R\dd x\wedge\dd y$. 则
$$
\dd\omega=\left(\pard{P}{x}+\pard{Q}{y}+\pard{R}{z}\right)\dd x\wedge\dd y\wedge\dd z
$$

    从而 G-O 公式可以写成如下的抽象形式
$$
\int_{\partial D}\omega=\int_{\overline{D}}\dd\omega
$$
\end{hint}

\begin{example}
    取 $P=x,Q=y,R=z$ 可知
$$
V(D)=\int_{\partial D}x\dd y\wedge\dd z=\int_{\partial D}y\dd z\wedge\dd x=\int_{\partial D}z\dd x\wedge\dd y
$$
\end{example}

\begin{example}[ 椭球的体积]
\end{example}

\begin{example}[ 浮力定律(Archimedes)]
    设水面为平面 $z=0$. 设物体 $D$ 浸入水中,其表面 $S$ 会受到水的压力.

    \img{0.5}{13.3.6.png}

    在水深为 $z$ 处的一个小片上水的压强为 $\rho g(-z)$,其方向为 $-\mathbf{n}$,其中 $\mathbf{n}$ 为 $S$ 的外法向. 从而小片受到的压力为
$$
\rho g(-z)(-\mathbf{n})\dd\sigma=\rho gz\mathbf{n}\dd\sigma
$$

    已知 $\mathbf{n}\dd\sigma=(\dd y\wedge\dd z,\dd z\wedge\dd x,\dd x\wedge\dd y)$.

    从而物体所受的压力为
$$
\begin{aligned}
    F&=\int_S\rho g z\mathbf{n}\dd\sigma\\
    &=\left(\int_S\rho g z\dd y\wedge\dd z,\int_S\rho g z\dd z\wedge\dd x,\int_S\rho g z\dd x\wedge\dd y\right)\\
    &=\left(0,0,\int_D\rho g\dd x\wedge\dd y\wedge\dd z\right)\\
    &=(0,0,\rho g V)
\end{aligned}
$$

    即物体收到竖直向上的浮力,其大小等于物体所排开水的重力.
\end{example}

\mysubsection{Stokes 公式}

\begin{theorem}[Stokes 公式]
    设 $G\subset\RR^3$ 为开集,$P,Q,R\in C^{(1)}(G)$.
    
    设 $S\subset G$ 为紧 $2$ 维分片光滑可定向带边曲面,则如下公式成立
$$
\begin{aligned}
    \int_{\partial S}P\dd x+Q\dd y+R\dd z=\int_S&\left(\pard{R}{y}-\pard{Q}{z}\right)\dd y\wedge\dd z\\
    &+\left(\pard{P}{z}-\pard{R}{x}\right)\dd z\wedge\dd x\\
    &+\left(\pard{Q}{x}-\pard{P}{y}\right)\dd x\wedge\dd y
\end{aligned}
$$
\end{theorem}

我们仅对一类特殊的带边曲面证明 Stokes 公式.

\begin{definition}
    称 $S$ 为 $z$-型曲面,若存在开集 $D\subset\RR^2,\overline{D}$ 紧,$\partial D$ 为分片光滑曲线以及 $\varphi:\overline{D}\to\RR$ 连续且分片光滑使得
$$
S=\Gamma(\varphi)=\set{(x,y,\varphi(x,y))|(x,y)\in\overline{D}}
$$
\end{definition}

\img{0.5}{13.3.7.png}

类似地可以定义 $x$-型与 $y$-型曲面.

\begin{lemma}
    若 $S$ 为 $z$-型曲面,则 Stokes 公式成立.
\end{lemma}
\begin{proof}
    记 $\omega=P\dd x+Q\dd y+R\dd z$. 则
$$
\dd\omega=\left(\pard{R}{y}-\pard{Q}{z}\right)\dd y\wedge\dd z+\left(\pard{P}{z}-\pard{R}{x}\right)\dd z\wedge\dd x+\left(\pard{Q}{x}-\pard{P}{y}\right)\dd x\wedge\dd y
$$

    Stokes 公式即为 $\displaystyle\int_S\dd\omega=\int_{\partial S}\omega$. 以下我们将其转化为平面上的 Green 公式.
    
    由 $S$ 为 $\varphi$ 的图象知
$$
\Gamma:\overline{D}\to S,(x,y)\mapsto(x,y,\varphi(x,y))
$$

    为 $S$ 的参数化. 从而
$$
\int_S\dd\omega=\int_{\overline{D}}\Gamma^*(\dd\omega)=\int_{\overline{D}}\dd(\Gamma^*\omega)
$$

    不妨设 $\partial D$ 可以参数化(否则可以考虑每个段). 设 $\gamma:[a,b]\to\partial D$ 为 $\partial D$ 的参数化.

    从而 $\Gamma\circ\gamma:[a,b]\to\partial S$ 为 $\partial S$ 的参数化. 于是有
$$
\int_{\partial S}\omega=\int_{[a,b]}(\Gamma\circ\gamma)^*\omega=\int_{[a,b]}\gamma^*\Gamma^*\omega=\int_{\partial D}\Gamma^*\omega
$$

    此时对 $\Gamma^*\omega$ 应用 Green 公式即得
$$
\int_{\partial S}\omega=\int_{\partial D}\Gamma^*\omega=\int_D\dd\Gamma^*\omega=\int_S\omega
$$
\end{proof}

\begin{definition}
    称 $S$ 为简单曲面,若 $S$ 可以分割为有限个 $z$-型(或 $x$-型,或 $y$-型)曲面的并,且只在边界相交.
\end{definition}

从而我们即对简单曲面证明了 Stokes 公式.

\begin{property}
    设 $G\subset\RR^3$ 为开集,$P,Q,R\in C^{(1)}(G),S\subset G$ 紧,$\partial S$ 分段光滑,且 $S$ 为简单曲面,则
$$
\int_{\partial S}\omega=\int_S\dd\omega
$$

    其中 $\omega=P\dd x+Q\dd y+R\dd z$.
\end{property}

\mysubsection{一般形式的 Stokes 公式}

\mysubsubsection{区间上的 Stokes 公式}

\begin{lemma}
    设 $I=[0,1]^k$ 为 $k$ 维闭区间. 设 $G\subset\RR^k$ 为开集,$I\subset G$. 设 $\omega\in\Omega^{k-1}(G)$,则
$$
\int_{\partial I}\omega=\int_I\dd\omega
$$

    其中 $I$ 取标准定向 $(e_1,\cdots,e_k)$,而 $\partial I$ 取由 $I$ 的定向诱导的定向.
\end{lemma}
\begin{proof}
    设
$$
\omega(x)=\sum_{j=1}^k(-1)^{j-1}a_j(x)\dd x_1\wedge\cdots\wedge\widehat{\dd x_j}\wedge\cdots\wedge\dd x_k=\sum_{j=1}^k\omega_j
$$

    其中 $\omega_j(x)=(-1)^{j-1}a_j(x)\dd x_1\wedge\cdots\wedge\widehat{\dd x_j}\wedge\cdots\wedge\dd x_k$. 则
$$
\dd\omega=\left(\sum_{j=1}^k\pard{a_j}{x_j}\right)\dd x_1\wedge\cdots\wedge\dd x_k
$$

    从而我们只需证:$\displaystyle\int_{\partial I}\omega_j=\int_I\dd\omega_j,j=1,\cdots,k$.

    记 $\hat{I}=[0,1]^{k-1},I_0^l=\set{x\in I|x_l=0},I_1^l=\set{x\in I|x_l=1}$. 则由 Fubini 定理知
$$
\begin{aligned}
    \int_I\dd\omega_j=&\int_{\hat{I}}\left(\int_0^1\pard{a_j}{x_j}(x)\dd x_j\right)\dd x_1\cdots\dd x_{j-1}\dd x_{j+1}\cdots\dd x_k\\
    =&\int_{\hat{I}}a_j(x_1,\cdots,x_{j-1},1,x_{j+1},\cdots,x_k)\dd x_1\cdots\dd x_{j-1}\dd x_{j+1}\cdots\dd x_k\\
    &-\int_{\hat{I}}a_j(x_1,\cdots,x_{j-1},0,x_{j+1},\cdots,x_k)\dd x_1\cdots\dd x_{j-1}\dd x_{j+1}\cdots\dd x_k
\end{aligned}
$$

    接下来我们计算 $\displaystyle\int_{\partial I}\omega$. 已知 $I_0^l,I_1^l$ 均为 $\partial I$ 的一部分,且其定向为

    \img{0.6}{13.3.8.png}

    它们有参数化
$$
\begin{aligned}
    \varphi_0^l:\hat{I}\to I_0^l,(t_1,\cdots,t_{k-1})\mapsto(t_1,\cdots,t_{l-1},0,t_l,\cdots,t_{k-1})\\
    \varphi_1^l:\hat{I}\to I_1^l,(t_1,\cdots,t_{k-1})\mapsto(t_1,\cdots,t_{l-1},1,t_l,\cdots,t_{k-1})
\end{aligned}
$$

    特别的,对 $s=0,1$ 有 $\varphi^l_{s}$ 在 $I_s^l$ 上诱导的定向均为 $(e_1,\cdots,\widehat{e_l},\cdots,e_k)$. 从而
$$
\begin{aligned}
    \int_{I_1^j}\omega_j&=(-1)^{j-1}\int_{\hat{I}}(\varphi_1^j)^*\omega_j\\
    &=(-1)^{j-1}\int_{\hat{I}}(-1)^{j-1}a(t_1,\cdots,t_{j-1},1,t_j,\cdots,t_{k-1})\dd t_1\cdots\dd t_{k-1}\\
    &=\int_{\hat{I}}a(x_1,\cdots,x_{j-1},1,x_{j+1},\cdots,x_k)\dd x_1\cdots\dd x_{j-1}\dd x_{j+1}\cdots\dd x_{k}
\end{aligned}
$$

    同理
$$
\int_{I_0^j}\omega_j=-\int_{\hat{I}}a(x_1,\cdots,x_{j-1},0,x_{j+1},\cdots,x_k)\dd x_1\cdots\dd x_{j-1}\dd x_{j+1}\cdots\dd x_{k}
$$

    而当 $l\ne j$ 时
$$
\int_{I_s^l}\omega_j=0
$$

    综上有
$$
\int_{\partial I}\omega_j=\int_I\dd\omega_j
$$
\end{proof}

\mysubsubsection{一般的 Stokes 公式}

\begin{theorem}
    设 $G\subset\RR^n$ 为开集,$\omega\in\Omega^{k-1}(G)$.

    设 $S\subset G$ 是一个 $k$ 维分片光滑定向曲面,$\overline{S}$ 紧,则
$$
\int_{\partial S}\omega=\int_S\dd\omega
$$
\end{theorem}

我们仅对一类特殊的曲面来证明该定理.

\begin{definition}
    设 $S$ 为 $k$ 维光滑曲面. 称 $S$ 为正则曲面,若其存在参数化 $\varphi:I\to S$,其中 $I=[0,1]^k$.
\end{definition}

\begin{lemma}
    设 $G\subset\RR^n$ 为开集,$\omega\in\Omega^{k-1}(G)$. 设 $S\subset G$ 为正则曲面,$\varphi:I\to S$ 为参数化. 则
$$
\int_{\partial S}\omega=\int_S\dd\omega
$$
\end{lemma}
\begin{proof}
    由 $\varphi:I\to S$ 为参数化知 $\varphi|_{\partial I}:\partial I\to\partial S$ 为 $\partial S$ 的参数化(更严格的说是有限个参数化的并). 从而

$$
\int_{\partial S}\omega=\int_{\partial I}\varphi^*\omega=\int_I\dd\varphi^*\omega=\int_I\varphi^*\dd\omega=\int_S\dd\omega
$$
\end{proof}

\begin{definition}
    称 $S\subset\RR^m$ 为简单 $k$ 维曲面,若存在 $k$ 维正则曲面 $S_1,\cdots,S_l$ 使得
$$
S=\bigcup_{i}S_i,\qquad S_i\cap S_j\subset\partial S_i\cup\partial S_j,\forall i\ne j
$$
\end{definition}

从而我们即对简单曲面证明了 Stokes 公式.

\begin{example}[ $n$ 维 Brouwer 不动点定理]
    设 $f:\overline{B(0,1)}\to\overline{B(0,1)}\subset\RR^n$ 满足 $f\in C^{(1)}$. 则 $f$ 有不动点.
\end{example}

\begin{hint}
    Newton-Leibniz 公式也可以解释为 Stokes 公式在 $k=1$ 时的情形.
$$
\begin{aligned}
    f(b)-f(a)&=\int_a^bf'(x)\dd x\\
    \int_{\partial[a,b]}f&=\int_{[a,b]}\dd f
\end{aligned}
$$

    其中 $f\in\Omega^0([a,b])$.
\end{hint}
