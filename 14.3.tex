\mysubsection{具有标量势的场}

\begin{definition}
    设 $D\subset\RR^n$ 为开集,$A:D\to\RR^n$ 为 $C^{(1)}$ 光滑.
    
    称 $U:D\to\RR$ 是 $A$ 的势,若 $\nabla U=A$.

    此时称 $A$ 是一个势场.
\end{definition}

\begin{property}
    若 $D$ 为区域,$A:D\to\RR^n$ 为势场,$U$ 与 $U'$ 均为 $A$ 的势,则 $\exists C\in\RR,U'=U+C$.
\end{property}

\begin{example}[ 万有引力场]
    定义 $F:\RR^3\setminus\set{0}\to\RR^3$ 为
$$
F(r)=-\frac{GM}{\abs{r}^2}\frac{r}{\abs{r}}=-\frac{GMr}{\abs{r}^3}
$$

    $F$ 的一个势为
$$
U(r)=\frac{GM}{\abs{r}}
$$
\end{example}

\begin{example}[ 点电荷的静电场]
    定义 $E:\RR^3\setminus\set{0}\to\RR^3$ 为
$$
F(r)=\frac{q}{4\pi\eps_0}\frac{r}{\abs{r}^3}
$$

    $F$ 的一个势为
$$
U(r)=-\frac{q}{4\pi\eps_0\abs{r}}
$$
\end{example}

\begin{property}
    设 $D\subset\RR^n$ 为开集,$A:D\to\RR^n$ 为势场,$U:D\to\RR$ 为 $A$ 的势.
    
    任给 $\gamma\subset D$ 为光滑曲线,有
$$
\int_\gamma A\cdot\dd\mathbf{s}=U\biggr |_{\partial\gamma}=U(q)-U(p)
$$

    其中 $p,q$ 分别为 $\gamma$ 的起点与终点.
\end{property}

\mysubsection{存在标量势的必要条件}

\begin{property}
    设 $D\subset\RR^n$ 为开集. 则若 $A:D\to\RR^n$ 为势场,必有
$$
\dd\omega_A^1=0
$$

    特别的,若 $n=3$,则 $A$ 为势场 $\implies\nabla\times A$=0.
\end{property}
\begin{proof}
    设 $U:D\to\RR$ 为 $A$ 的势. 即 $A=\nabla U$. 从而
$$
\dd\omega_A^1=\dd\dd\omega_U^0=\dd^2\omega_U^0=0
$$

    当 $n=3$ 时,$\dd\omega_A^1=\omega_{\nabla\times A}^2=0\implies\nabla\times A=0$.
\end{proof}

\begin{hint}
    一般来讲,$\dd\omega_A^1=0$ 仅为必要条件,不一定充分.
\end{hint}

\begin{example}
    定义 $A:\RR^2\setminus\set{0}\to\RR^2$ 为
$$
A(x,y)=\left(\frac{-y}{x^2+y^2},\frac{x}{x^2+y^2}\right)
$$
    
    则直接计算可得 $\dd\omega_A^1=0$.

    但我们断言:$A$ 不是势场. 反证,设 $A$ 为势场.

    取 $\gamma$ 为单位圆周,由 $A$ 为势场可得
$$
\int_\gamma\omega_A^1=0
$$

    但另一方面,我们已经计算过
$$
\int_\gamma\omega_A^1=2\pi
$$

    矛盾. 故 $A$ 不是势场.
\end{example}

\mysubsection{存在标量势的充要条件}

以下我们给出一个重要的判定准则来判定向量场何时具有标量势. 为此我们首先给出一个定义.

\begin{definition}
    设 $D\subset\RR^n$ 为开集,$\omega\in\Omega^1(D)$.

    设 $\gamma:[a,b]\to D$ 是一条分段光滑路径,我们定义 $\omega$ 沿 $\gamma$ 的积分为
$$
\int_\gamma\omega\triangleq\int_a^b\gamma^*\omega
$$
\end{definition}

\begin{hint}
    \begin{enumerate}
        \item 当 $\gamma([a,b])$ 为光滑曲线且 $\gamma$ 为其参数化时,上面的定义即为通常的曲线积分的定义.
        
        \item 但以上的定义更广泛,例如我们可以取一些光滑路径,使得其像并非光滑曲线,但此时我们依然可以谈论 $\omega$ 沿 $\gamma$ 的积分.
    \end{enumerate}
\end{hint}

\img{0.8}{14.3.1.png}

现在我们可以来谈论这个充要条件了.

\begin{property}
    设 $D\subset\RR^n$ 为开集,$A:D\to\RR^n$ 为 $C^{(1)}$ 光滑.

    则 $A$ 为势场 $\iff$ 对任意 $D$ 中的分段光滑闭路径 $\gamma$ 有
$$
\oint_\gamma A\cdot\dd\mathbf{s}\triangleq\int_\gamma\omega_A^1=0
$$
\end{property}
\begin{proof}
    $\implies$:设 $U\to\RR$ 满足 $A=\nabla U$.
    
    设闭路径 $\gamma:[a,b]\to D$. 则 $\gamma(a)=\gamma(b)$. 从而
$$
\begin{aligned}
    \int_\gamma\omega_A^1&=\int_\gamma\omega_{\nabla U}^1=\int_\gamma\sum_{i=1}^n\pard{U}{x_i}(x)\dd x_i\\
    &=\int_a^b(U(\gamma(t)))'\dd t=U(\gamma(t))\biggr |_a^b=0
\end{aligned}
$$

    $\impliedby$:不妨设 $D$ 为区域(即联通开集).

    取定 $x_0\in D$. 我们首先断言:任取 $x\in D$,设 $\gamma_1$ 与 $\gamma_2$ 是连接 $x_0$ 与 $x$ 的两个分段光滑路径,则
$$
\int_{\gamma_1}\omega_A^1=\int_{\gamma_2}\omega_A^1
$$

    证明:设 $\gamma_1:[0,1]\to D,\gamma_2:[0,1]\to D$ 为两条分段光滑路径,$\gamma_1(0)=\gamma_2(0)=x_0$ 且 $\gamma_1(1)=\gamma_2(1)=x$.

    定义 $\gamma:[0,2]\to D$ 为
$$
\gamma(t)=\begin{cases}
    \gamma_1(t) & t\in[0,1]\\
    \gamma_2(2-t) & t\in(1,2]
\end{cases}
$$

    \img{0.7}{14.3.2.png}

    则 $\gamma$ 为分段光滑闭路径. 从而
$$
\int_\gamma\omega_A^1=\int_{\gamma_1}\omega_A^1-\int_{\gamma_2}\omega_A^2=0
$$

    \qed

    由此,我们可以定义 $U:D\to\RR$ 为
$$
U(x)\triangleq\int_\gamma\omega_A^1
$$

    其中 $\gamma$ 为连接 $x_0$ 与 $x$ 的任意一条分段光滑路径.

    下证:$U$ 即为 $A$ 的势.

    取定 $x\in D$. 设 $r>0$ 使得 $B(x,r)\subset D$.

    对任意 $0<\eps<r,1\le i\le n$ 定义 $\gamma_\eps:[0,\eps]\to D$ 为 $\gamma_\eps(t)\triangleq x+te_i$. 则有
$$
\begin{aligned}
    U(x+\eps e_i)-U(x)&=\int_{\gamma_\eps}\omega_A^1\\
    &=\int_{\gamma_\eps}\sum_{j=1}^nA_j(x)\dd x_j\\
    &=\int_0^\eps A_i(x+te_i)\dd t\\
    &=\eps A_i(x+\eta e_i)
\end{aligned}
$$

    其中 $\eta\in[0,\eps]$.

    \img{0.3}{14.3.3.png}

    由 $A$ 为 $C^{(1)}$ 光滑可得
$$
\lim_{\eps\to 0}\frac{U(x+\eps e_i)-U(x)}{\eps}=A_i(x)
$$

    故 $\nabla U=A$. 故 $U$ 即为所求.
\end{proof}

\begin{hint}
    从证明的过程来看,我们完全可以考虑一类更小更简单的闭路径,即那些由有限条与坐标轴平行的线段构成的折线构成的路径类. 这个观察可以简化下一个性质的证明.
\end{hint}

现在,我们再次来看上一节的例子,即
$$
A(x,y)=\left(\frac{-y}{x^2+y^2},\frac{x}{x^2+y^2}\right)
$$

由于
$$
\oint_{\partial B(0,1)}\omega_A^1=2\pi
$$

我们再次断言 $A$ 在 $\RR^2\setminus\set{0}$ 没有势.

但另一方面,考虑函数 $U(x,y)\triangleq\arctan\dfrac{y}{x}$.

直接计算可知:$\nabla U(x,y)=A(x,y)$. 这说明 $U$ 是 $A$ 的势. 这当中存在矛盾吗?

当然,这里是没有矛盾的.

原因在于:以上的函数 $U(x,y)$ 的定义域为右半平面 $H^2_+=\set{(x,y)|x>0}$,而不是 $\RR^2\setminus\set{0}$.

\img{0.8}{14.3.4.png}

我们也许会说,这个定义域是可以延拓的. 例如,定义
$$
U(x,y)\triangleq\begin{cases}
    \arctan\dfrac{y}{x} & x>0\\
    \dfrac{\pi}{2}-\arctan\dfrac{x}{y} & y>0\\
    \arctan\dfrac{y}{x}+\pi & x<0
\end{cases}
$$

则由三角函数的知识可知,以上的函数是良定义的.

从而我们在除了负 $y$ 轴之外的地方处处定义好了 $U(x,y)$.

即定义域可以变为 $\widetilde{D}\triangleq\RR^2\setminus\set{(0,y)|y\le 0}$. 此时在 $\widetilde{D}$ 上仍然有 $\nabla U=A$.

但当我们希望再次扩充定义域,将负 $y$ 轴也加进来时,就会遇到麻烦. 我们试图按下式来扩充定义 $U$.
$$
U(x,y)\triangleq\dfrac{3\pi}{2}-\arctan\frac{x}{y}\qquad y<0
$$

此时 $U(x,y)$ 在第三象限的确可以吻合得很好,但是在第四象限,上式所定义的 $U(x,y)$ 比原始的 $U(x,y)$ 大了 $2\pi$.

因此,不可能将 $U(x,y)$ 连续地延拓到 $\RR^2\setminus\set{0}$ 上. 事实上,这是一个本质的困难. 因为 $\widetilde{D}$ 单连通,但 $\RR^2\setminus\set{0}$ 不再单连通,这导致不可能存在这样的延拓.

另一方面,由我们的论证知:若定义域设为 $\widetilde{D}$,则 $A$ 在其上确有势 $U$. 事实上我们很快将证明:若定义域是单连通的,则其上满足 $\dd\omega_A^1=0$ 的场 $A$ 必有标量势. 我们先证明如下的特例:

\begin{property}
    设 $D=B(x,r)\subset\RR^n$. 设 $A:D\to\RR^n$ 为 $C^{(1)}$ 光滑且 $\dd\omega_A^1=0$.

    则 $A$ 为 $D$ 上的势场.
\end{property}
\begin{proof}
    设 $I\subset D$ 是一个边与坐标轴平行的闭长方形. 由 Stokes 公式有
$$
\int_{\partial I}\omega_A^1=\int_I\dd\omega_A^1=0
$$

    为了证明该性质,我们仅需对每个封闭折线路径(其边与坐标轴平行)$\gamma:[a,b]\to D$ 验证
$$
\int_\gamma\omega_A^1=0
$$

    若 $n=2$,则可以将 $\gamma$ 分解成有限个简单的封闭折线的并. 在每个路径上作用 Green 公式即得结论.

    \img{0.3}{14.3.5.png}

    若 $n\ge 3$,则我们可以向封闭路径添加一些边,使得路径的边均为某个长方形的边,且添加的边必为两个长方形的公共边界. 由此再利用断言即得结论.
\end{proof}

\begin{hint}
    以上的结论可以解释为:若 $\dd\omega_A^1=0$,则 $A$ 在每个点的局部均为势场.
\end{hint}

\mysubsection{单连通区域与势}

本节我们将要证明:若区域单连通,且场满足存在势的必要条件,则其势必存在. 即该必要条件也充分.

为此,我们首先定义单连通区域. 直观的说,单连通区域是“没有洞”的区域. 或者说,其中的一条闭路径可以在不离开区域的条件下“连续地缩成一个点”.

\img{0.8}{14.3.6.png}

例如,$D_1$ 单连通,而 $D_2$ 不单连通,其为 $2$ 联通.

下面我们来讨论其严格的定义. 我们首先将连续形变的概念严格化.

\begin{definition}
    设 $D\subset\RR^n$ 为区域. 设 $\gamma_i:[0,1]\to D,i=0,1$ 为 $D$ 中的两条闭路径.

    称 $\Gamma:[0,1]^2\to D$ 是从 $\gamma_0$ 到 $\gamma_1$ 的一个同伦映射,若

    \begin{enumerate}
        \item $\Gamma$ 连续.

        \item $\Gamma(t,0)=\gamma_0(t),\Gamma(t,1)=\gamma_1(t),\forall t\in[0,1]$
        
        \item $\Gamma(0,\tau)=\Gamma(1,\tau),\forall\tau\in[0,1]$
    \end{enumerate}
\end{definition}

\img{1}{14.3.7.png}

我们来解释一下其几何意义:

\begin{enumerate}
    \item $\Gamma$ 将正方形的的\textcolor{Cyan}{\textbf{底边}}映成了 $\gamma_0$ 的像.
    
    \item $\Gamma$ 将正方形的的\textcolor{red}{\textbf{顶边}}映成了 $\gamma_1$ 的像.
    
    \item $\Gamma$ 将正方形\textcolor{Purple}{\textbf{左右两条边}}映成了 $D$ 中一条从 $A$ 到 $B$ 的路径.
    
    \item $\Gamma$ 将正方形任何一条\textcolor{LimeGreen}{\textbf{水平高度为 $\tau$ 的边}}映成了一条闭路径.
    
    \item 随着 $\tau$ 从 $0$ 到 $1$ 移动,我们得到了从 $\gamma_0$ 到 $\gamma_1$ 的一个连续形变.
\end{enumerate}

\begin{definition}
    设 $\gamma_0,\gamma_1$ 是 $D$ 中的两条闭路径. 称 $\gamma_0$ 与 $\gamma_1$ 在 $D$ 中同伦,若存在如上定义的 $\Gamma$.
\end{definition}

我们首先有如下重要观察:

\begin{property}
    设 $D\subset\RR^n$ 为区域. 设 $\omega\in\Omega^1(D)$ 且 $\dd\omega=0$.

    设 $\gamma_0,\gamma_1$ 为 $D$ 中的两条同伦的分段光滑闭路径,则
$$
\int_{\gamma_0}\omega=\int_{\gamma_1}\omega
$$
\end{property}
\begin{proof}
    在习题中我们会看到:为证明此性质,只需对光滑闭路径以及它们之间存在光滑同伦的情形证明即可.
    
    以下我们将作这一假设,即:$\gamma_0,\gamma_1$ 为光滑路径,且存在光滑同伦 $\Gamma:[0,1]^2\to D$ 满足 $\Gamma(t,0)=\gamma_0(t),\Gamma(t,1)=\gamma_1(t),\forall t\in[0,1]$.

    由 Green 公式,我们有
$$
\int_{\partial[0,1]^2}\Gamma^*\omega=\int_{[0,1]^2}\dd\Gamma^*\omega=\int_{[0,1]^2}\Gamma^*\dd\omega=0
$$

    \img{0.3}{14.3.8.png}

    如图记 $\partial[0,1]^2=\eta_1\cup\eta_2\cup\eta_3\cup\eta_4$. 则由定义可知
$$
\begin{aligned}
    &\int_{\eta_1}\Gamma^*\omega=\int_{\gamma_0}\omega\\
    &\int_{\eta_3}\Gamma^*\omega=-\int_{\gamma_1}\omega\\
    &\int_{\eta_2}\Gamma^*\omega=-\int_{\eta_4}\Gamma^*\omega
\end{aligned}
$$

    从而即有
$$
\int_{\gamma_0}\omega=\int_{\gamma_1}\omega
$$
\end{proof}

\begin{definition}
    设 $D\subset\RR^n$ 为区域. 称 $D$ 为单连通区域,若 $D$ 中的每条闭路径与一个常值路径同伦.
\end{definition}

\begin{theorem}
    设 $D\subset\RR^n$ 为单连通区域. 设 $A:D\to\RR^n$ 为 $C^{(1)}$ 光滑且 $\dd\omega_A^1=0$.
    
    则 $A$ 为势场.
\end{theorem}
\begin{proof}
    我们仅需验证 $A$ 为势场的判定准则. 任取 $\gamma$ 为 $D$ 中的分段光滑闭路径.

    由 $D$ 单连通知,存在常值路径 $\eta:[0,1]\to D,\eta(t)\equiv x_0$ 使得 $\gamma$ 与 $\eta$ 在 $D$ 中同伦. 从而由上面的性质知
$$
\int_\gamma\omega_A^1=\int_\eta\omega_A^1=\int_0^1\eta^*\omega_A^1=0
$$
\end{proof}

\begin{hint}
    \begin{enumerate}
        \item 易证 $B(x,r)$ 必为单连通区域,从而之前的性质(势局部存在)是这个定理的特例.
        
        \item 另一方面,之前性质的证明很几何化,而且是构造性的,且不需要光滑性的论证.
    \end{enumerate}
\end{hint}

\begin{example}
    \begin{itemize}
        \item $\RR^2\setminus\set{(0,y)|y\le 0}$ 单连通.
        
        \item $\RR^2\setminus\set{0}$ 不单连通.
        
        \item $\RR^3\setminus\set{0}$ 单连通.
    \end{itemize}
\end{example}

\mysubsection{向量势,闭形式与恰当形式}

之前我们讨论了向量场的标量势. 在物理中也会有关于向量势的问题:

\begin{definition}
    设 $D\subset\RR^3$ 为区域,向量场 $B:D\to\RR^3$ 光滑.

    若存在向量场 $A:D\to\RR^3$ 光滑使得 $B=\nabla\times A$,则称 $A$ 是 $B$ 的一个向量势.
\end{definition}

与标量势的讨论类似,我们有

\begin{property}
    设 $D\subset\RR^3$ 为区域,向量场 $B:D\to\RR^3$ 光滑.

    若 $B$ 存在向量势,则必有 $\nabla\cdot B=0$.
\end{property}
\begin{proof}
    设 $A$ 为 $B$ 的向量势,即 $B=\nabla\times A$. 则
$$
\begin{aligned}
    &\omega_B^2=\omega_{\nabla\times A}^2=\dd\omega_A^1\\
    \implies&\omega_{\nabla\cdot B}^3=\dd\omega_B^2=\dd^2\omega_A^1=0\\
    \implies&\nabla\cdot B=0
\end{aligned}
$$
\end{proof}

将以上的讨论推广到一般情形,我们给出如下的定义:

\begin{definition}
    设 $D\subset\RR^n$ 为区域. 设 $\omega\in\Omega^p(D)$.

    称 $\omega$ 是一个闭形式,若 $\dd\omega=0$.

    称 $\omega$ 是一个恰当形式,若 $\exists\widetilde{\omega}\in\Omega^{p-1}(D),\dd\widetilde{\omega}=\omega$.
\end{definition}

在这个语言下:

\begin{itemize}
    \item $A$ 有标量势 $\iff\omega_A^1$ 为恰当 $1$-形式.
    
    \item $\nabla\times A=0\iff\omega_A^1$ 为闭 $1$-形式.
    
    \item $B$ 有向量势 $\iff\omega_B^2$ 为恰当 $2$-形式.
    
    \item $\nabla\cdot B=0\iff\omega_B^2$ 为闭 $2$-形式.
\end{itemize}

一个重要的问题是:当 $\omega$ 为 $p$-闭形式时,在什么条件下其可以成为一个恰当 $p$-形式.

这个问题有如下的局部化答案:

\begin{theorem}[Poincaré]
    设 $D=B(x,r)\subset\RR^n$.

    若 $\omega\in\Omega^p(D)$ 为闭形式,则 $\omega$ 也是恰当形式.
\end{theorem}

即局部的,每个闭形式也是恰当形式.

\begin{hint}
    事实上,Poincaré 证明了只要 $D$ 是“可缩的”,即 $D$ 可以在不离开 $D$ 的情形下连续地“缩到一个点”,则有 $D$ 上的闭形式均为恰当形式.

    例如:$D=\RR^n$ 或 $D$ 为星形区域均为可缩区域.

    \img{0.4}{14.3.9.png}
\end{hint}

以下我们再举几个例子,以解释形式的闭性、恰当性及其定义域的拓扑性的一些联系.

\begin{example}
    令 $D=\RR^2\setminus\set{0}$. 设 $\omega$ 为 $D$ 上的闭 $1$-形式.

    设 $\gamma_1$ 是以 $0$ 为圆心,$1$ 为半径的圆周. 即
$$
\gamma_1:[0,2\pi]\to D,\theta\mapsto(\cos\theta,\sin\theta)
$$

    \img{0.5}{14.3.10.png}

    令 $\displaystyle T\triangleq\int_{\gamma_1}\omega$.

    我们称 $T$ 是 $\omega$ 沿 $\gamma_1$ 的周期. 直观来讲,设 $\gamma\subset D$ 是任何一个闭路径. 则若 $\gamma$ 逆时针绕 $0$ 转了 $k$ 圈,则有
$$
\int_\gamma\omega=kT
$$

    这是因为若 $\gamma$ 逆时针绕 $0$ 转了 $k$ 圈,则我们可以构造从 $k\gamma_1$ 到 $\gamma$ 的同伦. 从而二者积分相同. 其中
$$
k\gamma_1:[0,2k\pi]\to D,\theta\mapsto(\cos\theta,\sin\theta)
$$

    即绕以 $0$ 为圆心,$1$ 为半径的圆转 $k$ 圈.

    由此我们可以得到 $\omega$ 为恰当形式的充要条件. 即
$$
\omega\text{ 为恰当形式}\iff T=0
$$
\end{example}

\begin{example}
    令 $D=\RR^2\setminus\set{(1,0),(-1,0)},\omega$ 为 $D$ 上的闭 $1$-形式.

    \img{0.8}{14.3.11.png}

    设
$$
\begin{aligned}
    &\gamma_+:[0,2\pi]\to D,\theta\mapsto(1,0)+r(\cos\theta,\sin\theta),0<r<1\\
    &\gamma_-:[0,2\pi]\to D,\theta\mapsto(-1,0)+r(\cos\theta,\sin\theta),0<r<1
\end{aligned}
$$

    令 $\displaystyle T_+\triangleq\int_{\gamma_+}\omega,T_-=\int_{\gamma_-}\omega$

    与上一例类似,任取 $D$ 上的一个闭路径 $\gamma$. 设 $\gamma$ 绕 $(1,0)$ 逆时针转了 $n_+$ 圈,绕 $(-1,0)$ 逆时针旋转了 $n_-$ 圈,则有
$$
\int_\gamma\omega=n_+T_++n_-T_-
$$

    从而我们得到 $\omega$ 为恰当形式的充要条件
$$
\omega\text{ 为恰当形式}\iff T_+=T_-=0
$$
\end{example}

\begin{hint}
    以上的想法可以推广到任何区域 $D$,也可以推广到对 $p$-恰当形式的刻画.

    我们已知区域的单连通性对保证闭 $1$-形式成为恰当形式至关重要. 以下我们简要讨论区域什么样的性质与闭 $p$-形式成为恰当形式相关.
\end{hint}

\begin{definition}
    设 $D\subset\RR^n$ 为区域. 设 $S^k$ 为 $k$ 维球面.

    若每个连续映射 $\pi:S^k\to D$ 均可在 $D$ 中同伦于常值映射,则称 $D$ 有平凡的 $k$ 维同伦群. 记为 $\pi_k(D)=\set{0}$.
\end{definition}

则当 $k=1$ 时,$\pi_1(D)=\set{0}\iff D$ 单连通.

\begin{example}
    $\pi_1(\RR^2\setminus\set{0})\ne\set{0}$.

    $\pi_1(\RR^3\setminus\set{0})=\set{0},\pi_2(\RR^3\setminus\set{0})\ne\set{0}$.

    设 $D$ 为实心轮胎,$D\subset\RR^3$.

    则有 $\pi_1(D)\ne\set{0},\pi_2(D)=\set{0}$
\end{example}

\begin{property}
    设 $\pi_p(D)=\set{0},\omega$ 为 $D$ 上的 $p$-形式.

    若 $\omega$ 为闭形式,则 $\omega$ 也是恰当形式.
\end{property}