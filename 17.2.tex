本节我们讨论含参广义积分的各种性质.

含参广义积分的形式如下:
$$
F(y)=\int_a^\omega f(x,y)\dd x
$$

其中 $y\in Y$ 为参数.

在本节,我们应做一个如下的类比:
$$
S(y)=\sum_{n=1}^\infty f(n,y)=\sum_{n=1}^\infty f_n(y)
$$

即将关于 $x$ 的积分和关于 $n$ 的求和相类比.

这样一来,本节的所有结论在以前都会有一个类似的结论!

\mysubsection{含参广义积分关于参数的一致收敛}

\mysubsubsection{基本定义与例子}

设 $Y$ 是一个集合,$f:[a,\omega)\times Y\to\RR$ 满足对 $\forall y\in Y$,广义积分
$$
F(y)\triangleq\int_a^\omega f(x,y)\dd x
$$

收敛. 这里要么 $\omega=+\infty$,要么 $\omega$ 有限且 $f_y$ 在 $\omega$ 的某个邻域上无界.

\begin{definition}
    称广义积分 $\displaystyle\int_a^\omega f(x,y)\dd x$ 在 $E\subset Y$ 上一致收敛,若
$$
\forall\eps>0,\exists B>a,\forall b>B,\forall y\in E,\abs{\int_b^\omega f(x,y)\dd x}<\eps
$$
\end{definition}

\begin{hint}
    \begin{enumerate}
        \item 应用最开始的类比:在级数情形下,上面的条件相当于
$$
\forall\eps>0,\exists N\in\mathbb{N},\forall m\ge N,\forall y\in E,\abs{\sum_{n=m}^\infty f_n(y)}<\eps
$$

        从而其等价于 $S_n(y)\triangleq\displaystyle\sum_{j=1}^nf_j(y)\rightrightarrows S(y)\quad(y\in E)$.

        \item 受上面的类比启发,对 $\forall b\in[a,\omega)$ 定义
$$
F_b(y)\triangleq\int_a^bf(x,y)\dd x
$$

        则 $\displaystyle\int_a^\omega f(x,y)\dd x$ 在 $E$ 上一致收敛当且仅当
$$
F_b(y)\xrightrightarrows[b\to\omega-]{}F(y)
$$

        当且仅当
$$
\forall\eps>0,\exists B\in[a,\omega),\forall b\ge B,\forall y\in E,\abs{F_b(y)-F(y)}<\eps
$$

        接下来,我们会发现以上的两个等价定义在实际运用中非常有用.
    \end{enumerate}
\end{hint}

\begin{example}
    $F(y)=\displaystyle\int_1^\infty\frac{\dd x}{x^2+y^2}\dd x$ 在 $\RR$ 上一致收敛.
\end{example}

\begin{example}
    $F(y)=\displaystyle\int_0^\infty e^{-xy}\dd x$ 在 $y>0$ 时收敛.

    其在 $y\ge a>0$ 上一致收敛,但在 $y>0$ 上非一致收敛.
\end{example}

\begin{example}
    设 $\alpha,\beta>0$.
$$
\Phi(x)\triangleq\int_0^\infty x^\alpha y^{\alpha+\beta+1}e^{-(1+x)y}\dd y
$$

    在 $x\ge 0$ 上一致收敛.
$$
F(y)\triangleq\int_0^\infty x^\alpha y^{\alpha+\beta+1}e^{-(1+x)y}\dd x
$$

    在 $y\ge 0$ 上一致收敛.
\end{example}

\mysubsubsection{一致收敛的 Cauchy 准则}

\begin{property}
    定义 $f$ 同前. 则 $F(y)$ 在 $E\subset Y$ 上一致收敛当且仅当
$$
\forall\eps>0,\exists B\in [a,\omega),\forall b,b'\in[B,\omega),\forall y\in E,\abs{\int_b^{b'}f(x,y)\dd x}<\eps
$$
\end{property}
\begin{proof}
    由前面的注记以及 Cauchy 准则即得.
\end{proof}

\begin{inference}
    设 $f:[a,\omega)\times[c,d]\to\RR$ 连续,对 $\forall y\in(c,d)$ 有 $F(y)=\displaystyle\int_a^\omega f(x,y)\dd x$ 收敛.

    若 $y=c$ 或 $d$ 有 $\displaystyle\int_a^\omega f(x,y)\dd x$ 发散,则 $F(y)$ 在 $(c,d)$ 上非一致收敛.
\end{inference}
\begin{proof}
    我们证明其逆否命题:设 $F(y)$ 在 $(c,d)$ 上一致收敛. 则
$$
\forall\eps>0,\exists B\in[a,\omega),\forall b,b'\in[B,\omega),\forall y\in(c,d),\abs{\int_b^{b'}f(x,y)\dd x}<\eps
$$

    由 $f$ 在 $[b,b']\times [c,d]$ 连续知
$$
\abs{\int_b^{b'}f(x,c)\dd x}=\abs{\lim_{y\to c}\int_b^{b'}f(x,y)\dd x}\le\eps
$$

    由 Cauchy 准则知 $\displaystyle\int_a^\omega f(x,c)\dd x$ 收敛.

    同理 $\displaystyle\int_a^\omega f(x,d)\dd x$ 收敛.
\end{proof}

\begin{example}
    $F(t)=\displaystyle\int_0^\infty e^{-tx^2}\dd x$ 在 $t>0$ 时收敛.
    
    但其在 $t=0$ 时发散,从而其在 $(0,\infty)$ 上不一致收敛.
\end{example}

\mysubsubsection{一致收敛的充分条件}

\begin{property}[Weierstrass 判别法]
    设 $f,g:[a,\omega)\times Y\to\RR$ 满足任意固定 $y\in Y,f,g$ 在任意 $[a,b]\subset[a,\omega)$ 上可积. 若

    \begin{enumerate}
        \item $\abs{f(x,y)}\le g(x,y),\forall (x,y)\in [a,\omega)\times Y$
        
        \item $\displaystyle\int_a^\omega g(x,y)\dd x$ 在 $Y$ 上一致收敛.
    \end{enumerate}

    则 $\displaystyle\int_a^\omega f(x,y)\dd x$ 在 $Y$ 上一致收敛且绝对收敛.
\end{property}
\begin{proof}
    由 Cauchy 准则以及
$$
\abs{\int_b^{b'}f(x,y)\dd x}\le\int_b^{b'}\abs{f(x,y)}\dd x\le\int_b^{b'}g(x,y)\dd x
$$

    即证.
\end{proof}


特别的,若以上的 $g$ 不依赖于 $y$,即 $g(x,y)\equiv g(x)$ 且
$$
\int_a^\omega g(x)\dd x
$$

收敛,则 $\displaystyle\int_a^\omega f(x,y)\dd x$ 在 $Y$ 上一致收敛且绝对收敛.

\begin{example}
    $\displaystyle\int_0^\infty\frac{\cos\alpha x}{1+x^2}\dd x$ 对 $\alpha\in\RR$ 一致收敛.
\end{example}

\begin{example}
    $\displaystyle\int_0^\infty\sin xe^{-tx^2}\dd x$ 在 $t\ge t_0>0$ 上一致收敛,但在 $t>0$ 上不一致收敛.
\end{example}

\begin{property}[Abel-Dirichlet 判别法]
    设 $f,g:[a,\omega)\times Y\to\RR$ 满足任意固定 $y\in Y,f,g$ 在任意 $[a,b]\subset[a,\omega)$ 上可积.

    若以下两组条件之一成立:

    \begin{itemize}
        \item \begin{enumerate}
            \item 存在 $M>0$ 使得
$$
\forall b\in[a,\omega),\forall y\in Y,\abs{\int_a^bf(x,y)\dd x}\le M
$$

            \item $g$ 对 $\forall y\in Y$ 关于 $x$ 单调,且
$$
g(x,y)\xrightrightarrows[x\to\omega]{}0
$$
        \end{enumerate}

        \item \begin{enumerate}
            \item $\displaystyle\int_a^\omega f(x,y)\dd x$ 在 $Y$ 上一致收敛.
            
            \item $g$ 对 $\forall y\in Y$ 关于 $x$ 单调且存在 $M>0$ 使得
$$
\abs{g(x,y)}\le M,\forall(x,y)\in[a,\omega)\times Y
$$
        \end{enumerate}
    \end{itemize}

    则 $\displaystyle\int_a^\omega f(x,y)g(x,y)\dd x$ 在 $Y$ 上一致收敛.
\end{property}
\begin{proof}
    对 $\forall a\le b<b'<\omega$,由第二积分中值定理有
$$
\exists\xi\in[b,b'],\int_b^{b'}f(x,y)g(x,y)\dd x=g(b,y)\int_b^\xi f(x,y)\dd x+g(b',y)\int_\xi^{b'}f(x,y)\dd x
$$

    结合条件以及 Cauchy 准则即证.
\end{proof}

\begin{example}
    $\displaystyle\int_1^\infty\frac{\sin x}{x^\alpha}\dd x$ 仅在 $\alpha>0$ 时收敛.

    其在 $\alpha=0$ 时发散. 从而其在 $\alpha>0$ 时不一致收敛.

    由 Abel-Dirichlet 判别法知其在 $\alpha\ge\alpha_0>0$ 上一致收敛.
\end{example}

\begin{example}
    $\displaystyle\int_0^\infty\frac{\sin x}{x}e^{-xy}\dd x$ 在 $y\ge 0$ 上一致收敛.
\end{example}

\begin{hint}
    \begin{enumerate}
        \item 可以验证,以上的结论对向量值函数也成立(即将到达域换成任何一个 Banach 空间). 当然,在使用 Abel-Dirichlet 判别法时,$g$ 必须为实值函数.
        
        \item 以上仅讨论了积分上界为奇点的情形,即 $b=\omega$.
        
        同理可以得到 $a=\omega$ 情形的所有性质. 进一步,若 $a,b$ 均为奇点,则任取 $c\in(\omega_1,\omega_2)$,可将积分写成
$$
\int_{\omega_1}^{\omega_2}f(x,y)\dd x=\int_{\omega_1}^cf(x,y)\dd x+\int_c^{\omega_2}f(x,y)\dd x
$$

        此时定义积分一致收敛为以上的两个积分均一致收敛.

        从而上面的讨论可以给出相应的所有性质. 此时也容易验证:一致收敛性不依赖于 $c$ 的选取.
    \end{enumerate}
\end{hint}

\mysubsection{广义积分与极限交换次序;广义积分关于参数的连续性}

\begin{property}
    设 $f:[a,\omega)\times Y\to\RR$ 满足对 $\forall y\in Y,\displaystyle\int_a^\omega f(x,y)\dd x$ 收敛.

    设 $\mathcal{B}$ 为 $B$ 的一个基. 若

    \begin{enumerate}
        \item 对 $\forall b\in[a,\omega)$ 有:在 $[a,b]$ 上
$$
f(x,y)\xrightrightarrows[\mathcal{B}]{}{\varphi(x)}
$$

        \item $\displaystyle\int_a^\omega f(x,y)\dd x$ 在 $Y$ 上一致收敛.
    \end{enumerate}

    则 $\varphi$ 在 $[a,\omega)$ 上广义可积且
$$
\int_a^\omega\varphi(x)\dd x=\int_a^\omega\lim_{\mathcal{B}}f(x,y)\dd x=\lim_{\mathcal{B}}\int_a^\omega f(x,y)\dd x
$$
\end{property}
\begin{proof}
    令 $F(y)\triangleq\displaystyle\int_a^\omega f(x,y)\dd x,F_b(y)\triangleq\int_a^bf(x,y)\dd x$.

    则由 $\displaystyle\int_a^\omega f(x,y)\dd x$ 在 $Y$ 上一致收敛知 $F_b(y)\xrightrightarrows[b\to\omega-]{} F(y)$.

    由 $f(x,y)\xrightrightarrows[\mathcal{B}]{}\varphi(x)$ 知 $F_b(y)\triangleq\displaystyle\int_a^bf(x,y)\dd x\xrightarrow[\mathcal{B}]{}\int_a^b\varphi(x)\dd x$.

    从而由定理 \ref{sl} 有交换图
    
    \begin{center}
        \begin{tikzcd}[column sep={3.5cm,between origins},row sep={3.5cm,between origins}]
            F_b(y) \dar["\mathcal{B}"] \rar["b\to\omega-",transform canvas={yshift=0.45ex}] \rar[transform canvas={yshift=-0.45ex}] & F(y) \dar["\mathcal{B}"]\\
            \displaystyle\int_a^b\varphi(x)\dd x \rar["b\to\omega-"] & \displaystyle\int_a^\omega\varphi(x)\dd x
        \end{tikzcd}
    \end{center}

    从而 $\displaystyle\lim_{b\to\omega}\int_a^b\varphi(x)\dd x$ 存在,即 $\varphi$ 在 $[a,\omega)$ 上广义可积.

    且 $\displaystyle\lim_{\mathcal{B}}\int_a^\omega f(x,y)\dd x$ 存在,且二者相等. 从而
$$
\int_a^\omega\varphi(x)\dd x=\lim_{\mathcal{B}}\int_a^\omega f(x,y)\dd x
$$
\end{proof}

\begin{hint}
    需要指出的是:仅有第一个条件不能保证结论成立(在正常含参积分情形,第一个条件已经足够). 这可以从以下的反例中推出:
\end{hint}

\begin{example}
    定义 $f:(0,\infty)\times(0,\infty)\to\RR$ 为
$$
f(x,y)=\begin{cases}
    \dfrac{1}{y} & 0\le x\le y\\
    0 & x>y
\end{cases}
$$

    \begin{center}
        \begin{tikzpicture}[scale=2]
            \draw[->] (0,0)--(2,0) node [right] {$x$};
            \draw[->] (0,0)--(0,2) node [right] {$y$};
            \draw[-] (0,0)--(1.8,1.8) node [right] {$y=x$};
            \draw[-] (1,1)--(0,1) node [left] {$y$};
            \draw[-,red] (1,1)--(2,1);
            \node at (0.5,1) [above] {$\frac{1}{y}$};
            \node[red] at (1.5,1) [above] {$0$};
            \node at (0,0) [below left] {$O$};
        \end{tikzpicture}
    \end{center}

    则易见:$\displaystyle\int_0^\infty f(x,y)\dd x=\int_0^y\frac{1}{y}\dd x=1$ 收敛. 且 $f(x,y)\xrightrightarrows[y\to+\infty]{}0$.

    但 $\displaystyle\int_0^\infty 0\dd x=0\ne\lim_{y\to\infty}\int_0^\infty f(x,y)\dd x=1$.

    注意到 $\displaystyle\int_0^\infty f(x,y)\dd x$ 在 $y>0$ 上不一致收敛:

    任取 $B>0$,取 $y=2B$. 则
$$
\int_B^\infty f(x,2B)\dd x=\int_B^{2B}\frac{1}{2B}\dd x=\frac{1}{2}\not\to 0
$$
\end{example}

作为上一性质以及 Dini 定理的推论,我们有

\begin{inference}\label{1722}
    设 $Y\subset\RR,f:[a,\omega)\times Y\to\RR_+$,且 $\forall y\in Y$ 有 $f(x,y)$ 关于 $x$ 连续. 若

    \begin{enumerate}
        \item $f(x,y)$ 关于 $y$ 单增,且 $f(x,y)\to\varphi(x),y\to y^*=\sup Y$.
            
        \item $\varphi(x)$ 在 $[a,\omega)$ 上连续.
            
        \item $\displaystyle\int_a^\omega\varphi(x)\dd x$ 收敛.
    \end{enumerate}

    则 $\forall y\in Y$ 有 $\displaystyle\int_a^\omega f(x,y)\dd x$ 收敛且 $\displaystyle\lim_{y\to y^*}\int_a^\omega f(x,y)\dd x=\int_a^\omega\varphi(x)\dd x$.
\end{inference}
\begin{proof}
    由 Weierstrass 判别法知 $\displaystyle\int_a^\omega f(x,y)\dd x$ 一致收敛.

    由 Dini 定理知对 $\forall b\in[a,\omega)$ 在 $[a,b]$ 上有 $f(x,y)\rightrightarrows\varphi(x)$.

    从而由上一性质,结论成立.
\end{proof}

\begin{example}
    $f_n(x)=f(x,n)=n(1-x^\frac{1}{n}),x\ge 0$.

    已知 $f_n(x)\uparrow\ln\dfrac{1}{x}$.

    且 $\ln\dfrac{1}{x}$ 在 $x>0$ 上连续,$\displaystyle\int_0^1\ln\frac{1}{x}\dd x<+\infty$.

    从而由推论知
$$
\lim_{n\to\infty}\int_0^1n(1-x^\frac{1}{n})\dd x=\int_0^1\ln\frac{1}{x}\dd x
$$
\end{example}

接下来我们研究广义积分关于参数的连续性.

\begin{property}
    设 $f:[a,\omega)\times [c,d]\to\RR$ 连续.

    且 $F(y)\triangleq\displaystyle\int_a^\omega f(x,y)\dd x$ 在 $[c,d]$ 上一致收敛.
    
    则 $F$ 在 $[c,d]$ 上连续.
\end{property}
\begin{proof}
    令 $F_b(y)=\displaystyle\int_a^b f(x,y)\dd x$,则已知 $F_b\in C[c,d]$.

    由条件知在 $[c,d]$ 上 $F_b\rightrightarrows F$.

    从而 $F\in C[c,d]$.
\end{proof}

\begin{example}
    $F(y)=\displaystyle\int_0^\infty\frac{\sin x}{x}e^{-xy}\dd x$ 在 $\RR_+$ 上一致收敛.

    则 $F$ 在 $[0,+\infty)$ 上连续. 特别的,$\displaystyle\lim_{y\to 0+}F(y)=\int_0^\infty\frac{\sin x}{x}\dd x$.
\end{example}

\mysubsection{含参广义积分关于参数求导}

\begin{property}
    记 $R=[a,\omega)\times [c,d]$. 设 $f:R\to\RR$ 满足

    \begin{enumerate}
        \item $f\in C(R),f_y'(x,y)$ 存在且 $f_y'\in C(R)$.
        
        \item $\Phi(y)\triangleq\displaystyle\int_a^\omega f_y'(x,y)\dd x$ 在 $[c,d]$ 上一致收敛.
        
        \item $\exists y_0\in[c,d]$ 使得 $\displaystyle\int_a^\omega f(x,y_0)\dd x$ 收敛.
    \end{enumerate}

    则 $F(y)\triangleq\displaystyle\int_a^\omega f(x,y)\dd x$ 在 $[c,d]$ 上一致收敛,且 $F\in C^{(1)}[c,d]$,且
$$
F'(y)=\int_a^\omega f_y'(x,y)\dd x=\Phi(y)
$$
\end{property}
\begin{proof}
    对 $\forall b\in [a,\omega)$ 令 $F_b(y)\triangleq\displaystyle\int_a^b f(x,y)\dd x$.

    则由第一节的性质知:$F_b\in C^{(1)}[c,d]$ 且
$$
F_b'(y)=\int_a^b\pard{f}{y}(x,y)\dd x
$$

    记为 $\Phi_b(y)$.

    由 $\Phi(y)$ 在 $[c,d]$ 上一致收敛知在 $[c,d]$ 上 $F_b'(y)=\Phi_b(y)\xrightrightarrows[b\to\omega]{}\Phi(y)$.

    由 $\displaystyle\int_a^\omega f(x,y_0)\dd x$ 收敛知 $F_b(y_0)\xrightarrow[b\to\omega]{}F(y_0)$.

    由性质 \ref{limd} 知在 $[c,d]$ 上 $F_b(y)\rightrightarrows F(y)$ 且 $F$ 在 $[c,d]$ 上可导,$F'(y)=\Phi(y)$.

    特别的 $\Phi\in C[c,d]\implies F\in C^{(1)}[c,d]$.
\end{proof}

\begin{example}
    设 $\alpha>0$,则 $\displaystyle\int_0^\infty x^\alpha e^{-xy}\dd x$ 在 $y\ge y_0>0$ 上一致收敛.

    令 $F(y)\triangleq\displaystyle\int_0^\infty e^{-xy}\dd x$.

    则我们我们有 $F\in C^\infty(0,\infty)$ 且
$$
F^{(n)}(y)=(-1)^n\int_0^\infty x^ne^{-xy}\dd x
$$

    注意到 $F(y)=\dfrac{1}{y}$. 从而 $F^{(n)}(y)=(-1)^n\dfrac{n!}{y^{n+1}}$.

    由此可得
$$
\int_0^\infty x^ne^{-x}\dd x=n!
$$
\end{example}

\begin{example}
    计算 $\displaystyle\int_0^\infty\frac{\sin x}{x}\dd x$.

    定义 $F(y)\triangleq\displaystyle\int_0^\infty\frac{\sin x}{x}e^{-xy}\dd x$. 则 $F'(y)=-\displaystyle\int_0^\infty \sin x e^{-xy}\dd x$.

    直接计算得 $F'(y)=\dfrac{1}{y^2+1}\implies F(y)=-\arctan y+c$.

    令 $y\to\infty$ 得 $F(y)\to 0$. 从而 $c=\dfrac{\pi}{2},F(y)=-\arctan y+\dfrac{\pi}{2}$.

    故 $F(0)=\dfrac{\pi}{2}=\displaystyle\int_0^\infty\frac{\sin x}{x}\dd x$.
\end{example}

\mysubsection{关于含参积分的参数积分}

\begin{property}
    设 $f:[a,\omega)\times [c,d]\to\RR$ 连续.

    且 $F(y)=\displaystyle\int_a^\omega f(x,y)\dd x$ 在 $[c,d]$ 上一致收敛.

    则 $F$ 在 $[c,d]$ 上可积,且
$$
\int_c^d\int_a^\omega f(x,y)\dd x\dd y=\int_c^dF(y)\dd y=\int_a^\omega\int_c^d f(x,y)\dd y\dd x
$$
\end{property}
\begin{proof}
    令 $F_b(y)=\displaystyle\int_a^bf(x,y)\dd x$. 则 $F_b\in C[c,d]$ 且 $F_b\rightrightarrows F$.

    则 $F\in C[c,d]$,从而可积,且
$$
\begin{aligned}
    \int_c^dF(y)\dd y&=\lim_{b\to\omega}\int_c^d F_b(y)\dd y=\lim_{b\to\omega}\int_c^d\left(\int_a^b f(x,y)\dd x\right)\dd y\\
    &=\lim_{b\to\omega}\int_a^b\left(\int_c^df(x,y)\dd y\right)\dd x=\int_a^\omega\int_c^d f(x,y)\dd y\dd x
\end{aligned}
$$
\end{proof}

\begin{hint}
    一般来说若没有一致收敛的假设,结论有可能不对. 这可以从以下的反例中推出.
\end{hint}

\begin{example}
    $f:[0,+\infty)\times [0,1]\to\RR,f(x,y)\triangleq(2-xy)xye^{-xy}$.

    此时有
$$
\int_0^1\int_0^\infty f(x,y)\dd x\dd y=0\ne\int_0^\infty\int_0^1f(x,y)\dd y\dd x=1
$$
\end{example}

在函数非负时,我们有如下的性质

\begin{property}
    设 $f:[a,\omega)\times [c,d]\to\RR$ 满足

    \begin{enumerate}
        \item $f$ 连续.
        
        \item $f\ge 0$.
        
        \item $F(y)=\displaystyle\int_a^\omega f(x,y)\dd x$ 在 $[c,d]$ 上连续.
    \end{enumerate}

    则 $\displaystyle\int_c^dF(y)\dd y=\int_a^\omega\int_c^d f(x,y)\dd y\dd x$.
\end{property}
\begin{proof}
    这是 Dini 定理的推论. 我们仅需验证上一定理的条件,即 $F(y)$ 在 $[c,d]$ 上一致收敛.
    
    但由 $f\ge 0$ 知
$$
F_b(y)\uparrow F(y)\implies F_b(y)\rightrightarrows F(y)
$$
\end{proof}

\begin{hint}
    回到上一个例子,此时 $F(y)=\displaystyle\int_0^\infty (2-xy)xy e^{-xy}\dd x=0$ 连续.

    但此时 $f$ 不是非负(或非正)函数,从而导致了交换积分次序后积分值不相等.
\end{hint}

最后,我们再来证明一个关于 $x,y$ 均是广义积分时交换积分次序的充分条件.

\begin{property}
    设 $f:[a,\omega)\times[c,\widetilde{\omega})$ 连续. 且

    \begin{enumerate}
        \item $F(y)\triangleq\displaystyle\int_a^\omega f(x,y)\dd x$ 在 $\forall[c,d]\subset[c,\widetilde{\omega})$ 上一致收敛.
        
        $G(x)\triangleq\displaystyle\int_c^{\widetilde{\omega}} f(x,y)\dd y$ 在 $\forall[a,b]\subset[a,\omega)$ 上一致收敛.

        \item 如下两个积分至少有一个存在
$$
\int_c^{\widetilde{\omega}}\int_a^\omega\abs{f(x,y)}\dd x\dd y,\qquad\int_a^\omega\int_c^{\widetilde{\omega}}\abs{f(x,y)}\dd y\dd x
$$
    \end{enumerate}

    则以下两个累次积分均存在且相等
$$
\int_c^{\widetilde{\omega}}\int_a^\omega f(x,y)\dd x\dd y=\int_a^\omega\int_c^{\widetilde{\omega}}f(x,y)\dd y\dd x
$$
\end{property}
\begin{proof}
    不妨设
$$
\int_a^\omega\int_c^{\widetilde{\omega}}\abs{f(x,y)}\dd y\dd x<+\infty
$$

    我们希望使用第二节的性质. 为此取 $Y=[c,\widetilde{\omega}),\mathcal{B}_Y$ 为 $d\to\widetilde{\omega}$.

    令 $G_d(x)\triangleq\displaystyle\int_c^d f(x,y)\dd y$. 由第一个条件知
$$
\forall b\in[a,\omega),\text{在 }[a,b]\text{ 上 }G_d\xrightrightarrows[d\to\widetilde{\omega}]{}G\quad
$$

    以下验证:$\displaystyle\int_a^\omega G_d(x)\dd x$ 关于 $d\in[c,\widetilde{\omega})$ 一致收敛.

    令 $\widehat{G}(x)=\displaystyle\int_c^{\widetilde{\omega}}\abs{f(x,y)}\dd y$. 由条件知 $\displaystyle\int_a^\omega\widehat{G}(x)\dd x$ 收敛.

    显然有 $\abs{G_d(x)}=\displaystyle\abs{\int_c^df(x,y)\dd y}\le\widehat{G}(x)$.

    由 Weierstrass 判别法知 $\displaystyle\int_a^\omega G_d(x)\dd x$ 在 $[c,\widetilde{\omega})$ 上一致收敛.

    从而由第二节的性质可得
$$
\lim_{d\to\widetilde{\omega}}\int_a^\omega G_d(x)\dd x=\int_a^\omega G(x)\dd x=\int_a^\omega\int_c^{\widetilde{\omega}}f(x,y)\dd y\dd x
$$

    另一方面,由本节第一个性质可得
$$
\int_a^\omega G_d(x)\dd x=\int_a^\omega\int_c^df(x,y)\dd y\dd x=\int_c^d\int_a^\omega f(x,y)\dd x\dd y
$$

    从而广义积分 $\displaystyle\lim_{d\to\widetilde{\omega}}\int_a^\omega G_d(x)\dd x=\int_c^{\widetilde{\omega}}\int_a^\omega f(x,y)\dd x\dd y$ 存在,且
$$
\int_c^{\widetilde{\omega}}\int_a^\omega f(x,y)\dd x\dd y=\int_a^\omega\int_c^{\widetilde{\omega}}f(x,y)\dd y\dd x
$$
\end{proof}

\begin{hint}
    在缺少第二个条件时,结论不一定成立. 反例如下:
\end{hint}

\begin{example}
    设 $A>0$. 考虑 $f:[A,+\infty)^2\to\RR$
$$
f(x,y)=\frac{x^2-y^2}{(x^2+y^2)^2}
$$

    则直接计算可得
$$
\begin{aligned}
    &\int_A^{+\infty}\int_A^{+\infty}f(x,y)\dd y\dd x=-\frac{\pi}{4}\\
    &\int_A^{+\infty}\int_A^{+\infty}f(x,y)\dd x\dd y=\frac{\pi}{4}
\end{aligned}
$$

    注意到对 $f$ 条件 1 成立而条件 2 不成立.
\end{example}

\begin{example}
    作为一个正面的例子,设 $\alpha,\beta>0$.

    定义 $f(x,y)=x^\alpha y^{\alpha+\beta+1}e^{-(1+x)y}$.

    则我们之前已经验证了
$$
F(y)=\int_0^\infty f(x,y)\dd x,\qquad G(x)=\int_0^\infty f(x,y)\dd y
$$

    在 $\RR$ 上一致收敛. 另一方面
$$
\int_0^\infty\int_0^\infty f(x,y)\dd x\dd y=\left(\int_0^\infty e^{-y}y^{-\beta}\dd y\right)\left(\int_0^\infty(xy)^\alpha e^{-xy}\dd(xy)\right)<+\infty
$$

    故上一性质的两个条件均满足,从而
$$
\int_0^\infty\int_0^\infty f(x,y)\dd y\dd x=\int_0^\infty\int_0^\infty f(x,y)\dd x\dd y
$$
\end{example}

\begin{inference}
    设 $f:[a,\omega)\times[c,\widetilde{\omega})\to\RR$ 连续且 $f\ge 0$. 若

    \begin{enumerate}
        \item $F(y)=\displaystyle\int_a^\omega f(x,y)\dd x$ 与 $G(x)=\displaystyle\int_c^{\widetilde{\omega}} f(x,y)\dd y$ 均连续.
        
        \item $\displaystyle\int_c^{\widetilde{\omega}}F(y)\dd y$ 或 $\displaystyle\int_a^\omega G(x)\dd x$ 存在.
    \end{enumerate}

    则另一个也存在,且 $\displaystyle\int_c^{\widetilde{\omega}}F(y)\dd y=\int_a^\omega G(x)\dd x$.
\end{inference}
\begin{proof}
    我们来验证上一性质的条件. 令
$$
F_b(y)=\int_a^bf(x,y)\dd x,\qquad G_d(x)=\int_c^d f(x,y)\dd y
$$

    由 $f\ge 0$ 知 $F_b(y)\uparrow F(y),G_d(x)\uparrow G(x)$.

    由 $F,G$ 连续知 $\forall [c,d]\subset[c,\widetilde{\omega}),F_b(y)\rightrightarrows F(y)$ 且 $\forall[a,b]\subset[a,\omega),G_d(x)\rightrightarrows G(x)$.

    从而上一性质的两个条件均成立. 即证.
\end{proof}

\begin{example}
    计算 $\displaystyle\int_0^\infty e^{-x^2}\dd x=\frac{\sqrt{\pi}}{2}$.

    令 $\Gamma\triangleq\displaystyle\int_0^\infty e^{-u^2}\dd u$. 令 $u=xy,y>0$,则
$$
\Gamma=y\int_0^\infty e^{-(xy)^2}\dd x,\forall y>0
$$

    进而
$$
\begin{aligned}
    \Gamma^2&=\int_0^\infty Je^{-y^2}\dd y\\
    &=\int_0^\infty\left(y\int_0^\infty e^{-(xy)^2}\dd x\right)e^{-y^2}\dd y\\
    &=\int_0^\infty\int_0^\infty ye^{-(1+x^2)y^2}\dd x\dd y
\end{aligned}
$$

    令 $f(x,y)\triangleq ye^{-(1+x^2)y^2}$. 令 $F(y)=\displaystyle\int_0^\infty f(x,y)\dd x,G(x)=\int_0^\infty f(x,y)\dd y$.

    则 $\displaystyle\int_0^\infty F(y)\dd y=\Gamma^2<+\infty$.

    另一方面,计算得到 $F(y)=e^{-y^2}\Gamma,G(x)=\dfrac{1}{2}\dfrac{1}{1+x^2}$ 在 $[0,+\infty)$ 上连续.

    从而由推论知
$$
\Gamma^2=\int_0^\infty F(y)\dd y=\int_0^\infty G(x)\dd x=\frac{1}{2}\int_0^\infty\frac{\dd x}{1+x^2}=\frac{\pi}{4}
$$

    即证 $\displaystyle\int_0^\infty e^{-x^2}\dd x=\frac{\sqrt{\pi}}{2}$.
\end{example}